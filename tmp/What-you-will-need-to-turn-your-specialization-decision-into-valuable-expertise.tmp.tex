\chapter{What You Will Need to Turn Your Specialization Decision Into a Valuable Expertise}

You may be specializing for one or both of two reasons: 1) to make your efforts to connect and build trust with ideal prospects yield better results and 2) to make it possible to cultivate exceptionally valuable expertise. It's possible to focus on the first reason and avoid the second (remember, that’s what I call superficial specialization), but I think the deepest rewards of specialization--both economically and personally--accrue to those who pursue both reasons for specializing. This chapter is an introduction to how you might use time and focused effort to turn your specialization decision into economically valuable expertise.

Ideally, you'll recruit all of these six allies as you work too cultivate deep expertise:

\begin{enumerate}
\item A focus
\item A vision
\item A method
\item An accelerant
\item Resilience
\item Flexibility
\end{enumerate}

Across many interviews with self-made experts (\href{http://consultingpipelinepodcast.com}{http://consultingpipelinepodcast.com} and a new body of work I’m building at \href{http://theselfmadeexpert.com}{http://theselfmadeexpert.com})—aside from a reasonably high risk profile--these are the characteristics I most commonly observe. I refer to them as allies because what you're seeking to do when you cultivate deep self-made expertise is not easy, and we all need all the help we can get on that journey.

\section{A Focus}

Your specialization decision might create enough focus such that a disciplined pursuit of that focus yields over time deep expertise. But, to be honest, it's unlikely because of the nature of impactful expertise.

If I had to limit myself to one observation about impactful expertise, it's this: the kind of expertise that creates business impact has a keen awareness of \emph{context}. It acknowledges the importance of and seeks to understand the \emph{context in which skill is applied} in order to create some desired outcome or group of outcomes.

In other words, you can't be an expert who ignores context. If you are, you'll fit most of the negative or wrong stereotypes of specialized expertise. And even from folks who should know better, I've run across plenty of these. For example:

\textless{}blockquote class=''twitter-tweet'' data-lang=''en''\textgreater{}\textless{}p lang=''en'' dir=''ltr''\textgreater{}If you'{}re a specialist the only thing you can really do is optimize. This also makes you fragile to a (rapidly) changing environment.\textless{}/p\textgreater{}---{} Shane Parrish (@farnamstreet) \textless{}a href=''https://twitter.com/farnamstreet/status/1076960397209432064?ref\_src=twsrc\%5Etfw''\textgreater{}December 23, 2018\textless{}/a\textgreater{}\textless{}/blockquote\\\\\\\\\\\textgreater{}
\textless{}script async src=''https://platform.twitter.com/widgets.js'' charset=''utf-8''\textgreater{}\textless{}/script\\\\\\\\\\\textgreater{}

\href{https://twitter.com/farnamstreet/status/1076960397209432064}{https://twitter.com/farnamstreet/status/1076960397209432064}



There's this idea that specialization leads you inevitably to a fragile, out-of-touch position. This could happen, I suppose, but I've almost never seen it happen in real life, except when someone has pursued a platform specialization, got complacent, and then allowed their skills to age out of relevance. I certainly understand how someone could imagine a fragile, out-of-touch position being the inevitable outcome of specialization, but I'd also have to ask the person suffering this fear two things: 1) do you know any highly paid specialists, and 2) how addicted are you to chasing shiny objects instead of business impact? I'm not asking if you \emph{know of} highly paid specialists, but actually \emph{know any of them personally}? If you don't then you run the risk of imagining what their life is like rather than knowing what it actually is like.

I spent almost the first two decades of my career chasing shiny objects of a technical nature. So trust me, I'm not throwing rocks here at you if you are currently addicted to chasing shiny objects. It's fun and hurts almost nobody. And furthermore, it often is a flat out good thing to do for a while. It gives you at least a thin layer of broad technical context within which to better develop focused expertise. So a few years of chasing shiny objects, especially at the outset of a career? There's far more good than bad in that! But 20 or 40 years of chasing shiny objects because you can't settle down, focus, and make yourself into a valuable expertise asset to businesses that need that expertise? That's a recipe for a disappointing career, one marked by an intimate understanding of the meaning of the word ageism.

If the defining characteristic of expertise has to do with context, and if chasing shiny objects--which can be seen as a sort of broadening of context--for your entire career is bad, then how do these two ideas fit together? It has to do with \emph{time}, or rather with how you deploy an ability fo focus over time. It looks like this:

\{illustration with a hypodermic needle injecting medicine into some tissue. The needle is labeled ``initial specialized focus'', the contents of the syringe ``contextually-aware expertise'', and the tissue effected by the injection is labelled ``business impact''.\}

If your goal is to create economically valuable business impact, then every component of the system pictured above is critical, and the removal of any component will fail to achieve business impact. The reality, though, about cultivating deep expertise is that you don't start with every component of this system. You build them sequentially over time, starting with the needle. Allow me to explain.

Consider what happens if you start \emph{without} a needle. If you have the syringe barrel that contains the medication but no needle, then you could kind of spray the medication against the surface of the patient's skin (the epidermis), but that would be completely ineffective at delivering the medication to the place it needs to go. And--heaven forbid--you could grit your teeth and try to plunge the broad end of the needle-less syringe barrel into the patent's skin, creating a half-inch diameter hole in 2 or 3 layers of skin in the process. You'd probably give up before getting past the dermis and the patient would probably be halfway through successfully suing you for everything you've got before the wound heals, but hey, you could try! So, the needle is vital. It's the right tool for going deep in the patient's tissue, or reaching a vein/artery, with minimal pain and tissue damage. The needle in this analogy is your initial specialized focus. You cannot start without the needle.

At first, this initial specialized focus might be a specialized application of skill. In other words, you may have an ``empty syringe'', because you haven't seen enough repeated applications of the same skill to develop the contextual awareness that creates real expertise. But over time, you will.

Over time, you'll see the ways in which skill is not enough to create the desired impact. You'll learn from others who guide your skill with expertise. You'll see common failure points and learn to see them coming so you can avoid them in the future. You'll get curious about what might create greater business impact, and this curiosity will lead to self-directed learning or research that expands your understanding of the context in which your skills operate. You'll learn what stuff your clients tend to be blind to, or what assumptions turn out to be harmful later in the project when they're harder to correct.

So over time, you will fill the barrel of your ``syringe'' with what I call a complete solution, which is a blend of skill and contextual understanding that's optimized for maximum business impact on an important business issue.

Here's how this works over time. It might be possible to start with the syringe and the serum inside it, but it doesn’t actually work that way.

You start by getting a ``needle''. That's your \textbf{decision about how to specialize}. You spend time building up a market position based on this initial choice of what your ``needle'' looks like.

For a while, you are walking around with a syringe that has a needle, but not much in the way of medicine inside it. You could think of it as a very elementary medicine, mostly made up of skills. This is not to dismiss the importance of skill, but remember that skill alone doesn't usually achieve the most valuable forms of business impact. It's the deployment of skill guided by deep expertise that achieves that impact. So to extend the analogy, your medicine in the needle is \{TODO: example of well-known outdated medicine\}.

So if you're like most experts I've interviewed, you become dissatisfied by the gap between what your current skill can do and the impact you'd like to be able to create. And so you focus on filling the barrel of your syringe with more expertise. You don't change the needle. That would be chasing shiny objects. Instead, you focus on \emph{more}. More depth of understanding of your clients and their business. More nuanced understanding of how to apply your skills within the context of your clients business. A broader understanding of how to mitigate risk and maximize opportunity while applying your skills. You recruit other forms of skill or expertise necessary to achieve the impact you seek. And in so doing, you develop a proprietary ``medicine'' that fills your ``expertise syringe''.

And when this medicine is delivered to the right area, it creates valuable impact. It gets to work improving the patient's condition.

Now that we have this idea pretty well established, I want to return to the idea of \emph{focus}. Earlier I said your specialization decision might create enough focus such that a disciplined pursuit of that focus yields over time deep expertise but, to be honest, it's unlikely because of the nature of impactful expertise. I said this because the work of developing a proprietary medicine to fill your expertise syringe can be the kind of work that takes a decade or more, though you can make meaningful progress in about the time it takes to earn a master's degree.

The pursuit of deep expertise will often lead you on ``side quests''. This is distinctly different than chasing shiny objects, though it might look similar at first. Chasing shiny objects is spontaneous, unplanned, and undisciplined. Side quests are a focused, disciplined, intentional search for complementary skill, expertise, or data that helps you achieve greater impact in your client work. These side quests are guided by your clear, narrow focus.

Here's where the element of time becomes important. You'll choose a needle by deciding how to specialize. At this point, you won't be aware of everything that's needed to cultivate deep expertise. It's literally not possible, because you haven't used your specialization needle to get past the epidermis of the problem, so you simply can't see the full extent of the problem. But your needle gives you access to those deeper layers of the problem. As it does, you start to see how much more difficult, or complex the problem really is.

At this point you can either give up entirely, retreat to a more superficial layer, or get serious about solving the deeper problem. Even if you decide to get serious about solving the deeper problem (and I hope you do, because the world needs more self-made experts like you!), you will spend some time living in that satisfaction gap I mentioned earlier. You'll spend some time knowing that there's a problem that you have access to but can't solve to your full satisfaction.

This dissatisfaction will lead you to organize and pursue side quests, in search of the context, skill, complementary expertise, or data you need to really solve the deeper problem. It is during this phase of your career that you'll need even more discipline and discernment, because you now have a focus problem all over again.

You thought it was over when you made that initial courageous specialization decision! Again, if you were simply pursuing marketing efficiencies, it might have ended there. But you're pursuing deep self-made expertise that creates valuable business impact. And there's almost certainly no curriculum to guide you as you do this. So now not only do you have to learn a lot of stuff, you have to invent the curriculum that guides your own learning!

If you're like me, this kind of challenge gets you out of bed in the morning, excited about the day ahead. It makes me happy to wake up a few hours before my body is ready to so my mind can take advantage of those quiet early morning hours for writing, research, etc. It brings out the best in you. And you feel a sense of obligation to bring the best of yourself to this challenge. And so you'll think carefully about focus, prioritization, and maintaining a healthy average velocity as you pursue this challenge. Occasional sprints, yes, but most importantly you want to keep your average velocity steady through this expertise-building phase of your career.

If you could see the entire project of building your deep self-made expertise all at once, it might feel like ``eating an elephant''. But in reality it's much more likely to unfold one piece at a time over multiple years. So you choose to focus on the side-quest that has just unfolded before you. But occasionally, you'll be presented with several potential side-quests. Should you pursue them all at once in parallel, or should you prioritize and pursue them serially? I can't answer that definitively for every possible person and situation, but in general, you might use the following to prioritize:

\begin{itemize}
\item Which side-quest will be most impactful on your ability to improve your clients condition? You might look at this through the perspective of risk mitigation, or maximizing opportunity.
\item Which side-quest would most increase your ability to deliver value to your clients with the least amount of work on your behalf? In other words, is there some sort of Pareto Principle 80/20 opportunity you could pursue?
\item Which side-quest would address the most significant client sales objections? In other words, is there a side-quest that leads to expertise that makes it possible to sell more impactful, ambitious projects, or move from the vendor position to the expert practitioner position during the sale?
\item What side-quest would put you in an advantageous position 3, 5, maybe 10 years from now? In other words, which one is the best investment in future advantage?
\end{itemize}

An ability to deploy disciplined focus is advantageous at every stage of your career. As you build your expertise syringe, this ability to focus helps you choose the right needle, and focus then helps you organize the side-quests necessary to invent the most potent medicine possible to fill the syringe barrel so you can create deep, valuable client impact.

\section{A Vision}

Back to our missionary-mercenary spectrum.

I suppose the Platonically maximized form of the mercenary type would be one who is seeking opportunity based on the opportunity's ability to yield a financial profit with no regard to anything else. I've never met anyone like this, so I'm not confident they actually exist in this pure form. In other words, even the most mercenary business owner is going to express a desire for things beyond pure profit. And even the most missionary business owner is going to need to turn a profit eventually, even if that's way down their list of needs relative to a need for other things like meaning, connection, or impact.

When it comes to cultivating really valuable expertise, it helps to have a significant portion of the missionary type in your personality makeup. This is because the journey to economically valuable expertise is not easy, and it may be temporarily less profitable than other ways of deploying the same amount of time and effort. You need to be able to look beyond the short term in order to pursue this path.

Missionaries are more naturally inclined to organize their work around some kind of vision: an imagined and desired future state. The missionary vision tends to be a vision for how things could be better for a group of people the missionary cares about. If this vision is deeply informed by the actual desires or needs of that group, then the vision is a generous and relevant one. And if this vision is achievable, then it is a realistic one.

In my work with self-made experts, I've found this kind of vision to be a powerful component in the cultivation of expertise. It does several things:

\begin{enumerate}
\item Gives you \textbf{more than just a financial motivation} for cultivating expertise. This leads to a greater resilience as you pursue the vision.
\item Because you are interested in more than just profit, you may be \textbf{willing to endure lower short-term profitability} in exchange for greater long-term impact, which can be a good tradeoff.
\item A vision helps you \textbf{identify and choose the specifics} of how you will work towards realizing the vision. Often referred to as a mission or tactics, these specifics become more effective when they're organized around a singular, specific goal (the vision).
\end{enumerate}

A simple but powerful way to understand the role a vision plays in your cultivation of expertise would be to ask yourself--I mean \emph{really} ask and \emph{deeply} ponder--the following question:

\textbf{What is your expertise for?}

If that question stumps you with its apparent simplicity, then ask yourself these more specific variations:

\begin{itemize}
\item Who would benefit the most from the expertise I could cultivate? How exactly might they benefit?
\item Of all the business people out there who have a budget to spend on professional services, what one single person would I most want to help? If I could give them the gift of my expertise and meaningfully improve their life (they sleep better, they feel less stress, they get a bonus because of hiring me, their boss compliments/promotes them, they do a better job as a manager, the people who work for them benefit, their company performs better because they chose to hire me, their company's clients or customers benefit, etc.), who would that person have to be to benefit from the gift of my expertise?
\item What observable changes happen in the world or in a very small part of the world as a result of the application of expertise I could cultivate? In other words, what new possibilities for other people do my expertise unlock?
\item Alternately, how would the world be worse off 5, 10, 25 years from now if I didn't cultivate this expertise?
\item In what situation would the expertise I could cultivate be \emph{wasted}? And does that lead me to a vision for where it would be most useful or valuable?
\end{itemize}

Your vision for the role your expertise could play is going to be deeply personal. It won't be about you because self-centered visions are pretty useless gifts to the world, but your vision will be deeply informed by who you are, your life experiences, and what's important to you. It won't be about you, but it will emerge from you. I can't say much about \emph{your} vision specifically, but I can say the following generally:

\begin{itemize}
\item It's worth the work it takes to cultivate a vision for \textbf{what your expertise is for}. It keeps you going when the going gets tough.
\item If in doubt, \textbf{cultivate the most generous version} of your vision you can. Even if it doesn't work out financially, you'll have fewer regrets if you choose generosity, and you’ll become a person you can love more.
\item As you cultivate a vision, \textbf{do lean out over your skis}, but not so far that you crash. Seeing no meaningful progress towards a vision that's important to you is flat out discouraging. Working towards a vision that could be fully realized in a year is too small a challenge to bring out the best in you. There's going to be a happy medium that's different for different folks. Choose a level of ambition that excites you--maybe even scares you a little--but also brings out the best in you.
\item Finally, cultivate a vision that is relevant to those you seek to serve—one that is grounded in their needs or desires, not yours. If following this maxim leads you to an overly ambitious, unrealistic vision, reduce the scope of who you seek to serve \emph{before} you scale back the impact you hope to create. In other words, go more narrow on who it is you want to serve before you lower your expectations about how much you can help them. In fact, avoid at all cost lowering your expectations about how much you can help. It will be far more satisfying to deeply impact 1000 people than to superficially impact 10,000, and it's significantly easier to do connect and build trust with 1000 than 10,000.
\end{itemize}

Finally finally, notice that I'm most often talking about your vision effecting \emph{people}. Now in reality, your vision needs to be relevant to \emph{people who have the ability to pay you for delivering services that are profitable to you}. So I definitely mean for you to cultivate a vision that draws revenue, profit, and business opportunity towards you. But at the same time, \emph{people} need to be central to your vision, because ultimately it's \emph{people} who need to resonate with and be inspired by your vision when they're exposed to it. Thinking in terms of people rather than businesses tends to make your thinking more specific and less abstract, and that will work out better for you in the end.

\subsection{Cultivating a Vision}

I can't leave this topic without a few notes on how exactly you might cultivate a vision. I know you're waiting with bated breath here. Am I about to suggest an ayahuasca-powered vision quest in a jungle? A sweat lodge ceremony? Dream interpretation?

Sure, if those things have a track record of leading you to useful ideas, then go for it! But for most of us, there are two more practical, market-first methods we might use.

\textbf{Ask Your Clients}

I know you're already rolling your eyes, because you think you know what your clients need and want. But I'd bet in most cases, your insight into what they need or want is constrained precisely to the contours of the project they hired you to help with. You know a lot about the project, and relatively little about the people behind it.

Here, in ascending order, is what insight into your clients needs and desires looks like:

\begin{enumerate}
\item Understanding the \textbf{scope} and other details of the project they hired you to help with.
\item Understanding \textbf{why they defined the project} the way they did.
\item Understanding \textbf{what they hope the project will achieve}.
\item Understanding \textbf{why they care} about \#3.
\item Knowing \textbf{who participated in the decisions} leading to this project being commissioned and funded.
\item Knowing \textbf{what second-order consequences} could be created by this project in the context of the specific client you're working with.
\item Knowing \textbf{how and why these second-order consequences could be good or bad or neutral} in the context of your client's business.
\item Understanding the \textbf{common patterns of relationships} between people, business process, customers, money, and software across many of your clients. In other words, you understand how their business works.
\item Understanding the \textbf{common patterns of decision-making and behavior} across most job functions across most of your clients. In other words, you can reasonably accurately model \emph{why} almost anybody at a client company would decide to do what they actually decided to do.
\item Understanding the \textbf{common patterns of business performance} across most companies within your area of focus. In other words, you can reasonably accurately model why the best or worst performing companies within you area of focus are performing as they are. You get lots of bonus points if you can not just model backward in time (figuring out why the things that worked actually worked) but also forward (knowing what investments are most likely to pay off).
\end{enumerate}

How high on this scale does your insight into your clients go? If it doesn't go all the way to number 10, then you don't have enough insight into your market to cultivate the most powerful possible vision. Perhaps you'd like to change that?

Then ask your clients. :)

As you do this you may be entering uncharted territory. You may be finding yourself talking about stuff that has nothing to do with your current project, and that may feel uncomfortable at first. And of course, if you are at level 3 of insight, then you don't want to try to jump straight to level 10. In fact, you might consider each of these 10 levels as its own milestone. So if you're at level 3, consider just focusing on gaining a level 4 understanding. Even that increase in insight can pay considerable dividends.

Notice that as you ascend this scale of insight, what changes is that you understand greater and greater amounts of \emph{context} around the project that drew you into working with your client. You can think of it like Matryoshka dolls, those Russian dolls where progressively smaller dolls nest inside the larger ones. Being the outermost ``Matryoshka doll'' gives you the most credibility in influencing your client's decision-making.

\textbf{Ask Your Leads}

The second market-first method you might use for cultivating a vision is to interact with your leads, specifically with an email list of leads.

Email marketing--even done very simply by broadcasting to a non-automated, unsegmented list of leads--has a unique power because it combines scale and intimacy. My friend and colleague Jonathan Stark first pointed out this unique property of email marketing to me. You can easily reach thousands of people with a single email, but that scalable communication converts instantly into what is often a very private, intimate conversation as soon as a list member replies to an email from you.

This offers you the ability to gain valuable insight about your leads directly from your leads. How? Again, you ask them. Or, if you're like me, you persistently and gently \emph{provoke them}, and the provocation leads to conversation. You provoke them with a vision of what's possible for their business. You provoke them by reminding them of the pain they're living with but have the power to change. And finally, you provoke them by being persistently focused on these things, but approaching them from a stimulating, engaging variety of perspectives using a variety of content forms (teaching, humor, inspiring stories, unexpected associations, etc.).

Over time, a pretty high resolution picture of what your market needs, desires, fears, hates, and loves emerges from this kind of intentional interaction with a list of email leads. And with this picture in view, you are better able to formulate a generous, relevant vision for how your expertise could benefit your list of leads. And again, that vision helps you along the sometimes difficult journey towards cultivating real expertise.

\section{A Method}

Focus and vision are important, but by themselves they are not enough to allow you to cultivate valuable expertise. You also need a \emph{method} for cultivating that expertise.

How will you actually come to understand greater and greater amounts of the context around your area of focus? How will you learn about the potential and actual second-order consequences that flow from your client work? How will you learn more about the patterns of relationship, decision making, and business performance that surround the projects you work on?

I could say somewhat unless things like ``be curious'' or ``always be asking questions'', but I can do better. There are two specific methods you can use to cultivate valuable expertise. They are 1) to seek ambitious client work, and 2) to perform self-directed research.

TODO: Explain what kind of ambitious client work you're looking for: leads you higher up that ladder of greater context. Specifically, if you're moving up the org chain of command or getting involved earlier in the process of defining and executing a project, you're probably moving up the ladder of expertise. Explain why this upward movement can lead to the cultivation of expertise.

If you're used to measuring the value of client work in terms of financial profitability, don't stop doing that because it's a great way to think about client work, but also consider that when your goal is to cultivate valuable expertise, there are other, non-monetary forms of value that certain client work may create for you. Seeking ambitious client work might involve \emph{temporarily} taking on client work that is less profitable or is unlike the client work you would normally seek because it delivers pays some expertise-related dividend. Learning to see these kinds of expertise-building opportunities is valuable, so I'll spend some time elaborating on the non-monetary value of client work.

\subsection{The Strategic Value of Client Work}

TODO: adapt the strategic value of client work piece here

TODO: infuse some “everything is an opportunity” thinking + Blair’s ideas about creative ways to get paid (licensing, equity, etc.)

Even if you completely agree with my suggestion that you seek ambitious client work and think more broadly about what kind of projects might have long-term expertise-building value for you, the blocker tends to be actually finding that more ambitious client work. It's not enough just to want it, you have to connect and earn trust from the clients who can hire you to do that work!

\subsection{Finding Ambitious Client Work}

TODO

\begin{itemize}
\item \textbf{Pick a specific vector of improvement}: start with your existing skill advantage and add the ``spice'' of lower risk, greater reliability, greater or faster ROI, greater effectiveness, etc. What would that combination look like? Be both ambitious and careful. Probably stick to one primary vector of improvement. ``What would a smartphone app with only 4 and 5-star reviews look like?'' ``What would a marketing campaign that paid for itself in 2 months look like?'' ``What would a web app that increased your NPS in 6 months look like?'' This is the dream you are selling.
\item \textbf{Sketch out a plan of attack}: If there were no budget constraints, how would you pursue this vector of improvement? If you had access to all the world's skill and expertise, how would you pursue this improvement? Dream big, but temper your dreaming with a bit of realism. Even with unlimited money and expertise resources, things will go wrong. How will you plan for these inevitable failure points? A contingency plan is part of this kind of dreaming big.
\item \textbf{Define 2 or 3 service tiers}: If your plan of attack were packed into 2 or 3 tiers ranging from low-touch to high-touch, what would those tiers look like? You can almost certainly define the lowest touch version as a ``workshop'', where you and your client get together for between a half day and two days to hash things out and move from relative confusion towards relatively greater clarity, and the deliverable is some sort of action plan. Paid research could also be a good lowest tier. You can almost certainly define the highest-touch tier as implementation of a solution combined with some form of ``insurance'' on the implementation, like a money back guarantee, or your ongoing involvement to insure against downstream failures or unanticipated second-order consequences, or deeper research, or more of something that lowers risk. Sometimes a middle tier makes sense, either because the cognitive by-products of a 3-tier proposal make it easier for prospects to choose the middle tier, or because the actual solution naturally divides into 3 tiers of progressively more complete, robust solutions.
\item \textbf{Talk it up}: Share the dream and your thinking about how to achieve it with relevant strangers, leads, and prospects as frequently and consistently as you can. Turn it into spitballed ideas during sales conversations or conversations with peers, content marketing, calibrated questions for leads (ex: “if we could guarantee X, what would it be worth to you?”), networking inquiries to relevant strangers (including competitors), and anywhere else you can. Infuse the dream into every client/prospect touchpoint you can, or alternately focus on building a vision for this dream only with a certain segment of your audience or certain types of prospective clients. Either way, be persistent and consistent about putting it out there in the world.
\end{itemize}

There's no guaranteed way to find more ambitious client work, but the framework above is the one I've used with good success, and so it's worth trying.

\subsection{Self-Directed Research}

Self-directed research is the second practical method for finding more ambitious client work. Well, not self-directed research alone. But self directed research combined with sharing what you find with the right audience is a practical method for finding more ambitious client work.

If you're anything like most software developers, you've done tons of self-directed research. It's simply been focused on acquiring new skills, or alternately, figuring out why something in the fast-changing world of software doesn't work the way it's supposed to. What if you expanded that well-honed ability to do research and directed some of it at climbing the ladder of expertise? What might that look like?

You would probably investigate one or more of the following areas of inquiry:

\begin{itemize}
\item \textbf{What makes some projects succeed and others fail} in terms of ROI, second-order consequences, or other important measures of success? You're looking beyond the code itself to the \emph{context} in which the software operates. The people, relationship, and business context.
\item \textbf{What are the common ``unknowns'' that clients solve or guess their way through individually} that could be solved for a group or category of clients through a research project? In other words, if Companies A, B, and C, are figuring out basically the same question individually, why not figure it out for all three of them through a research project you initiate and run? Candidates for these unknowns will most certainly include:


\begin{itemize}
\item What makes this kind of software succeed?
\item What do our users want/need but have difficulty communicating to us for any reason?
\item What kind of risks will interfere with this software, this project, or both? What kinds do we know about but underestimate, what kinds do we know about and overestimate, and what kinds do we not know about but should?
\item What have others in our space learned but we don't know about? Are there emerging best practices or other patterns that we would benefit from if only we knew about them?
\item What do the top performers in our space do? What do they avoid doing? What relationship does their behavior have to their performance?
\item What is happening outside our space but will effect businesses in our space soon enough that we need to know about it?
\item What opportunities for efficiency or business impact or profitability could we take advantage of if we had better data or know-how?
\item What opportunities for reduced frustration or increased morale could we take advantage of if we had better data or know-how?
\item What academic research is being done but lacks a bridge to the business world? How could I help build that bridge and take partial credit for doing so?
\end{itemize}
\end{itemize}

There is some overlap between these questions, and that's intentional. I wanted to ask them in a way that spurs thinking for you, and the way several of them approach the same issue from slightly different angles hopefully helps with that.

What I'm proposing here is simply a form of service (“serve, don’t sell!”) to your target market based on collecting, organizing, and sharing useful information with them. I wonder if that changes your concept of research? As you consider how you might use research to find more ambitious client work, you're really answering these two questions for yourself:

\begin{enumerate}
\item As I climb the ladder of expertise, I'll interact with buyers who are higher up the org chart. What do these buyers care about? What worries them? What would address those cares and worries?
\item If my goal was to demonstrate my ability to deliver a reliable, effective solution to their needs and concerns, what gift of organized, useful information or data-based insight could I give these buyers? What would lead them to say ``Wow, if this person did \emph{this}, they must be a really good option for building a software solution to that same problem!''
\end{enumerate}

As with most of the specific activities I recommend, this kind of research has two benefits. It has a business benefit. While this isn't guaranteed because there are lots of variables that effect the exact business outcome, it can benefit your business. But the second benefit is to \emph{you}. What you learn and the mental and human connections you build by doing this kind of research can be very valuable. And the disciplined objectivity that you cultivate from doing research can also be very personally beneficial.

A final note on research. Like mine used to be, your mental picture of research is probably heavily influenced by science and academia. While peer-reviewed, statistically valid research is the baseline in those fields, you do \emph{not} need to live up to the same standard in order to create value for yourself and your prospective clients.

The book that changed my thinking about all of this is called \emph{How to Measure Anything}, by Douglas Hubbard. I'll summarize a few of the key point below, and if those pique your interest then do give Douglas' book a read. It'll give you a much more useful view of how research and measurement work in the world of business.

\begin{itemize}
\item \textbf{Measurement reduces uncertainty}: This definition of measurement makes it possible to view the whole idea of measurement differently because instead of trying to achieve an artificially high level of precision, you are simply looking at measurement as a way to reduce uncertainty. This allows us to think of many many things that we might have previously considered to be beyond measurement as phenomenon that we can measure.
\item \textbf{The rule of five}: There is a 93.75\% chance that the median of a population is between the smallest and largest values in any random sample of five from that population.
\item \textbf{Single Sample Majority Rule} (i.e., The Urn of Mystery Rule): Given maximum uncertainty about a population proportion— such that you believe the proportion could be anything between 0\% and 100\% with all values being equally likely— there is a 75\% chance that a single randomly selected sample is from the majority of the population.
\item One way to underestimate the amount of available data is to assume that only direct answers to our questions are useful.
\item Mathematically speaking, when you know almost nothing, almost anything will tell you something.
\item The first few observations are usually the highest payback in uncertainty reduction for a given amount of effort.
\item Don’t assume that the only way to reduce your uncertainty is to use an impractically sophisticated method. Are you trying to get published in a peer-reviewed journal, or are you just trying to reduce your uncertainty about a real-life business decision? (Hint: it’s the latter :) )
\item Prior to making a measurement, we need to answer the following:


\begin{itemize}
\item What is the decision this measurement is supposed to support?
\item What is the definition of the thing being measured in terms of observable consequences and how, exactly, does this thing matter to the decision being asked (i.e., how do we compute outcomes based on the value of this variable)?
\item How much do you know about it now (i.e., what is your current level of uncertainty)?
\item How does uncertainty about this variable create risk for the decision (e.g., is there a “threshold” value above which one action is preferred and below which another is preferred)?
\item What is the value of additional information?
\end{itemize}
\end{itemize}

Again, the list above is my summary of some of the key points in Douglas Hubbard's book, \emph{How to Measure Anything}. I hope this gives you a new, more accessible idea about research. Statistical validity and big sample sizes are good things, but if you know literally nothing about a consequential matter, then \emph{any data}, even if it's not statistically valid or comes from only a few observations, is potentially useful. Direct, exact measurements are a good thing, but if you know literally nothing about a consequential matter, then imprecise measurements, or finding proxies for the thing you want to measure are both potentially useful. Overall, your goal is not 100\% certainty, it's \emph{reduced uncertainty}.

\section{An Accelerant}

Cultivating vegetables in a garden may benefit from fertilizer or soil amendments. Likewise, I think your project of cultivating exceptionally valuable expertise benefits from an accelerant of some sort. Ambitious client work and research are your methods and even without an accelerant they can work, but what if you want to speed things up? What else might serve as an accelerant?

Part of the reason I placed this chapter near the end of the book is because I'm about to propose something that most of you will find so crazy that I fear you'll put the book down and never pick it up again, and I wanted you to get through most of it before that happens. But I assure you what I propose is not crazy, it's simply unusual and audacious.

The accelerant I propose is made up of 3 elements:

\begin{itemize}
\item Build a relevant audience. It can be very very small.
\item Work in public for the benefit of that audience.
\item Publish for that audience at high frequency.
\end{itemize}

Like sulfur, charcoal, and potassium nitrate become gunpowder when combined, the 3 things above become an accelerant when combined. I think omitting any one of them seriously degrades the accelerant's potency.

\subsection{Build a Relevant Audience}

The purpose of building an audience is to get you out of your own head and into conversation with people who could benefit from your self-made expertise. You can think of expertise as applied skill, and the place where the application ultimately matters is on a client project. But, regular interaction with potential clients is also an opportunity to cultivate expertise. It's not as real as a client project, but it's far more real than thinking through things in isolation.

The absolutely most effective way to interact with this kind of audience is through email, so I strongly suggest this audience be built as an email list. It does not have to be a large email list. I'd gladly trade a highly engaged email list of 50 people for 5,000 who mostly ignore what I publish. What you want is not a large list, but a vocal one. One full of people who speak up when they disagree or don't understand what you're saying. Who send you a virtual ``hell yeah!'' when you touch a nerve in a positive way, and a ``wow, that really hurt but thanks for saying it'' when you touch a painful nerve. Who send you questions that relate to and enrich your thinking. This doesn’t mean they have to be extroverts, it just means you need to create a challenging but “safe” space in which they can click REPLY on an email you’ve sent them and tell you what’s on their mind.

\subsection{Work in Public}

Working in public means exposing your thinking to the audience you've built. It means not hiding your in-progress thinking from them until it's fully baked, but instead engaging them in the process of cultivating your expertise. It means occasionally being wrong and later correcting yourself. Again, you're doing this in public, meaning in full view of your email list.

The purpose of working in public is to raise the stakes and create for yourself a sense of productive discomfort. I fully realize this is creating a seeming contradiction for most of my readers.

In your personality, most of you combine a strong element of drive or dominance (I'm using DiSC personality assessment terminology here) with an equally strong element of conscientiousness, or a desire to get things \emph{right}. This is a potent combination, and it's often what makes you successful as a software developer, and it can also help make you successful as a self-made expert.

So I realize that when I propose working in public to someone with this combination of drive and conscientiousness, I'm asking them to do something very difficult, which is to \textbf{be OK with imperfection}. In software terms, I'm asking you to ``ship it with known bugs''.

Furthermore, you'll be exposing your thinking to a critical audience. One that is partially made up of people you would like to take you seriously as an expert and perhaps hire you for your expertise. They know enough to see the ``bugs'' in your work.

Do it anyway. In fact, this exact uncomfortable situation is what accelerates your cultivation of expertise! It brings out the best in you. It keeps you intellectually honest, and working hard to impress your critical audience with your ever more nuanced, deeply considered versions of your self-made expertise.

If you're a musician, private practice is important, but if you can't \emph{emotionally move} an audience from the stage (or at least from the controlled environment of a recording studio), then you're not the kind of musician the world rewards with a great career. Likewise, if you're a self-made expert, you'll cultivate some of that expertise in private, but if you can't articulate your expertise in front of buyers (or articulate it through writing), then you won't be rewarded with profitable opportunities to apply that expertise. Working in public gives you practice articulating your expertise.

\subsection{Publish at High Frequency}

The final ingredient of my suggested accelerator is to publish to your audience at high frequency, at least for the first few years after you've gotten serious about cultivating expertise. High frequency publication is when you publish at least 3 times per week, every week. Having done this myself since January 2016, I can tell you the easiest version is to publish something every day that you work. So if you work a typical 5-day week, then publish 5 times per week. Before I started daily publication, I was lucky to publish something twice per month, and it was a struggle. The first 90 days of publishing daily were brutally hard for me, and then it got easier. After about a year of publishing daily, I found it dramatically easier to publish something worth reading \emph{daily} than I did to publish something worth reading \emph{twice per month}. Daily habits are simply easier to execute on after they're in place.

Publishing at high frequency simply offers you more opportunities to work on cultivating self-made expertise. This means that you will more quickly exhaust what you think you know about your area of expertise, and when you reach this limit, you'll face a small crisis of confidence. ``Is this all I have to say about this topic?'' If you persist instead of retreating or giving up, this crisis can lead you to look closer, dig deeper, and cultivate new expertise to share with your audience. You'll realize the limit you reached was a self-imposed limit, and the true boundaries of your expertise are much further away than you thought. What creates this crisis-breakthrough dynamic is an unwavering commitment to publishing daily. If you promise to yourself and your audience that you will show up in their inbox in a way that creates value for them at least 3 days per week every week and if your intention also is to help them by cultivating expertise, then you \emph{will} become the self-expert they need. And you will do so faster than you thought possible. Not overnight, but much faster than you would have thought.

Publishing at high frequency does some interesting things in the minds of your audience too. It portrays you as someone with a well of expertise so deep that you just can't help but share it generously and frequently. There's so much of it that you \emph{simply must} share it often. For those that enjoy your email list, it creates an addictive relationship with you. They look forward to hearing from you and are disappointed if they don't hear from you when expected.

The best scrap of evidence I have about this is an email (shared with permission) from Tricia, a member of my email list:

\begin{quote}
I've been wanting to email you for a little while with this little story.
\end{quote}

\begin{quote}
Initially I kind of hated your emails. I sneered at them and celebrated a little every time I DIDN'T click the link to your offer (kinda sad).  After all, I didn't really know you that well (I found you via Brennan's podcast).  Thought it was a little crazy that you were emailing me every day AND had a link to your offer each and every time.  I contemplated unsubscribing, but never did.
\end{quote}

\begin{quote}
Over time I started to get to know you a little and eventually like you.  Then I bit the bullet and became a customer in November, some email correspondence followed and interaction in Brennan's DYFA.  Somewhere along the way I started to love you in a non-creepy, you're-about-to-change-my-biz/life kinda way (Btw, the audio response to one of my emails is what made me an INSTANT raving fan.  I told EVERYONE about that.).
\end{quote}

\begin{quote}
Now I look forward to my 9:01 EST emails!  I think one came later in the day a little while ago and I actually wondered where was Philip?  I figured you were taking some time off. My two fave subject lines so far is this one and the man-boobs!!!  Such good little stories + tidbits that reinforce the positioning mindset.  Love it.
\end{quote}

Let me say this one more time: my proposed accelerant is unusual and audacious. It will make almost anyone who tries it uncomfortable. I'd bet that nothing you were taught in your formal schooling will prepare you for it, and it will push even the strongest of personalities to--and past--some sort of self-imposed limit. It is a potent accelerator for self-made expertise, and that's exactly why I want you to consider embracing it.

\section{Resilience}

I can keep this part pretty short, so I will.

The reason there are relatively few self-made experts who are killing it in business is because there are \emph{so many easier ways to make an OK living}. There are less risky ways. There are ways where you don't have to generate demand for your services; someone else does that part for you. There are ways where you don't have to figure out how to spend your time; someone else does that part for you. And there are ways where you don't have to take responsibility for the outcomes; someone else does that part for you. These ways of making an OK living we call a job, and especially for developers willing to invest heavily in new skills every 3 to 5 years, there are lots of jobs out there that pay pretty well.

In fact, the ``plan b'' is often this: ``I can just get up to speed on whatever language or framework is hot right now and get a job that pays in the 87th percentile.'' And it's true! If you are a reasonably good, reasonably skilled software developer in the US, you have been able turn your interest (probably self-cultivated) and skills (probably mostly self-taught) into an income that puts you in the upper 80\% percentile of Americans. According to Glassdoor.com at the time of this writing, the average software engineer salary in the US is an 87th percentile income. That means that out of 100 Americans, only roughly 13 are likely to earn more than you do.

You earn this enviable salary with less formal training, certification, or licensing than is required of a dental hygienist (73rd percentile) or massage therapist (48th percentile).

A ``plan b'' like this means your ``plan a'' of self-employment better be pretty compelling. But sometimes it's difficult. It requires more investment than most jobs do. It requires patience. Business setbacks can effect you more severely than they would an FTE.

So if your plan a is to cultivate self-made expertise, you'll need resilience to avoid the tempting siren call of your plan b.

The world \emph{needs} your self-made expertise, but there's no reserved seat at the table with your name on it. You have to create that for yourself by applying imagination, focus, discipline, and sustained effort over multiple years. And the best thing you can add into the mix in order to get there is \emph{resilience}.

\section{Flexibility}

Earlier I said: Expertise naturally seeks Venn Diagram overlaps between impact and risk. In other words, \emph{experts} feel the gravitation pull of this place on the venn diagram, and they deploy their curiosity and energy in this direction. Alternately, maybe they just follow their muse and the market rewards those that focus on these kinds of VD overlaps and ignores those that don't (survivorship bias).

To follow the gravitational pull of impact and risk, you'll need to be flexible, because your initial decision about how to specialize may need to be updated as you move up the ladder of expertise. As you do this, you're following the ``scent'' of value.

There's something about the combination of medium to high risk and important impact that makes problems that are valuable to solve. If the risk is low, then businesses tend to be more comfortable with commoditized solutions, or handling it internally, or some other non-expert solution. And if the importance is low, then business definitely exhibit price-sensitive behavior as they seek a solution.

But when the risk of creating a solution is high but the impact of a potential solution is also high, then businesses seek and readily pay for expert-level help. Not always, of course. In almost any market there will be a segment of highly price-sensitive or poorly educated buyers, where their psychology or ignorance overrides their business sense and they don't respond normally to this combination of risk and reward. But in general, enough businesses seek expert solutions where they face problems where the solution would be impactful but the risk of creating a solution is rather high.

And so if your initial thesis about how to specialize is not ideally aligned with a venn diagram overlap between risk and impact, then as you climb the ladder of expertise, you'll tweak it so that it is. This may involve varying levels of change in your focus and message. I can't promise that change will be easy, but it will be easier than the initial specialization decision was. You'll probably be somewhere in the dermis of expertise, and as a result you'll be more confident in your own abilities, including your ability to make and execute on an important decision about where your business is headed.

\section{Chapter Summary}