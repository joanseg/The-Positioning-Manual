\chapter{How Specialization Creates Value}

Specialization helps your business create more value, and creating more value makes it easier to charge premium rates and attract good clients. This ability to create more value is the core reason to specialize.

Specialization adds value in two specific ways. First, it \emph{addresses marketing inefficiencies}. And second, specialization can \emph{help you cultivate exceptionally valuable expertise}. Much more on both of these in a moment.

If I'm going to talk about how specialization addresses marketing inefficiencies, then I need to start with this question: \textbf{what actually \emph{is} marketing}?

There are lots of definitions, and a few really useful definitions. There are clever ones too, like Peter Drucker saying, ``The aim of marketing is to make selling superfluous.'' This is a warm and fuzzy sentiment for marketers, but it doesn't really say anything about what marketing is and does.

On the more useful end of the spectrum, let's start with Seth Godin’s definition: (I’m paraphrasing here): \textbf{marketing is changing the culture}. I know it's incredibly broad, but it's an important fundamental truth about marketing. But, as you know by now, as specialized businesses our target market is never ``the world'' or ``every user of \_\_\_\_\_\_\_\_'' or ``the culture at large''. We can't cultivate a reputation (aka market position) among groups of people that large.

OK fine, let’s scope Seth's definition down a bit to the level of \emph{your} business. When we do that, we get a definition more like this: \textbf{Marketing is changing how prospective clients understand your business}. In this view, marketing is merely taking charge of how your prospective clients understand your business and what you can do for them.

Here's another useful definition, and it's mine: \textbf{Marketing is connecting and building trust with prospective clients}. This looks at marketing as a set of two kinds of activities: ones that create a connection between you and prospective clients, and ones that build trust with those prospects, which is actually \emph{earning} their trust.

No matter which of these definitions you use, the default state of marketing is to \emph{fail to work}. To be \emph{ineffective}. I call these failures to work \emph{marketing inefficiencies}.

At this point, it's totally fair to ask, ``What does it look like when marketing \emph{does} work?'' Does it mean you will have a flood of highly qualified clients knocking down your door? Does it mean you will guaranteed revenue growth?

No, it does not mean any of those things. When marketing works, it looks like one  or more of the following outcomes:

\begin{itemize}
\item You earn and maintain attention over time
\item You earn word of mouth
\item You earn and maintain trust over time
\item You offer new ways of thinking that some choose to adopt
\item You offer next steps that some choose to pursue
\item You offer an exchange of value that some choose to buy into
\end{itemize}

Notice that these are all changes either in someone's \emph{relationship with you} or changes in their behavior based on \emph{choices you have offered and they have freely chosen}. These changes might lead to improvements in revenue for you or improvements in the quality of clients you get to work with, but revenue or other desirable business outcomes are \emph{second-order effects of marketing}, not first-order effects. What creates better clients and more revenue is a better \emph{system for finding better clients and generating more revenue}, and marketing is just one part of that system. It's an important part, for sure, but marketing is not solely responsible for your business results.

I hope that if you've thought of marketing as manipulative, sleazy, or somehow magical, I've managed to help you see it differently. If you think of marketing as modestly scalable activities for connecting and building trust with prospective clients, you won't act in a manipulative or sleazy way, and you won't attribute to marketing magical powers that it doesn't actually have.

Let's get back to that idea of marketing inefficiencies.

\section{About marketing inefficiencies}

If you neglect the following ``defaults'' of human behavior, you will find that your marketing often doesn't work:

\begin{itemize}
\item The default response of a human being to seeing your website or other artifacts of your marketing is to \textbf{misunderstand or ignore it}.
\item The default response of a human being to hearing what you do for a living is to \textbf{quickly forget it}. (How quickly have you forgotten just the \emph{name} of a person you’ve just met?)
\item The default response of a prospective client to hearing about what you can do for them is to \textbf{not trust you}.
\end{itemize}

These defaults are what causes most marketing to fail. Al Ries, in his book on positioning, describes the problem of information overload from the perspective of someone living in the early 1980's, \emph{before we invited the Internet and ubiquitous wireless computing into every corner of our lives}. Whatever level of information overload Ries was describing circa early 1980's has been amplified many time since then. And it's primarily this information overload combined with our ancient human hard-wiring that causes the default human behavior that causes most marketing to fail.

Each of these human behavioral defaults has an inverse quality that can be helpful. If you recognize and work with these inverse qualities of these defaults, your attempts to connect and build trust with prospective clients can \emph{not fail}. They can be \emph{effective}.

\begin{itemize}
\item The default response of a human who is a member of a social group is to \textbf{treat other members of their social group better than outsiders}.
\item The default response of a human who has a problem that they are aware of is to be on \textbf{high alert for a solution to that problem}. The more important and urgent the problem, the greater the solution-awareness.
\item The default response of a human who encounters another human with high perceived status is to \textbf{treat that high-status human better than lower-status humans}.
\end{itemize}

Specialization helps you use these \emph{helpful} defaults in human behavior to connect and build trust with clients more effectively.

There are 9 specific ways that specialization addresses marketing inefficiencies.

\subsection{Insider status}

When you specialize, you become an insider to an industry vertical, an audience, or a functional area of the business world (the supply chain, to pick just one example from many). You won't become a highly connected, highly trusted insider quickly. That takes time and work. But you will become an insider rather than an outsider, and in so doing you've become a member of a new \emph{social group}. Because humans tend to treat members of \emph{their social group} better than outsiders, you will gain an advantage.

\subsection{Solution-seeking}

Especially if you specialize in some sort of evergreen business problem or functional area of the business world, you can benefit from the solution-seeking behavior we humans default to.

I'm reminded of the 1984 film \emph{The Terminator}, and I like to imagine prospects who are feeling lots of urgency in their focused search for a solution as Terminator T-800's, with their heads-up displays seeking and scanning for their target which--in a happy departure from the film's plot--might be you and your ability to provide a solution to the problem that's driving their mission. Most people are not ``Terminators'', and so they'll ignore you or dedicate only a little bit of attention to what you're doing. As a generalist, you're probably used to that anyway. :) After you specialize, even more will ignore you. That's also OK, because the Terminators definitely will \emph{not} ignore you. Their internal heads-up-display will light up like a Christmas tree when they find you, and that's exactly what you want.

\subsection{Cultural status}

Western cultures confer elevated social status on experts. Experts are almost always specialized, and even polymaths tend to be serial specialists.

Much of how we humans make determinations about the status of others is based on a rapid, often subconscious process of feeding signals into mental heuristics to arrive at a determination of status. To choose just one example, the signal might be: being seen driving a luxury car, and the heuristic is one that associates driving luxury cars with high status. In another example, the signal might be: wearing a well-tailored suit, and the heuristic is one that associates wearing well-tailored suits with high status.

These status heuristics get us close enough to the truth often enough, and they save our cognitively-stingy brains lots of energy and do so with little risk of harm, so we use these imperfect heuristics almost every time. We develop these heuristics individually through observation, and we receive them already-formed through a sort of collective cultural inheritance.

Part of this cultural inheritance is a heuristic that confers relatively high status on specialist experts. Even if we didn't receive this heuristic from the culture, we would arrive at it through observation that:

\begin{itemize}
\item Specialists often go through specialized, advanced, or simply grueling training.
\item Specialists can often fix certain problems faster, better, or more easily than generalists can because of their depth of specialized experience.
\item Specialists are able to ask for and receive more financial compensation for their labor or expertise than those without any specialized expertise.
\item Generalists seem to outnumber specialists, so there must be something unique or different about specialists, and those with the courage to be different must have a good reason for doing so.
\item Specialists tend to value \emph{their own} specialization or expertise, and this self-valuing passes for confidence much of the time, and confidence/self-valuing sends signals that we associate with relatively high status.
\end{itemize}

Is it \emph{true} that specialists are more important, and therefore worthy of higher social status, than non-specialists? It doesn't matter, because there are enough heuristics out there running on autopilot in enough people that we--at least in Western cultures--act as if specialists are of higher status.

\subsection{Memorable, shareable, discoverable}

Specialization often makes your message about what you do or the value of what you do more memorable, shareable, and discoverable.

Most humans, when they see something, are going to wonder \emph{what it's good for}. That's the question we ask in order to contextualize value: ``What is this person or thing \emph{good for}?''. A variation of this question is:''Where does it fit into my world as I currently experience it?'' If we can't answer that question, we'll do our best to ignore or forget or \emph{maybe} file for future reference a memory of this thing. But we almost certainly won't take action with or towards this thing we can't contextualize because we can't value it.

Specialization helps others contextualize your services. Specialization helps  them understand \emph{what your services are good for}, and that contextualized understanding makes your services more memorable.

Your message becomes more shareable because it's specific. That specificity makes it able to--as my friend and pricing expert Jonathan Stark likes to say--create a ``Rolodex moment'', which is moment when you react to a specific message by wondering who you know who could benefit from it.

I suppose the following won't be true of every human. There is, after all, a small percentage of sociopaths among us. But most humans, when they can contextualize and understand the value of something they've just discovered, will want to share the discovery. They might do so for essentially selfish reasons. ``If I help my friend by sharing this with them, their perception of my status will increase, and that will be good for me.'' Or you might do so for more altruistic reasons. ``Life has been good to me, and other people have been good to me, and so I'd like to pay that forward any way I can.''

Either way, your response to this impulse to share will be to\ldots{} share, but because you are sharing something that has a specific use embedded in it and will therefore only create value in a specific context, then you won't share your new discovery with just anybody. You'll search your ``mental Rolodex'' for a really good match and share your new discovery with those right matches. It's much easier to find these right matches when the thing you want to share is specialized such that it's for a specific type of person or a specific situation. This makes the mental Rolodex matchmaking much easier.

The dynamics I've described above also work to make specialized services more discoverable. The ``mental Rolodex'' gets accessed around the moment of discovery, when someone stumbles across your specialized business and wonders who might benefit from their discovery. But that's not the only time it gets accessed. It also gets accessed when that person is asked, ``hey, do you know anybody who \_\_\_\_\_\_\_\_\_\_\_\_?'' If I'm asked that question, it's easier for me to make a \emph{confident} recommendation if the match between what someone is looking for and the file card in my mental Rolodex is an exact match. There will be less hesitation in my voice. I'll feel like the recommendation is more valuable for you because the match between your needs and what I know of the business I am recommending is a more exact match. So your services become more discoverable through word of mouth.

They also become more easily searchable on the Internet. Google and other search engines have moved beyond the days of using simple keyword matching alone to rank search results, but keyword matching still matters! And among the millions of \textbf{niche}, long-tail searches that Google will handle in a typical day, keyword matching will play an outsized role in determine what search results to present and how to rank them. In other words, if you are searching for some general topic, like ``migraine headaches'', then Google has to prioritize and \emph{ocean} of content. This is the opposite of searching for a niche topic. Keywords alone are not enough to allow Google to present useful search results, so they're going to incorporate other ranking factors like inbound links, domain age, etc., etc.

On the other hand, if you are searching for a very niche topic, there will simply be less raw content for Google to prioritize. Other ranking factors like inbound links, domain age, etc. still matter even with niche searches, but simple keyword matching will be relatively more able to allow Google to present useful search results. This means that your specialized business will be more easily discoverable through a search engine like Google.

Finally, as you specialize, you will start having more contact with prospective clients within your area of focus. So their chance of ``bumping into you'' online or otherwise will increase because you'll both be part of the same social group. This also serves to make your business more discoverable.

\subsection{Better targeting}

Specialization makes it dramatically easier to conceive of \emph{who} you are trying to reach, which creates a multitude of cascading positive effects.

There are many reasons why marketing as it's usually practiced sucks, but lack of clarity about \emph{who the thing is for} is chief among them. This lack of clarity leads to unrealistic claims and manipulation, vague and weak messages, and a tendency to chase numbers rather than impact.  At best, attempts to connect and build trust with\ldots{} eh, pretty much everybody will be boring and ineffective. At worst, attempts to connect and build trust with a poorly defined target market or audience will be offensively filled with pressure and desperation.

The first thing I see happen with people who have chosen a specialization is usually a flood of new ideas about how they can start talking to prospective clients in dramatically more specific, relevant ways. If they're already producing content marketing or have held back on doing so because of a lack of clarity on who they are trying to reach, it's a flood of ideas about content they could create to reach and serve a specific type of services buyer. Or if they're not even that far along in doing disciplined lead generation, then it might be a flood of ideas about how that can--for the first time--clearly communicate the value of what they do. It's often as simple as knowing who to ask for referrals.

\subsection{Marketing message}

As soon as you decide who your services are for, you can communicate more effectively with them. I wrote the previous sentence, went on a short break, and then when I returned and looked at it, it seemed ridiculously simple. Like, insulting-your-intelligence simple. And yet, how many of your peers know \emph{exactly who} to recommend your services to? How many of you know exactly what keeps your clients awake at night? How many of you have received compliments from prospective clients about the powerful message they saw on your website? Specialization isn't complicated, it's just something that requires courage, and so it's less common than the other approach of trying to serve basically everybody.

Think of your marketing message as a short (usually extremely short) story about how your services are uniquely able to create exceptional value.

Stories can be surprisingly short. Although here's no evidence to support the apocryphal tale that Ernest Hemingway wrote the following 6-word story on a bet, it's fun to imagine him doing it anyway: “For sale: Baby shoes. Never worn.” Short, but it packs a punch doesn't it? Good stories have action and consequences, and with enough specificity about either action or consequences we can infer the rest, which is what allows Hemingway's 6-word story to say so much more than you'd think 6 words could.

One of my favorite marketing messages that has a sort of story embedded in it comes from Corey Quinn. As of the time of this writing, his consulting website headline reads: ``\textbf{I help with horrifying AWS bills-- both reducing the dollars spent as well as understanding your spend.}'' Short, but it packs a punch doesn't it? :) For me, Corey's language evokes a short film clip of someone from finance walking into the CTO's office, closing the door--the camera cuts away here to a long shot from outside the CTO's office--following by inaudible yelling and wild gesturing at a stack of paper.

You can't bring a story to life without specifics. Details. Characters. A protagonist, a problem to be solved in search of a goal, and obstacles in the protagonists way. A \emph{specific} protagonist, a \emph{specific} problem, and \emph{specific} obstacles. Andy Wier's book, \emph{The Martian}, is a great example of incredible specificity in all of those story elements, and for me it made the book more enjoyable than the film.

Your marketing message is a short story about how a specific protagonist (your ideal client, narrowly defined) can overcome a specific obstacle (the problem you help them solve) in search of a specific goal (the business impact you help them create). Notice who is the protagonist here: your client. Notice who is \emph{not} the protagonist here: you, your skills, your process, or your team. Those are the supporting characters, not the protagonist/hero. That's exactly the mindset you should be in when you're trying to connect and build trust with prospective clients. They are the hero, you are the enabler of their heroic success. They get the credit for the success, you get the satisfaction of knowing they couldn't have done it without you (and hopefully a case study saying just that).

Remember that earlier in this chapter I said that the default human behavior is to misunderstand, forget, or not trust/believe your marketing message/story. For this reason, your marketing message contains a story, but it also needs to include other elements to help strangers \emph{believe that the story could be true for them}. I won't get into those other elements here, but in brief they are \emph{proof} and \emph{demonstrations of expertise}.

\subsection{More effective inbound}

Inbound marketing, if you're not familiar with the idea, is activities that are meant to attract prospective clients to you. A great example is guesting on a podcast, saying something that resonates with a prospective client, and they respond by seeking you out and engaging your services. Inbound marketing often takes the form of publishing useful stuff and hoping that it connects and resonates with prospective clients.

Specialization can make your inbound marketing activities more effective because it gives you clarity about who you are trying to connect with. That clarity makes it possible to understand what those you are trying to reach care about. Knowing what they care about makes it possible to earn their attention. Earning attention is the first step towards connecting and building trust.

\subsection{Easier outbound}

If you're not familiar with the idea, outbound marketing is reaching out to prospective clients and checking for need or interest, or simply inviting them to become aware of your services. Specialization makes outbound marketing easier to execute and more effective for the same reasons it makes inbound more effective.

In fact, effective outbound marketing is literally impossible without a clear idea of \emph{who} you are trying to reach. Most outbound marketing is going to take the form of some kind of interruption. At the very best, this interruption will be similar to a restaurant server briefly interrupting your conversation to refill your drink. Yes, it's an interruption, but yes you're also happy for the outcome of the interruption. In other words, it was an interruption in service of your needs, not in service of the waiter's needs. Furthermore, it was a contextually appropriate interruption. The waiter can observe that your drink was in need of refilling. That's why they approached the table. The subsequent interruption was necessary so they could first check for permission from you before refilling the drink, and then for the brief moment necessary to actually refill the drink. But again, you appreciate the outcome and see how the interruption led to a desirable outcome.

You can't interrupt people in a contextually appropriate way that they ultimately appreciate without being very clear about who you're trying to reach and why your interruption will benefit them. To continue the restaurant analogy, you need to approach the right table. You need to be pretty sure your customers there might want a refill of their drink(s). You need to make your interruption clearly relevant, polite, and efficient. Specialization makes it possible to do outbound marketing in a way that is relevant, polite, and efficient.

It's also worth pointing out that there are certain professions you will \emph{never} see doing outbound marketing because a) they don't need to and b) using outbound marketing would send a signal that they are not busy and therefore not valuable or competent. But even if you want your business to be perceived as not needing to do outbound marketing, in the early days of establishing your business and cultivating a reputation, outbound marketing can be useful for ``priming the pump''.

\subsection{Force multiplier}

Finally, specialization acts as a force multiplier for your efforts to connect and build trust with prospective clients because it allows you to focus the same amount of force on a smaller area to achieve more impact faster. You could think of how the Allied forces in Word War 2 chose a beachhead for their invasion of occupied France. They focused a lot of force on a very small area in order to push through enemy defenses. The same amount of force applied over a larger area would have been less effective. You could also think of how a hypodermic needle, with its extremely sharp tip, makes it easier to penetrate multiple layers of skin to deliver its contents to a vein or to deep tissue without harming the surrounding area.

This makes it easier for you to ``be everywhere'', because everywhere really isn't everywhere, it's just the places the people you are trying to reach pay attention to. It's not the \textbackslash{}\ensuremath{\sim}500,000 podcasts on the iTunes podcast store, it's the 10 or 20 podcasts your target market pays the most attention to. It's not the thousands of conferences that happen every year, it's the 3 or 5 that are most important to your target market. It's not running paid advertising on every possible platform, it's focusing that ad spend tightly where it will have the most impact.

When I say that specialization addresses marketing inefficiencies, it does so for one, several, or all of the 9 reasons described above. Even having just one or two of those 9 advantages working in your favor can make your efforts to connect and build trust with prospective clients dramatically more effective.

\section{Examples}

\begin{itemize}
\item Boatload of examples/case studies


\begin{itemize}
\item 2 or 3 vertically specialized examples


\begin{itemize}
\item Maybe Mike Julian for a horizontally specialized example?
\end{itemize}
\end{itemize}
\end{itemize}

\section{About cultivating exceptionally valuable expertise}

Earlier I said: Specialization adds value in two specific ways. First, it \emph{addresses marketing inefficiencies}. And second, specialization can \emph{help you cultivate exceptionally valuable expertise}.

The second benefit of specialization is an added ability for you to cultivate expertise. This idea really needs its own chapter, so what the next chapter is about.

\section{Chapter Summary}

TODO