\chapter{Superficial vs. Deep Specialization}

There are two valid ways to look at specialization. I will refer to one of them as \emph{superficial}, and I don't mean that as an insult. I simply mean it doesn't go that deep. It therefore has different requirements and delivers different benefits to your business.

For a long time, I felt like these two meanings of specialization were in tension. Maybe they are, but that need not stop me from helping you use either or both.

\section{Superficial Specialization}

Superficial specialization is a marketing tactic, a market research technique, or some combination of the two. It might also be relatively temporary in duration, or done as a first transitional move out of the generalist market position.

Generally, superficial specialization means changing your marketing message to focus on business outcomes or the needs of a specific market vertical or specific type of buyer rather than technical inputs. This change will trickle down into how you talk to clients as well.

You'll still have room to talk about technical inputs, but those get pushed down to the level of a supporting detail in your message. At the headline level, your message becomes one that's focused on business problems and outcomes, contextualized within a particular vertical market or horizontal problem or audience of buyers.

Superficial specialization might also mean specializing a marketing funnel so it speaks more specifically to a vertical or problem domain. You might have multiple ``micro funnels'' in place, each with a different flavor of specialization. I worked with a client who used this approach to experimentally identify where he could find the strongest interest in his DevOps consulting services. When he got lukewarm interest in a campaign specialized in migrating away from Heroku but got \textbackslash{}\ensuremath{\sim}80\% open rates on a campaign focused on Kubernetes, it became pretty clear where he could more easily gain attention for his work. Building these two micro-funnels required work, but less work and investment than a ``blind pivot'' would require.

Superficial specialization is generally campaign-based, rather than a career development strategy. As a reminder from a previous chapter, this specialization approach lines up with the notion that positioning is a verb; something you do actively over a short time frame in order to achieve some sort of tactical win (lead gen, dealing with a famine phase, etc.)

\section{Deep specialization}

Deep specialization is a career development strategy. It looks as your market position as a noun; an asset that you build over time. It looks as your market position as the result of coordinated, disciplined action over a longer time frame to achieve a strategic win (become attractive to vastly better clients, become seen as a category expert, move from implementation to advisory work, etc.)

\section{Overlap between these two concepts}

There's some inevitable overlap between these two ways of understanding positioning. They both involve focusing or re-focusing, but the demands of deep specialization require that we commit basically 100\% to a single chosen path for specialization.

That said, you might use multiple superficial specialization experiments to validation a direction for deep specialization

The deep specialization approach is more ambitious and takes longer to execute on. It's also the one that's more likely to result in a strong market position after all that work.

They start at roughly the same place, but they end up at very different places because of one thing: \emph{expertise}. Superficial specialization is a way to address some marketing inefficiencies; deep specialization is a way to transform your business model from one based on outputs to one based on expertise.

How should you choose between the two?

\begin{itemize}
\item If \textbf{low risk tolerance and low business maturity}: probably superficial
\item If \textbf{low risk tolerance and greater business maturity}: validate via superficial, then go deep
\item If \textbf{high risk profile}: you've got lots of choices if you're smart about them. High risk profile doesn't mean you are guaranteed to succeed, but it does mean you are less likely to pull back or ``flinch'' at a critical moment when staying the course instead of flinching is the difference between success and failure.
\end{itemize}

TODO: Interesting idea: business maturity metric (based on years working for self, number of clients/projects/etc, and any other important factors with some kind of metric to compare against) as a way to think through superficial vs. deep positioning. Back-test this against my clients

\section{Chapter Summary}

\begin{center}
\rule{3in}{0.4pt}
\end{center}

PL:

Your market position is a lagging indicator of the relevance and value of your specialized expertise. It's the mental residue of your marketing message and client interactions over time. It's your reputation. \{might move this to ch 04, or just have this be a reminder of this same point as elaborated in ch 04\}