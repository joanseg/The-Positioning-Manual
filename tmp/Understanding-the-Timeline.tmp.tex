\chapter{Understanding the Timeline}

Specialization is the decision to focus; a market position is the reputation you build up over time as a result of consistent, disciplined followthrough on that decision.

\section{How much time does it take to build a market position?}

\textbf{\emph{``Most people overestimate what they can do in one year and underestimate what they can do in ten years.'' --Bill gates}}

3 months to 3 years. That's what I want you to be prepared for. It generally requires 3 months to 3 years of consistent work to build up a market position.

3 months is the lower limit of what I've seen in special cases. Folks starting with an extremely small market, or starting with a significant head start that lets them get the word out about their new specialized focus.

3 years is more like what it takes if you don't have any particular head start, or perhaps if you're starting with a very short runway and have to execute a slow-motion transition from generalist to specialist because you can't afford to say no to many opportunities.

Depending on how much fame you want to be associated with your reputation, it can certainly take more like 5 to 10 years to get where you want to go. If you feel like getting invited to give a TED talk or publishing a NY Times best-seller is validation of your market position, then be prepared for 10-year journey that you could shorten a bit with a lot of hustle, hard work, or both.

But if validation of your market position looks more to you like starting to get great referrals from within your area of focus or starting to notice a more favorable power dynamic in your sales conversations, then 3 years is a pretty realistic timeframe.

\section{Earning your ability to say no}

It's easy to see how specialization works, get excited, and then get unreasonable. For most people, the unreasonable part looks like fantasizing about turning down every prospect that's not a perfect fit for your new specialized focus. When you're at your personal ``peak of inflated expectations'' around specialization, it's easy to imagine things working this way.

In reality, it's much more likely you'll have to ``earn'' your ability to say no to work outside your area of specialization. Now granted, if you're just drowning in great leads and enough of them are within the area you've chosen to specialize in, then great! You can do just what you're fantasizing about. You can say no to or refer away prospects that are outside your new area of focus.

But again, what's much more likely in my experience is a multi-year transition period where the following happens:

\begin{enumerate}
\item You decide to specialize. You probably do so in a way that builds on some advantage you have.
\item While continuing to say yes to whatever work you need to in order to preserve cashflow, you double down on your area of advantage.
\item You start investing in marketing to get more clients that are aligned with your specialization decision.
\item As more and more of those new prospects start to show up, you can say no to more and more prospects that don't fit your new area of focus or, if you're pursuing a pure horizontal specialization, you can suggest engaging with new prospects just within your area of horizontal focus.
\end{enumerate}

This is how you earn your ability to say no. For most of us, it's either this way, have a trust fund, burn a year or more of runway, or take on debt.

\subsection{Whales, Jonah, and Ahab}

This is a good place to talk about the two kinds of whales some of you will have to deal with as you cultivate a market position over time. Whale \#1 is the one that swallowed Jonah, and whale \#2 is the one Ahab wanted to kill.

The first whale is the whale client. Your relationship to them is one of relative powerlessness. In a symbolic and real way, they’ve swallowed your business and they call the shots.

One interesting manifestation of this I’ve seen (rarely, thankfully) looks like this:

\begin{itemize}
\item You look at your specialization options and identify a head start that is vertical in nature.
\item Your current whale client is in the vertical it makes sense for you to specialize in.
\item You gingerly feel out how your whale feels about you specializing in a way that would have you working with other companies in their vertical. You get strong, negative feedback. This would be a real problem from their perspective.
\end{itemize}

Now what? Your whale client is cock-blocking your best specialization option!

Make no mistake. I see this as a very difficult situation, and sympathize with the difficulty of getting out of it. But the exit plan is pretty simple (unless you have massive runway or other resources to soften the blow of firing the whale client, in which case it’s \emph{really simple}: harpoon that whale and take back control of your business):

\begin{enumerate}
\item Preserve the whale and work like hell to get out from under the whale. Do anything you can to address this client concentration problem, and seek advice from David C. Baker, Blair Enns, and Jonathan Stark on practical steps for doing this.
\item In terms of specialization, pursue a horizontal specialization until you've escaped the belly of the whale, even if your ultimate plan is a pure vertical specialization. Yes, you’re not playing your strongest hand, but you’re being proactive and future-focused, and that’s really valuable.
\end{enumerate}

The second whale is sunk cost. I haven’t measured this precisely, but I’d estimate maybe half of those who struggle with making a good specialization decision are held back by some sort of sunk cost.

Getting personal for a moment, I have some advantages in this area that I think are somewhat rare. First, Cheryl—my wife of 10 years who I love even more than when we first got together—is absolutely, 100\% \emph{immune} to most forms of sunk cost. I’ve never met anyone who makes decision in a way that is as unhindered by sunk cost. There is this pretty famous TED talk by a woman who had a stroke \{TODO: research rest of this story\}. Sometimes I wonder if this happened to my wife, except it only effected whatever part of the brain responds to sunk cost (I know there probably isn’t such a thing, but it’s a fun way to imagine how she got so immune to sunk cost). \{TODO: Add to sunk cost section: bombing in front of a bunch of my peers.  Farting student, maybe. \}

Second advantage: I have done a lot of crazy shit in my life. I lean \emph{way out} over my skis a lot. I've lived in 24 houses if you don't count college dorm rooms and the 8x8 foot pump house I lived in for 6 weeks on the Oregon coast, some because my parents moved a few times, but most because I wanted to move. I know firsthand what it’s like to join the YMCA because that’s the cheapest way to get access to a place to take a shower.  I was 100\% convinced I was seconds away from my own death—and astonishingly lucid about the whole thing—when my truck spun off Interstate 24 on the way to Chattanooga in a snowstorm. My ex-wife left me for a cult leader in the same 18 month period I did a bankruptcy and foreclosure. I found some weird loophole in the USDA regulations that let me get a license to import the constituent ingredients for Ayahuasca from Peru (semi-legal) and sell them on eBay (definitely not legal). There’s less crazy shit to tell you about \emph{within} the statute of limitations because even I’ve calmed down a bit as I’ve aged. But I have lots of practice torching sunk costs I’m pretty sure many others would fight to preserve.

And so when I see you fighting to preserve some kind of sunk cost that looks to \emph{me} like a go-kart at the expense of pursuing an opportunity that looks to \emph{me} like a BMW 5-Series, a wonder why you’re fighting so hard for that sunk cost. I know it feels safe because it’s what you know, but I also see it blinding you to the opportunity.

I’m not irresponsible when giving my clients advice. I know full well that you’ve fought for what you have and you’re not going to “put it all on red”. That’s not at all what I’m advocating. Here’s what I am:

Be \emph{brutally honest} about whether you perceive the sunk costs in your life and business \emph{accurately}. Are they a tuna that looks like a whale to you? According to some guy on Quora, a blue whale (190 short tons) could feed between 120,800 and 604,000 people a single (disgusting) meal. Using that guys' pretty decent math, an Atlantic bluefin tuna fish (1,508 lb) could create one 1/2-pound meal for between 482 and 2,412 humans . Big difference. If your tuna of a sunk cost looks like a whale to you, you’re going to make some decisions that aren’t in your best interest.

I put this in the chapter on timeline because sunk cost is a psychological factor with significant business implications. Like a lot of psychological aspects of our business, change in this department can be slow and painful. Or not! But, it’s often slower and more painful than we’d like.

And so time can help, if you use time the right way. Example: I almost never wash dishes immediately after getting them caked with oil or whatever. I generally let them soak. My wife hates this, but I hate scrubbing dishes more, and I hate doing work that some time plus a bit of soap and hot water can do for me.

A simple awareness of how you might be holding on to sunk cost at the expense of future opportunity might be the first step. With some time and attention to this new awareness, you might start seeing sunk costs that used to look like whales start looking smaller and smaller, because they’re actually tuna fish and you’re getting better at accurately sizing them up.

You might start to practice walking away from sunk cost on a small scale to build up whatever “muscle” is responsible for walking away from sunk cost. Maybe that un-used treadmill that’s become a laundry drying rack is a good practice round for building this muscle. :)

Seth Godin describes sunk costs as a gift from our past self. And like any gift, we don’t have to receive it. We’ll feel social pressure to do so, but you can drive straight to the Goodwill donation center on your way home from the party if you want to. Nobody will know except the future better-off version of you.

\section{Adjusting/refining over time}

Quick question: what is the oldest memory you have of your answer to the ``what do you want to do/be when you grow up?'' question?

My answer to that question was: ``aeronautical engineer''. From reading this book, you can tell I've ended up a pretty long way away from that original destination. :) Apparently, this is also true of 94\% of the population: \href{https://web.archive.org/web/20181210143117/https://www.elitedaily.com/money/entrepreneurship/only-6-percent-of-people-acheive-their-childhood-dreams/855270}{https://web.archive.org/web/20181210143117/https://www.elitedaily.com/money/entrepreneurship/only-6-percent-of-people-acheive-their-childhood-dreams/855270} and \href{https://academic.oup.com/sf/article-abstract/93/1/31/2338003?redirectedFrom=fulltext}{https://academic.oup.com/sf/article-abstract/93/1/31/2338003?redirectedFrom=fulltext}

As an adult, you're in a better position to make a good decision about how to change your career, but you're never equipped with perfect information and an ability to predict the future. None of us are. So, it's totally natural that you'll iterate on your initial specialization decision as time passes.

Some of these adjustments will be based on new insight into the market you serve or new information about it. As you move deeper into your specialized focus, you'll gain more clarity on what really moves the needle for clients and perhaps tighten your focus to focus more on needle-moving activities.

Some adjustments will be based on changes within you. Changes in preferences, a desire to move from implementation to advice, and so forth.

And in some cases, you'll thoughtfully decide to pour gasoline on a market position you've painstakingly built up, throw a match on it, watch it burn all Halloween-orange and chimeney-red, and then walk away gleefully cackling, knowing that you possess the skills and hutzpah to build up a new one that supports a whole new chapter in your life. My wife recently did this, and it's already shaping up to be a smart, well-timed career move on her part.

\section{Accelerants}

As we think about the timeline of specializing and then building up a strong market position around that specialization, it's smart to ask what: ``might accelerate things?'' ``How can I get to a strong market position faster?''

Here are the accelerants I've seen actually work:

\subsection{Connections, good timing, and luck}

I often like to say: unfortunately you’re in a relationship business. I say this because most of us would rather invest in fun tech skills than relationships. But if you happen to be blessed with useful business connections, it can really help because… you’re in a relationship business. And the right kind of relationships or connections can accelerate your move from generalist to well-positioned specialist. In fact, one of the ways of deciding how to specialize that we’ll explore in Chapter 12 is to specialize in the area where you find your most significant head start. If you have especially useful relationships or connections in a certain area, that often represents a significant head start that you can build up into a great specialization.

Good timing is another accelerant. If you’re insistent on specializing in a platform, then one form of good timing is to “get in on the ground floor” when a platform is in the very early part of the Gartner Hype Cycle. This is an accelerate that is partially within your control. After all, Gartner frequently publishes updates to their Hype Cycle research, and even if the particular Hype Cycle you want to reference is one of their paid products, they freely publish the table of contents, which lists everything you need to know right there in text. So it’s pretty easy to know what platform is where in the Hype Cycle.

What is less easy (ok, impossible) to know with early stage innovations is whether they will become mature tech in 5 to 7 years or whether they will instead join Segway, \{TODO a few more examples\} in permanent limbo between niche adoption and mainstream adoption or whether they will join \{TODO: examples of dead\} in the boneyard of failed innovations. And without that knowledge, a platform specialization is you placing a bet with less than 100\% odds. As I’ve mentioned before, when you’re young and early in your career, those bets feel fun and less risky than they feel later in your career.

And the final accelerant in this group is \emph{luck}. Luck isn’t necessary for a successful career, but man, when luck smiles on your efforts it does act as a strong accelerant. And there certainly are proven ways to use hard, disciplined work to increase your “luck surface area”. Specialization helps you aim your “hard disciplined work cannon” in the right direction, which helps increase your luck surface area as well. So as you specialize and work hard at the followthrough you’ll usually get incrementally more “lucky”. But sometimes you get a massively lucky break. When that happens, be deeply grateful, run with the lucky break as far as it will take you, and don’t let your ego take credit.

\subsection{Building on a head start}

A head start is some advantage that you possess. Building on an advantage is a specialization accelerant, at least compared to undertaking the same specialization journey without that head start.

You \emph{can} specialize without any particular head start, but if you ignore a signifiant head start, you’re making more work for yourself. You might decide to do this for good reasons, but know what you’re giving up by ignoring that head start.

I’ve lost count of the number of generalists I’ve worked with who have a head start story that goes something like this: “I’ve worked with a bunch of companies in finance, but I hate working with those kinds of companies.” I support their walking away from this head start, but only after I make sure they know what it is they’re walking away from—which is a significant accelerant that could help them build a good market position more quickly—and after I make sure they have the risk tolerance to handle a more difficult specialization journey.

Head starts generally take the form of one or more of these advantages:

\begin{itemize}
\item A significant number or depth or quality of \emph{access} or \emph{connections} to a target market.
\item A significant amount of \emph{credibility} in a market vertical or within a certain audience.
\item A significant depth of \emph{expertise} in some area.
\item A significant depth of \emph{insight} into a target market’s needs.
\end{itemize}

\subsection{Serving an audience generously}

A willingness to invest in serving a market or audience \emph{generously} with few or no expectations of immediate return is another powerful accelerant I’ve seen.

I’ll provide a few great examples momentarily, but in general this accelerant looks like you informing, educating, inspiring, or connecting those you are specializing in serving. It often involves finding a “hole”—a place where information or connection is needed and under-supplied. It involves doing your best to create something so good that if it was free some would gladly pay for it and so useful that many choose to share it with others with little or no prompting from you. And—crucially—it involves stepping in to this leadership-service role \emph{before you feel fully qualified to do so}. I can’t fully explain why, but if you wait until you feel 100\% ready to step into this role, it will probably be too late.

Here are three very good examples of this kind of service:

\begin{enumerate}
\item Mike Julian started a free weekly newsletter called \href{http://monitoring.love}{\emph{Monitoring Weekly}}. It’s a weekly roundup of useful, informative, and relevant articles, blog posts, etc. Naturally, this newsletter aligns closely with Mike’s services, but it’s also done with generosity and quality, so it feels very light on any kind of sales intent. Each issue has a subtle, low-pressure invitation for recipients to learn more about Mike’s analyst and consulting services. Growth has been 100\% organic, meaning referrals from other newsletters and word of mouth have moved the subscriber count from 0 to \{@TK: check with Mike for subscriber growth and timeline\}. A side benefit is that Mike can sell sponsorship spots in the newsletter, though as he’s moved into more of an industry analyst role he has stopped selling ads in order to avoid the perception of bias. This newsletter has been so successful that it’s inspired Mike to start a podcast with a similar function: generously serve the community by filling an information hole.
\item Corey Quinn started a free weekly newsletter called \emph{Last Week in AWS}. The goal and approach is nearly identical to what Mike Julian is doing with \emph{Monitoring Weekly}, but the audience is different. Corey also has a podcast that fills an information hole for the same community.
\item Kenna Cote organized a yearly conference for handmade soap makers. She saw a hole, and filled it with an event that connects and inspires her audience of handmade soap makers. From this central “superconnector” position, she sees and can act on opportunities to offer her services.
\end{enumerate}

It’s certainly possible to undertake this kind of service and \emph{not} see the kind of results that Mike, Corey, and Kenna have, but if you know your market and work from a place of true generosity, you are more likely to succeed than fail.

I should also note that this accelerant is less demanding on you than some others because you’re not claiming to be an expert or an authority. Often the content that you’re including in your newsletter, podcast, or other informational resource is not stuff you’ve created; it’s stuff you’ve found elsewhere. You’re acting as an editor of curator, not a subject matter expert. Over time, this curation work can make you an expert, or at least very well informed about the context and specifics of the topic of your newsletter. But at first, it doesn’t require much domain expertise. You need just enough to see the hole this information can fill and some validation that filling this hole is important to those you seek to serve. After that, it simply requires willingness to serve and some dedication to push through occasional inspiration dry spells.

\subsection{Sharing boldly as you go}

The next accelerant will feel more risky, but it’s not actually that much more risky. We could think of this accelerant in several ways:

\begin{itemize}
\item “Learning out loud”
\item Sharing what you learn shortly after you learn it
\item Doing hard, inefficient, risky work so others don’t have to
\item Exploring the bleeding edge so your prospective clients don’t have to
\end{itemize}

No matter exactly what form this accelerant takes, the key is \emph{sharing as you go}. This is the most important part of this, and it’s probably the part most of you are the least comfortable with. This is fine. It’s because you’re very likely both driven and detail-oriented. Of all the developers I’ve worked with, this is the most common combination of work personality characteristics. A healthy amount of drive, and a very healthy does of conscientiousness. Sharing something you’re less than 100\% clear or confident about probably does not come naturally to you.

Do it anyway. It will accelerate your ability to cultivate a good market position. Those you are sharing with see fewer of the flaws than you do, and they care about those they do see less than you do. If your sharing is wrapped in a spirit of generosity and humility, those qualities will outweigh almost all shortcomings in the content you are sharing.

It takes real boldness to share something you learned seemingly moments ago. It takes humility and courage to share something publicly, be called out as wrong, and correct yourself publicly. Do it anyway. The opportunity cost of \emph{not} doing this is usually more than the very small downside of occasionally being wrong in public.

It’s worth it. It’s worth it. It’s worth it.

\subsection{Leverage existing audiences}

If you’re specializing in a way that’s relatively low risk, then you are almost certainly specializing in serving an \emph{existing} market vertical, audience, or focusing on an \emph{established} (rather than emerging) problem domain. This means that somebody else—some kind of superconnector--has probably already gathered an audience that maps directly to the market vertical, audience, or problem domain you are focused on.

This existing audience might take the form of a regular conference, a podcast, a publication, or an online “watering hole” (private or public group, forum, or Slack group).

The pre-existence of a relevant audience can be an accelerant for you because you don’t have to build the audience yourself! You can just “borrow” the audience by giving a talk at that conference, guesting on that podcast, submitting an article to that publication, or becoming a generous participant in that community. If you execute several or all of these activities it’s possible to appear to “be everywhere” surprisingly quickly, and being everywhere is a real accelerant as you build your market position.

\subsection{Polarizing PoV}

A point of view (PoV) is your opinion on how best to achieve results for your clients.

\begin{itemize}
\item You make your point of view more \emph{memorable} or \emph{shareable} when it is specific, can be summarized in a short, punchy sentence or phrase, and addresses something that is important or of-the-moment for your clients.
\item You make your point of view more \emph{convincing} to your clients when you combine it with examples or evidence that supports it.
\item You make your point of view more \emph{attractive} to a subset of your prospective clients when it expresses an opinion they also hold but have to be somewhat reserved about because of company politics or other reasons.
\end{itemize}

A polarizing PoV is one that intentionally polarizes people into one of two camps: those that agree, and those that disagree. “The Earth is round, not flat” was a very-not-polarizing PoV for \emph{centuries} until the Internet connected a critical mass of dumbasses, and now this PoV (a \emph{fact} if I’m being careful with my language) is somewhat more polarizing now. Plenty of things can make for genuinely polarizing PoV’s. Pretty much every project management style, style of software development, or way of doing things in the world of custom software will have fans and foes, and none of them can be absolutely proven to be “correct” or “true” or universally best.

The goal with a polarizing PoV is not to upset or harass people, it’s simply to take a stand such that some will agree and some will disagree. Maybe you saw me do that in the previous paragraph with referring to flat Earthers as dumbasses. Those that agree with you will be on a fast track to remembering and trusting you, and you don’t have to call anybody a dumbass.

Your points of view (you can have more than one, of course) should relate to things your buyers care about. If your PoV(s) speak to risk or how best to move the needle, that’s better because it generally places you within the right kind of conversation at potential clients.

\subsection{Build a promotable thing}

Building a somewhat unique “thing” you can promote can be an accelerant. Especially if you don’t have a strong, clear PoV that you can articulate, a “thing” that you can promote is your next best accelerant.

I say “thing” because it could take multiple forms:

\begin{itemize}
\item A blue ocean productized service
\item A free gift of information
\item A relevant, valuable dataset or useful by-product of research
\item Open source software
\end{itemize}

All of those “things” are accelerants because they will tend to make you feel better about promoting what you do because you’ll be promoting something \emph{other than yourself}.  Any of those things are something that you could feel good about giving away, or asking strangers (there directly or indirectly) if they’d like to have it. In some cases, you can just publish it somewhere and rely on motivated seekers to find it through search or other means.

Ask yourself this right now: you have a choice between getting up on a stage and speaking for 30 minutes about you and your business, or you can be on the same stage for the same 30 minutes but talking about some open source software you built and think will be useful to some in the audience. Which of those talks would you rather give? I’m betting 80\% or more of you would rather eat a bowl of broken glass than give the first talk, and 80\% of you would be relatively comfortable giving the second talk. The difference is not who is on the stage, what stage, or how long you’re there. It’s what you’re talking about. This paragraph also alludes to my best tip for writing an About page on your website. Don’t make it about you directly. Make it directly about what you’ve helped clients do and only indirectly about you.

\subsection{Consistency}

The next accelerant is consistency. Showing up consistently. Publishing consistently. Connecting consistently. Participating consistently. Caring consistently rather than caring only when you’re a month or two away from needing the next client.

In some ways, consistency is simply the necessary followup to your decision about how to specialize. In other words, it’s kind of table stakes if you want to specialize. And in another way, consistency really is an accelerant because it’s rare because so few of us are actually good at being consistent! Its rarity helps you stand out, which accelerates the speed at which you can become known for whatever it is you’ve specialized in. In other words, consistency makes you more \emph{visible} to those you are trying to connect and build trust with.

\subsection{Thinking like a media company}

The final accelerant might be thought of as a grouping of several previously mentioned accelerants. I refer to it as “Thinking like a media company”.

Media companies use content to attract an audience and monetize that audience through subscriptions, ad revenue, or creepy shit like gathering and selling personally identifying data, etc. If you think about your company using content to attract an audience and monetize that audience by offering your services to them, then you’re pretty close to what I mean by acting like a media company.

If you ignore the revenue model, what makes HBO different from prime time broadcast TV?

There might be multiple answers to that question, but the main one is: different content! HBO has invested in different content—and made much deeper investments in some of that content—than broadcast TV. Not just different shows, but a whole different category of shows.

To be clear, you don’t \emph{have} to act like a media company. There are plenty of great professional services businesses that sustain themselves just fine with referrals and a smattering of other bizdev activities that don’t involve producing a lot of content. But… if you adopt the \emph{mindset} of a media company, you will:

\begin{itemize}
\item \textbf{Work hard to make sure any content you produce attracts a monetizable audience}. You won’t blog (or podcast or whatever) casually or sloppily; you’ll produce interesting media that’s interesting and compelling to the right kind of people. People who are worried about risk, serious about change, eager to turn technology into business results, etc.
\item \textbf{Think about the content you produce as a product}. It will be free, but you’ll want it to be good enough that you could sell ads against it, or people would be willing to pay a subscription to access it even though you’re offering it for free.
\item \textbf{Learn more about your audience}. The quickest path to irrelevance for a media company is to lose touch with their audience. People who care deeply about music don’t care about MTV for this reason. People who love U2 mostly listen to their older albums for this reason. Incidentally, learning more about your audience has numerous other benefits, not just content-production benefits.
\item \textbf{Commit to marketing in a deeper way}. You’ll stop treating the act of connecting and building trust with prospects as an afterthought or a reactionary but unpleasant necessity and start treating it as vital to your business.
\end{itemize}

This media company mindset will accomplish two important things:

\begin{enumerate}
\item Almost guarantee that you avoid producing shitty content marketing
\item Accelerate the speed with which you cultivate a reputation in the market
\end{enumerate}

\section{Chapter Summary}