\chapter{About the Business Value of Expertise}

Specialization can transform your marketing, which is awesome, but it can also transform \emph{you}, and that's the real reason to get excited about it.

The words \emph{skill} and \emph{expertise} are often used interchangeably, but to me they are useful labels for two different but related concepts. Let's agree, at least for the purposes of the rest of this book, to use \emph{skill} to describe the technical ability to get something done and \emph{expertise} to describe understanding business context, planning effectively, and guiding the deployment of skill to that the skill creates desirable business impact. In other words, skill gets it done; expertise ensures it has meaningful impact for your client.

\section{About skill}

Before I go any further, I need to assure you that I deeply respect skill. It has tremendous value to the human race, and it is well worth investing in acquiring skill. Skill is a necessary building block, not an optional one. I am sure that you have worked hard to acquire and refine whatever skill you currently command. So I hope you do not feel insulted when you hear me detail the numerous problems that flow from trying to differentiate yourself from competitors using \emph{skill alone}. Flour alone is not enough to make a cake, and that fact doesn't insult the value of flour, so the fact that skill alone is not enough to make your business special does not insult the value of that skill either.

If your business model is based on renting access to your skill alone, then you have a three-pronged problem: 1) Skill markets tend towards commoditization, 2) skills are easily replicated, and 3) differentiating based on skill alone is almost impossible. Let's understand each of these problems.

\subsection{Commoditization}

I believe modern businesses and human resources departments are what has defined skill as ``an ability to perform a discrete, standardized task or group of tasks''. Some specific examples:

\begin{itemize}
\item Building an API is a different skill than building the front-end code for an app that uses that API.
\item Coding in Ruby is a different skill than coding in Python.
\item Using the OpenCV computer vision library is a different skill than building your own low-level computer vision algorithm.
\item Interviewing users is a different skill than writing code.
\end{itemize}

From your perspective, your skills are a beautiful tapestry of abilities you have woven together in order to make magic for clients. But from the perspective of an HR person, that beautiful tapestry is a bunch of ``Lego blocks'' of discrete, modular abilities, and their standardized shape and ability to interlock makes them \emph{highly} interchangeable. And for better or worse, the way an HR person views skill is pretty much the way all of us who are renting access to skill view it.

This view of skill as a discrete, modular ability makes skill the most easily commoditized aspect of what you do for clients. Commoditization is generally good for the buy side of a market, which gets a choice between many very similar options (what economists refer to as \emph{fungible}) at a price range that tends to be low. But if a supplier can't build a ``3-legged stool'' featuring high quality, high consistency, and low price in their offerings, they won't thrive in a commoditized market. To be clear, no service supplier really wants to create or participate in a commoditized market for their skills. Instead, commoditized markets emerge when the supply side participants fail or don't try to create \emph{meaningful differentiation} between their offerings and the offerings of others in the market. The demand side of a market will generally apply pressure to commoditize it because a commoditized skill market is easier to buy from. In the world of self-employed service providers and freelancers with business models based on skill rather than expertise, this kind of commoditization dynamic is not the exception, it's the norm.

\subsection{Easily replicated}

Your skills are easily replicated by competitors. The prevalence of open source in the modern custom software development industry and the eagerness of many advanced technologist to share what they know makes it relatively easy to acquire new skill, and the longer a skill has been in existence the more freely available documentation there will be for that skill. The longer a tech platform has been around, the more libraries, frameworks, and best practices will exist to support it and remove the need for advanced skill.

While formal education certainly exists within the world of software development, its importance and usefulness tends to be outweighed by self education. Self educated developers are rarely penalized for a lack of formal education, and there are certainly no formal barriers to entry like the kind of licensing and professional standards bodies that attorneys, accountant, and medical practitioners face. This means that almost anybody can become a professional software developer. This is both good and bad for you, but mostly bad if your only way of claiming you're better than others is to focus on your skillset.

Becoming a competent software developer requires little money or specialized equipment. It certainly helps a lot to have real world experience, but compare yourself to an attorney or surgeon. If you have the time, you can build all kinds of software that closely approximates commercial software or things clients would hire you to build. Via cloud computing and open source you have access to the exact same tools that Amazon or Netflix run multi-billion dollar businesses on. Now think about someone wanting to become a litigator or surgeon. How can they practice those skills in a situation approximating real life?

\subsection{Difficult differentiation}

Remember in chapter 4 I compared the inherent objective observability of products to the relative lack of observability that services have because services are essentially delivered in secret. Remember in chapter 5 I pointed out the human default to not trust your claims. This means that even if your skills are vastly superior to those of others, it's difficult for prospective clients to observe, measure, and trust this superiority.

We try to counter this lack of observability by showing the artifacts of our work to clients. We request testimonials and case studies from clients. We get permission to use their logo in our marketing materials. We ask for referrals. We use third-party proof (ex: if the New York Times writes about you, you sure as hell are going to feature that on your website) anywhere we can. In some cases you can point to something publicly available and say ``I played a key role in building that''. Except for the last one, these are all post-hoc artifacts of our work, and we hope that they will at least partially address the fact that the actual \emph{thing}--the actual experience of working with us and the actual deliverable--are usually something only a few people can directly experience. The actual thing is intangible to our prospective clients, and we hope to make it tangible with these artifacts like testimonials, case studies, and so on. But these artifacts only go so far in creating differentiation. They're important, but by themselves they're not enough because you have lots of competition with very similar skill and similar artifacts of that skill.

The next attempt at differentiation tends to be \emph{process} and \emph{team}. We talk about how we have some kind of great process, and we talk about how amazing our world-class team is. The last resort for differentiation tends to be \emph{service}: how service-oriented our business is. We talk about how good, flexible, or frequent our communication is. How transparent our project management system is. How technically fluent our account managers are. That kind of thing.

These are valiant attempts at differentiation because they're talking about how your skill is supported with a robust process and other forms of competence. But that's exactly the problem: having a good process, team, and ability to provide reasonable customer service are all forms of \emph{competence}, not forms of \emph{excellence} and they're certainly not demonstrating an ability to create outsized \emph{value}. They're simply the table stakes you need to buy into the services business game. To be clear, lots of us manage to get into this game \emph{without} those forms of competence (guilty as charged). That's why we're so proud of ourselves when we finally achieve them. We learned them the hard way, on the job. And that is something to be happy about. But to be clear, competence in skill delivery is the minimum, not something that more demanding clients will get excited about. The fact that so many of us are excited about mere competence is actually a sad condemnation of the state of our profession rather than a point of actual pride.

\subsection{Time-sensitive}

In the world of software and technology, skill has a relatively short half-life. To understand why, there are three closely related concepts you need to know about. They are what's known as the Rogers curve, Ronald Moore's variation of this idea as described in his book \emph{Crossing the Chasm}, and Gartner, Inc's elaboration of the same idea known as the Gartner Hype Cycle.

The Rogers adoption curve comes from a 1962 book by Everett Rogers called \emph{Diffusion of Innovations}. By dividing us--all of us humans--into 20th percentile groups, the Rogers curve actually describes pretty well how new technology is adopted over time.

\{Rogers curve illustration, based on the one from Wikipedia\}

The fundamental ideas behind the Rogers curve are change and risk. A very few of us, roughly 2.5\%, see an opportunity in the change that new technology presents and have the risk tolerance to eagerly and quickly start using that new tech, despite all the risk of it not working or turning into an expensive, underperforming boondoggle. These are the Innovators at the far left of the Rogers curve. On the far right of the curve are the Laggards--16\% of us!--who have the exact opposite relationship to change and risk. Change is seen as a problem, not an opportunity, and they seek to minimize risk as much as possible. This leads Laggards to only reluctantly accept change, and probably with a fair bit of grumbling and risk mitigation efforts along the way. Between Innovators and Laggards lie three more groups, each with a different style for adopting new technology that flows from their relationship particular to change and risk.

Geoffrey Moore evolved this idea in a seemingly small but very significant way.

\{Crossing the Chasm illustration\}

Moore inserted a ``chasm'' between the Early Adopters and the Early Majority. This chasm represents the difficulty that new technology ideas and products, especially disruptive ones, have in gaining adoption among the Early Majority.  In fact, ``crossing the chasm'' is so difficult that many new products never make it across.

Although Early Adopters and the Early Majority are depicted as existing right next to each other on the Rogers adoption curve, in reality they are quite different in terms of \emph{what they want} from new technology. The Early Majority is more pragmatic than the Early Adopters are, and this pragmatism leads them to expect a significantly more robust, reliable product than Early Adopters are willing to accept. This boils down, once again, to risk. The Early Majority is looking for risk mitigation in the form of what Moore calls a ``whole product'', meaning the new tech has an ecosystem of support, services, and best practices available to support its successful integration.

Finally, Gartner, Inc. has developed a very useful concept they call the Gartner Hype Cycle, which looks at the Rogers and Moore adoption curves from the perspective of customer expectations relative to the reality of the new idea or product.

\{Illustration of Gartner Hype Cycle\}

When a technology innovation is brand new, the hyperbole around it is quite high; unrealistically so. The press--especially the segment that uses a click-driven advertising-based revenue model--contributes to this unrealistically high level of hype. The press has discovered, for better or worse, that we want to feel safe by understanding threats. That is a basic human need that the press meets (or exploits) by creating hype around new technologies. Their coverage of new tech answers the following implied questions:

\begin{itemize}
\item ``Who is totally screwed if they don't know about this new thing soon enough?''
\item ``What implications does this new thing have for our future?''
\item ``Who wins and who loses because of this new thing?''
\end{itemize}

If you want a very thoughtful, evenhanded answer to questions like those, you'll pay an analyst or pay to subscribe to a publication like Ben Thompson's Stratechery. And if you instead donate your clicks and ad views in exchange for answers to those questions, you'll tend to get overhyped, over-generalized, overstated answers in the form of what most press delivers.  This drives the first part of the Gartner Hype Cycle, where the hype far exceeds what's realistic for this new piece of technology.

That phase doesn't last forever. At some point--about the time the tech is trying to cross the chasm so the Early Majority can begin integrating it--the hype flips and becomes pessimism. Gartner calls this the Trough of Disillusionment. Again, the part of the press with an ad-based revenue model is eager to help push things further towards an extreme yet again, except this time to a negative extreme. Remember, they are serving a human need to feel safe by knowing about threats.

Except this time, the threat is understood by the press's answer to these implied questions:

\begin{itemize}
\item ``Who is totally screwed if they made a mistake by jumping on the wrong tech bandwagon?''
\item ``What implications does the shortcomings of this new new thing have for our future?''
\item ``Who wins and who loses if they don't know about the shortcomings of this new thing?''
\item ``Who can we shame for the shortcomings of this new thing or for mistakenly embracing this new thing when they should have known better?''
\end{itemize}

Instead of trying to warn you about missing out on the next big thing, the press is warning you about the now-apparent shortcomings of the next possibly big thing. They're discussing the same technology, but now talking about a different threat it poses. This drives the anti-hype part of the Hype Cycle.

If the technology is able to achieve mainstream adoption, it climbs the Slope of Enlightenment and things stabilize on the Plateau of Productivity. There's no hype here, just a broadly-accepted, realistic view of what the technology can and cannot do and how it can and cannot create value. The press stops trying to drive positive or negative hype here because the threats this new tech might pose are well-understood, and so those threats are not very useful for driving clicks and ad views, and the tech isn't really new anymore anyway.

If we superimpose the Hype Cycle over Moore's variation of the Rogers curve, the relationship between them becomes crystal clear.

\{illustration of the above with the Rogers segments labeled across the bottom and the Hype Cycle phases labeled across the top\}

Here's what all this means for a business that focuses on skill alone. There are situations where \emph{skill alone} makes your services valuable, and you don't need much of a differentiator. Those situations all lie towards the left end--the head--of these three curves.

\{previous illustration updated: Moore/Rogers and Hype Cycle curves are lightened now and a new curve is overlaid. The new curve is the value of skill alone, and it's highest at the head, stays relatively high across the chasm, and then starts to diminish through early majority, late majority, and then is lowest for the laggards\}

Skill alone can make your services valuable when a new technology is very early on in its lifecycle because:

\begin{itemize}
\item There are few alternatives to your skill simply by virtue of the technology's youth, so it's relatively easy for buyers of your services to find you and choose you from among the relatively few alternatives to hiring you. Scarcity drives the value.
\item Your buyers are comfortable with risk because they are Innovators or Early Adopters.
\end{itemize}

Those situations where skill alone makes your services valuable are \emph{time sensitive}. They cluster around \emph{new technology}, and they don't last forever. Once that new technology isn't so new anymore and is supported by a whole product ecosystem of support, services, best practices, frameworks, libraries, and other forms of abstraction and ``implementation insurance'', skill alone loses much of its value.

If you're a typical young software developer, you're excited by this time-sensitivity because it gives you something to aim your ``learning cannon'' towards. It delivers periodic ``cocaine pellets'' of satisfaction as you conquer a series of learning curves, and when the skill is no longer scarce and valuable, you're fine with moving on to the next hot new thing.

If you think of yourself more as a business owner, this time-sensitive quality is somewhere between a neutral fact of life at best, a nuisance, or significant risk at worse. And if you're focused on using expertise as the raw material to build an awesome career that last 20 years or more, you'll do everything you can to wrap time-sensitive skill in evergreen expertise.

\subsection{Easily replaceable}

Focusing on skill alone makes your business easily replaceable. Remember that skills tend to be defined as discrete blocks of ability. This modularity makes it easier for the business to build a human-powered system, but those human building blocks are meant to be like the interchangeable parts that were such a huge part of the industrial revolution. If all you have to offer is skill--especially one that is a more mature skill--you will make your business easily replaceable.

\subsection{Own vs. rent}

If your skill is strategically important to a company, that company will prefer to ``own'' rather than ``rent'' it. Owning the skill means hiring a full time employee or building out an internal capability to deliver that skill rather than renting it, which means working with outside contractors, freelancers, or consultants. I don't think you can make a convincing case that owning a strategically-important skill \emph{actually} produces superior results to renting it, but that doesn't stop companies from acting as if it does.

Startups are one place where you can see this very clearly. Digital product startups do work with outside contractors, freelancers, or consultants, but they'd always prefer own strategically important skills because of how this effects company valuation during funding or acquisition rounds. They're incentivized to own rather than rent skill by incentives that have little to do with actual company performance.

Again, none of the above is meant to denigrate the value of skill. I simply want you to understand the limited ability of \emph{skill alone} to make your business valuable. Skill alone is quickly commoditized, easily replicated, and does little to make your company uniquely different from competitors. If your attempt at differentiation is to claim ``My company will meet spec and deliver this project on or under budget'', then you are absolutely setting yourself up to play the commoditization game. You are \emph{inviting} clients to think of you as a commodity.

\section{About impact}

I often say: skill gets it done; expertise creates impact. If skill gets it done and expertise creates impact, what is impact?

There are multiple ways you can think of impact, including:

\begin{itemize}
\item Improving your client's condition. This is Alan Weiss' way of describing the goal of consulting, and it's a good, broad way to think about impact.
\item Creating an observable--possibly measurable--change for your client. Often referred to as ``moving the needle'' for your client.
\item Reducing the risk of change, either a change the company has initiated (integrating new technology, for example), or risk from the outside the company is reacting to (complying with new government regulations, for example).
\item Helping a client create new opportunities, like entering a new market, for example.
\item Changing the broader culture of an industry or type of company. This is a change that's bigger than any one business, and is often done from a position of thought leadership. As an example, you might think of how test-driven development has become a popular working method due to the work of multiple thought leaders repeatedly making a case for the value of this approach. Their thought leadership has been impactful in the world of software development.
\end{itemize}

Skill, and an ability to work with clients to create impact, are relevant at all points on the Rogers curve, but their roles and relative value change as you traverse the curve.

\{TODO: consider superimposing several lines on the hype cycle, including perceived risk of deploying the skill, something about commoditization, etc.\}

At the head of the hype cycle, skill alone is often enough because \emph{nobody} really has real expertise deploying that skill, unless you have expertise with some adjacent skill (ex: your expertise with previous generation wireless sensors positions you well to gain skill with IoT and add value with IoT projects). Additionally, at this early date you are selling to innovators and early adopters, who have an entirely different relationship to risk than the rest of the groups on the Rogers curve. They're perspective is more like ``Let's see what this can do!'' rather than ``who do we blame/sue if things go wrong?''.

As you enter the middle of the hype cycle, skill starts to become commoditized and expertise becomes relatively more important because you're dealing with a much less risk-averse and much more demanding group of buyers: the Early and Late Majorities.

As you enter the rightmost region of the hype cycle, skill becomes very commoditized and expertise remains important but gains a competing and much cheaper alternative: best practices, libraries, toolkits, and frameworks. At this phase, risk generally declines, except for edge cases. Expertise still plays a role because despite its maturity, technology is complex and easily screwed up, even when supported with a full ecosystem of support, best practices, a mature services ecosystem, and other forms of risk mitigation.

Let's pull a few representative examples to build a timeline of how much time tends to pass before a given tech platform becomes mostly commoditized.

TODO: timeline constructed based on mobile, and a few others.

\section{About expertise}

If skill gets it done and expertise creates impact, what actually is \emph{expertise}?

There are multiple ways you can think of expertise, including:

\begin{itemize}
\item Expertise guides the deployment of skill. Skill knows how to do it, expertise decides what to do and knows why to do it.
\item Expertise guides, informs, or improves decision-making. It reduces uncertainty in decision-making, or offers a methodology to reduce uncertainty.
\item Expertise improves business outcomes in ways that skill alone cannot because expertise incorporates more context than skill does. Expertise can be a force-multiplier for skill.
\item Expertise helps you predict and manage the second-order consequences of applied skill or other causes of change.
\item The output of applied expertise is business impact; the output of abstracted expertise is best practices or other intellectual property (IP). IP in the world of services is expertise made usable by non-experts.
\end{itemize}

It often seems to me that expertise--if we can think of it as an elemental force of nature--seeks venn diagram overlaps between impact and risk in the world of business. Like water seeks to run downhill, expertise seems to seek out this particular combination of heightened impact and heightened risk. Experts feel a sort of gravitational pull towards this overlap between impact and risk, and that's how they choose where to deploy their curiosity and energy. I suppose you could also explain things through survivorship bias. It's possible that experts just ``follow their muse'' and the market rewards those that focus on the impact/risk overlap with more successful businesses. Either way, there's a strong relationship between expertise and this risk/impact overlap in the world of business.

The differences between expertise and skill make it possible to build two different kinds of businesses. They're both considered services businesses, but that's where the similarities end. I refer to a business where skill is the primary asset as an output-based business. And one where expertise is the primary asset is an expertise-based business.

\subsection{The output-based business model}

In an output-based business model, you're most likely promising outputs that are measured in the following ways:

\begin{itemize}
\item How well does it meet spec?
\item How much did it cost to build (measured solely in units of time)?
\item You might measure bugs, reliability, or other forms of functionality.
\item You might measure quality as well.
\end{itemize}

At the \emph{advanced} level of this business model, you're promising outputs that are measured differently:

\begin{itemize}
\item What potential value could this create for your client, and what actual value did it create after it was deployed?
\item ROI on the cost of building, where it's assumed that both client and service provider should get positive ROI from the project. Could be thought of as ``profit sharing'' in that the service provider realizes a good profit on the work and the work produces a good ROI for the client.
\end{itemize}

Implementation tends to play a critical role in the output-based business model. Providing advice and helping your client make better decisions will be an inevitable part of the output-based business, but these elements are secondary to how this kind of business works.

\subsection{The expertise-based business model}

With an expertise-based business model, you're most likely promising things measured in the following ways:

\begin{itemize}
\item Impact or transformation
\item Moving a revenue or cost needle
\item Risk mitigation
\item Gains in competitive advantage
\end{itemize}

If you can deliver on these promises then it is almost certainly because you are bringing a combination of expertise and implementation capacity to the table, but you are leading projects with the expertise. This changes almost everything about your business:

\begin{itemize}
\item You are more likely to interact directly with client-side people who can commission and fund projects without having to check with their boss.
\item You are more likely to be involved upstream in the project and have influence over key decisions about architecture, design, and other critical upstream choices.
\item You are more likely to be able to charge for your advice alone, or to frame having access to you as a valuable form of insurance that ensures the project goes well.
\item You can create and monetize intellectual property (IP), which is your expertise packaged in a way that makes it usable by non-experts.
\item You are more likely to push back against client assumptions or decisions. You lead in the project more than you follow.
\item Your sales conversations with clients change. You still discuss scope and specifications, but you probably spend more time asking simple questions about motivation and context. You spend more time helping your client explore other possibilities and less time assuming that their a priori assumptions and decisions are valid. And overall, you're more willing to walk away from a potential sale where the client won't let you lead.
\end{itemize}

\section{Expertise and marketing}

I often say: expertise is both the cause and effect of good marketing. This paradox is at the heart of how you rapidly cultivate self-made expertise.

If you're willing to do something quite simple that seems incredibly risky but actually is not very risky at all, you can use your marketing itself to accelerate your cultivation of valuable expertise. This is what people in my Expertise Incubator program do, so I'll outline the structure of that program as a way to get you thinking about how your marketing can feed forward into your cultivation of expertise.

\subsection{Marketing as an expertise accelerant}

If you're following my framework, you \emph{don't wait until you've cultivated this deep, valuable expertise to share it}. You share it as you go.

This part is \textbf{CRITICAL}.

You share. it. as. you. go.

This does two things.

First, it helps you more quickly develop the reputation you're trying to develop. We humans are amazing at ignoring things we deem irrelevant, but repetition--showing up over and over again in a consistent way--helps defeat this human default of ``ignore everything we can''. It helps you earn attention.

Second, it accelerates your cultivation of deep expertise.

Ponder this image for a moment, because this is basically what I'm suggesting you do to accelerate your cultivation of expertise:

TODO: 36th chamber image

That's a scene from \emph{The 36th Chamber of Shaolin}. A student is training by carrying heavy buckets of water and those little sword things attached to his arms will poke or stab him in the side if he drops his arms too much.

Bear with me here, because this is going to sound a little crazy. I am in fact suggesting that you do the following three things:

\begin{enumerate}
\item Build a small audience you can share with. Make sure it's on a platform with a good feedback mechanism, like an email list.
\item Share with them on an aggressive schedule; at least 3 times/week. Intentionally work in public \emph{at the edge of your expertise}. Use their questions and feedback to direct how you deepen your expertise.
\item Intentionally use the fear created by \#2 above to speed up how quickly you cultivate expertise.
\end{enumerate}

When you share your somewhat embryonic expertise with an audience that can give you feedback, you are doing something that will create \emph{productive discomfort}. This will incentivize you to get better \emph{fast}.

The stakes are high, but lower than promising a client expertise that you haven't yet cultivated. In other words, you're leaning out over your skis, but not on a client project. Doing that on a client project would be a kind of malpractice, and that's not what I'm suggesting here.

Instead, I'm suggesting you build a small audience (a few hundred or thousand people on an email list will do) and publish to your list frequently about your area of expertise.

At first you'll basically feel like a fraud.

Over time, the questions \& feedback you get from your list, and your own fear of being so close to the edge of your expertise will force rapid improvement. If you're in the ``epidermis of expertise'', you'll quickly move to the dermis.

At the same time as you're cultivating deeper expertise, you'll also be building a reputation--aka a market position--as the go-to person for whatever it is you're talking about 3 or more days a week to your list. The self-critical part of you is saying right now that this can't happen, but I'm here to tell you that part of you is wrong. I've seen it happen too many times to believe your inner critic over my own experience. :)

Your inner critic is saying ``nobody would trust or work with an expert who doesn't have 30 years of experience and doesn't have all the answers.'' And again, your inner critic is wrong.

This is my solution to the ``chicken and egg problem'' of becoming a self-made expert.

It's not easy, and it's not emotionally comfortable, but it is something I've tested and can tell you works, and it's faster than other methods (formal education, for example).

\subsection{Expertise as a marketing accelerant}

Which of this pair of content would you rather put behind an opt-in gate?

\begin{itemize}
\item The 5 things you must know to get your development environment set up
\item How your startup can gain a competitive advantage by knowing when to use and when \emph{not} to use micro services
\end{itemize}

It's probably obvious: the first one comes from skill, the second from expertise (if they live up to the promise embedded in their respective titles).

Which of the two following talks do you think would be more likely to attract someone who can cut a check for your services?

\begin{itemize}
\item TODO
\end{itemize}

Which of the two following emails to an email list do you think would be more likely to get a response from list members:

\begin{itemize}
\item Tabs vs. spaces
\item The business impact of blockchain
\end{itemize}

OK, that was a trick question. We all know the tabs vs. spaces one would get at least some replies because it's a polarizing topic, even though it has very little to do with expertise or business impact. Let's try again :)

\begin{itemize}
\item Why well-documented code matters
\item The business impact of blockchain
\end{itemize}

Also kind of a trick question! It really depends who is on your list, doesn't it? But that gets to an important point. If your list was mostly topics like the first one, it's likely your list will mostly be made up of your peers. A list that frequently discusses topics like the second one is much more likely to attract folks who desire insight on the business impact of \{things\}, and those folks--surprise surprise--tend to be able to also spend money to achieve business impact.

I hope these examples have made my point for me. That point being: demonstrating real expertise in your marketing attracts folks who need that expertise, and those folks also tend to seek business impact and be reasonably well-supplied with money to aid their quest for business impact. Demonstrating real expertise in your marketing helps you connect and build trust with buyers for your services.

\section{Chapter Summary}

\begin{center}
\rule{3in}{0.4pt}
\end{center}

Parking Lot

\begin{itemize}
\item Profit is a lagging indicator of impact


\begin{itemize}
\item Modeling expertise


\begin{itemize}
\item TODO: the 3-layer skin model for expertise
\end{itemize}
\end{itemize}
\end{itemize}