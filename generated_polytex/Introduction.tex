\chapter{Introduction}

I wrote the first version of TPM as an excited evangelist of an idea, and I wrote the second version as a diligent researcher seeking greater insight and fidelity. Now I approach this third version as a more seasoned practitioner, hoping my considerably deeper in-the-trenches experience translates to clearer thinking, better articulation of that thinking, and more practical, nuanced, and useful recommendations.

The updates to version 3 are numerous:

\begin{itemize}
\item Standardized terminology
\item More context around the business function of specialization
\item Way more context around the nuances of specialization
\item More grounded, practical recommendations based on your business maturity
\item More information on the relationship between specialization, your market position, and other common understandings of positioning
\item More information on the role of expertise in specialization, and recommendations on how you might aggressively cultivate valuable expertise
\item A different packaging and pricing model, with the same distribution model


\begin{itemize}
\item A tiered pricing approach is often very effective for maximizing info-product revenue. That's why I chose it for the second version of TPM. I needed the revenue, and it did just that. My business has matured since then, in large part thanks to TPM. My primary goal with this possibly final version of TPM is greater reach combined with access to buyers, but with the table tilted towards reach. That's why I'm pricing at \$19 for the electronic version and \$39 for print-on-demand. I believe it helps achieve the goal of greater reach.
\item I'm sticking with the self-published distribution model because I need the opportunity to connect with, build trust with, and further support book customers post-purchase. My business still needs this low-cost marketing approach to allow me to keep executing on the mission I'm here for. I felt good about the content in middle tier of the old version of TPM, which included a bunch of helpful guidance for implementation, but with this version I'm taking some of that content and pushing it down into what would have been the base tier of the old version (and is now the only tier of the new version), and taking other parts of that old middle tier content and replacing it with a group of pretty affordably priced online courses. Those courses are an upsell to the book, offered online and within the book itself, and they're a nice way of letting customers self-select the right level of implementation support. Those that are execution machines can take the ideas contained in the \$19 book and run with them on their own. They won't be manipulated by a pricing scheme into buying the old \$99 middle tier of TPM ``just in case'' they need that extra info. They will have gotten an amazing ROI at \$19 or \$39. Those that need more detailed guidance on execution make that decision after making the minimal investment in the \$19 book and they can then move on to buy one or more \$150 online courses if they see the need. They'll be motivated buyers because they've proven to themselves the book was good but not enough detailed guidance. They'll get great ROI from these courses, and I hope they feel that they bought them based on an actual need they've demonstrated to themselves rather than the psychological pressure of a tiered pricing scheme.
\item I'm not at the point in my businesses maturity (yet) where a full mass-market \emph{distribution model} (partnering with a publisher or self-publishing on Amazon's Kindle store) makes sense because that would deprive me of low-cost access to my book customers via email, but a more mass-market \emph{pricing model} does make sense, because it trades revenue for reach, and I'm ready to make that tradeoff.
\item I keep wanting to do an audio version, and I keep putting that off because of my perception of how difficult it would be to get it \emph{right}. But maybe one of these days\ldots{}
\end{itemize}
\end{itemize}

\section{How to Use This Book}

Just read it straight through. Each chapter is better than the one before it! Chapter 12 is a sort of climax, but making the best use of that chapter depends 100\% on understanding everything that comes before it in this book.