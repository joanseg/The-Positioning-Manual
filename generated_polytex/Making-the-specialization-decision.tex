\chapter{Making the Specialization Decision}

@PICKUPHERE

In earlier versions of this book, I looked at specialization through the lens of this, which I called the Value-Specificity curve: TODO link to image

My admonition at that time was simple: just move from where you are towards the global maximum on that curve. Where \emph{is} the global maximum on that curve? I don't know, because it's unique to you! Just do it!

It was the right idea, but expressed in a general, hand-wavey way. I think I can do better now. It's still very much a unique-to-you decision, but there is a standardized 7-step process almost anyone can follow to help make the decision about how to specialize. I'll present this process in this chapter.

\section{Everybody starts from a different place}

We all approach the specialization decision from a different starting point. Yes, we can abstract this a bit and say if you're not currently specialized, you're probably some sort of generalist (you'll say yes to almost any opportunity to apply your skill). But when you look closer, everyone is approaching the specialization decision with a different combination of skill, risk tolerance, business experience, and personality attributes.

If everyone was starting from the same place, I could offer perhaps just one ``specialization decision-making recipe'' that everyone could follow. But if I did that, some people's head start would be overlooked, and others lack of a head start--or different form of head start--would present them with insurmountable challenges. So in reality, the decision is very individualized.

When I'm working with a coaching client, I'm looking at several factors to define their starting point. These factors include:

\begin{itemize}
\item Your previous client experience
\item Your skillset
\item Your interests
\item Your risk tolerance
\item Your financial strength or lack thereof
\item Other aspects of your personality, like curiosity and assertiveness
\item Other aspects of your life situation, like your ability to work harder or longer hours for a time, or lack thereof
\end{itemize}

Taken together, these 7 factors tend to define your starting point.

\section{Some want different things}

Just like everybody has a different starting point, everyone wants to achieve something slightly different by specializing.

Now if we abstract this, you get the superficial vs. deep specialization distinction I explained in Chapter 7. You either want to use specialization in a superficial way (to get some quick marketing wins, for example) or a deep way (to make it possible to cultivate exceptionally valuable expertise).

But when I actually ask my clients why they want to specialize, I get an interesting range of answers. Some specific examples:

\begin{itemize}
\item ``Just make my marketing easier/better''
\item ``Just tell me how to have a pipeline so I'm not hand to mouth''
\item ``I want better clients''
\item ``I want to be a powerful expert''
\item ``I want to be wealthy''
\item ``Someone moved my cheese and I want it back''
\end{itemize}

Your answer to the question, ``what do you want to achieve by specializing your business?'' defines your end point.

With a starting and end point defined, we now have what it takes to draw a line, and the line between your starting and end point defines what you'll need to do to successfully specialize your business.

How you actually decide how to specialize is based on what you want, your starting point, and personal factors like risk tolerance.

\section{A Process For Deciding}

\subsection{1) Inventory}

All previous significant experience (through a lens of what vertical that experience is in)

All horizontal abilities to move the needle, both actual and theoretical

All entrepreneurial theses you're sitting on (difference between this and the above is that you might have to put together a team, or rent skill you don't have, or build something that requires up-front investment, or just generally take on more risk--perhaps by pursuing a specialization you just can't validate but you believe deeply in--as part of building out your market position)

\subsection{2) Extend}

All verticals that interest you

Horizontals that interest you

\subsection{3) Quantify}

Score all the items on the inventory according to:

\begin{itemize}
\item Access
\item Credibility
\item Impact
\item Profitability
\item Interest
\end{itemize}

\subsection{4) Rank and Eliminate}

Sort inventory based on aggregate score

Eliminate excessively risky items

TODO: consider including other elimination factors from SWF, like buying cycle length, etc.

\subsection{5) Enrich}

Size the market for top 3 vertical choices, make sure between 2k and 10k prospects

Identify options for reaching horizontal possibilities through networking or outbound. If none, think long and hard about the inbound marketing work you'll need to do, the skill and effort it will take to do it, and the time it will take to do it. I could see a little side-survey here where I present a curated list of lead gen options (based perhaps on trust velocity or something similar) and ask readers to think through what it would take for them to execute on 1 to 3 of them consistently.

\subsection{6) Validate}

You're looking to document relevant market interest, that's all

Options for doing this:

\begin{itemize}
\item Build a list of competitors, make sure you can find at least a few and that there aren't too many (\textgreater{}20)
\item Find conf talks that roughly map to your particular advantage
\item Find any other evidence that money is being spent in a way that maps to your particular advantage
\item Reach out to and speak with any competitors or non-competitive experts that you can
\end{itemize}

If you're pursuing an entrepreneurial thesis, then the above forms of market viability may not be present, so your best bet will be to:

\begin{itemize}
\item Reach out to and speak with potential clients/customers using a customer development style of interview-based market research
\item Additionally, you might create a strategy canvas to make sure your entrepreneurial theory targets a ``real'' opportunity in the market and validate that how you're differentiating makes good sense
\end{itemize}

\subsection{7) Decide}

You have enough information to decide now. You don't have enough information to guarantee that the decision will work out well (you can never collect this much information because there are factors that are impossible to predict), but you do have enough information to make an intelligent decision. So decide, dammit!

\{Upgrade to course offer via ``Go Deeper'' in-book sales pages. Course is a revised version of SS W1, revised along the lines above. There is no different process, there is only more detail and more hand-holding in the course. And maybe I'll throw in a DiSC profile as part of the price of the course.\}

\section{Chapter Summary}