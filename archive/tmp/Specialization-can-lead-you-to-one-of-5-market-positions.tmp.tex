\chapter{Specialization Can Lead You to One of 5 Market Positions}

Remember, specialization is the seed, time and discipline are the nutrients, and the part of the plant that's visible above ground is your growing market position (reputation).

You can cultivate one of 5 different market positions, depending on how you decide to specialize and nurture that seed of a decision. In the pure abstract, there's little reason to favor one of these market positions over the others, but once you start to apply this idea to \emph{your} specific situation with \emph{your} specific head start(s) (or lack thereof), interest, and appetite for various forms of marketing, then you'll start to develop a preference for one or more of these market positions. In other words, one or more of them will be a good fit for where you are now in your business, where you'd like to see it go, and how much of what kind of work you're willing to put in to see it get there.

These 5 market positions can be grouped into three groups of market positions with corresponding ways of specializing: vertical, horizontal, and one that I--mostly for lack of a better term--call blue ocean specialization. The first two are groups because they have subcategories within them.

\section{About Vertical Specialization}

This is focusing on a type of business or an audience of businesses or buyers. It leads to a vertical market position.

\subsection{Pure vertical specialization}

If you've ever heard someone say the \emph{\{fill in the blank\} industry}, you've heard a reference to what marketers and MBA types will refer to as a \emph{vertical}.

There's a useful list of verticals here: \textless{}https://www.naics.com/naics-drilldown-table/\textgreater{} Here's a few representative examples from that list: TODO

\subsection{Audience specialization}

A group of businesses that might not share the same vertical but share some other fundamental mission, need, desire, or business model form an audience. An audience specialization will lead you to cultivate an audience-focused market position.

You might consider all the businesses in the same city or locality an audience, but in the age of the Internet that is becoming less and less of a fundamental similarity. 20 years ago all the businesses in the neighboring town of Santa Rosa, CA might have used the same 2 or 3 creative agencies for their advertising needs. Now those same businesses are just as likely to meet those same needs over the Internet as they are by going to a local agency.

A more fundamental similarity would be businesses with the non-profit business model. Or businesses that primarily serve a LBGT customer base. Or family-owned businesses. Or businesses that are racing to get users for a new digital product (aka startups). You could find members of these audiences spread across many verticals, but their fundamental similarity would bring them together at conferences and industry associations. Their fundamental similarity will often extend to a shared worldview or narrative about what's important, and this allows you to connect and build trust with them as a group with shared needs, desires, and watering holes, and this makes an audience a viable way for you to specialize.

Specializing vertically is useful because:

It's relatively easy. Check out the LinkedIn advanced search feature. The list of industries there doesn't map 100\% to the NAICS list, but it's close enough to make finding prospective buyers using a LinkedIn search pretty trivial, at least for a pure vertical focus. This makes outbound marketing easier as well. The marketing message is also probably easier to figure out. A typical example will be moving from something like ``Elegant solutions to complex problems'' to something like ``Custom software for retail financial services''.

Word of mouth tends to spread most readily in a vertical fashion. Conferences, industry publications, asking a colleague for a recommendation, etc.

Past clients who leave their employer are most likely to land at a new employer in the same vertical. They can take you with them, or at least open doors for you at their new employer. This helps you ``stack'' related experience more readily.

It's unsexy and downright unappealing to lots of creative types (and devs are most often Creatives). This is a barrier to competition if you're not put off by the unsexieness of it, but on the flip side you might just not be able to conceive of how focusing on a single vertical could possibly lead to an interesting career. You're probably wrong, but I get it. On it surface it does seem limiting, but if you've never gone beyond the epidermis of expertise, I can understand how you'd have trouble seeing the richness that lies at the dermis and hypodermis of expertise.

It can be easier to forge strategic partnerships with: other firms practicing different disciplines (marketing, accounting, law, etc.) but specialized in the same vertical, superconnectors within the vertical (podcasters, associations, conference organizers, etc.), and product vendors like specialized SaaSes, etc.

It shortens the growth path from skill to expertise; from coder to consultant. By serving a group of businesses that will have a lot in common with each other, you will find more opportunity and incentive to become conversant in what's important to the business, which will allow you to more quickly learn how to create impact in the context of that vertical.

Specializing Vertically is Risky Because:

Not every vertical is equally receptive to outsiders. Chapter 12 digs deep into how to make the specialization decision, so I won't cover that here. For now, just know that if you're truly an outsider to a vertical, trying to specialize in that vertical is a risk you should measure against other options.

Verticals can suffer business downturns that are localized to that vertical. 2007 would have been a truly terrible year for anyone to decide to specialize in the real estate, mortgage, or construction verticals. The economic pain that ramped up in 2008 was more widespread than just those verticals, but they certainly were ``ground zero'' for a lot of economic carnage, and a service provider unlucky enough to get excited about specialization and then decide to specialize in serving one of \emph{those} verticals would have been in for a rough ride.

Vertical specialization ultimately requires you care about business results, people, or business as much or more than you care about your technical skillset. For some folks, this will be no problem. In fact, it will increase their enjoyment of their work. The ability to converse intelligently with their clients about the personalities, power dynamics, relationships, and culture of their business along with industry tends and the specifics of competitors will be a powerful way for these folks to deepen the impact their technical skills can create, and they'll love every minute of this expanded, more consultant-like role. However, others will find these exact same things a distracting nuisance, and would rather focus on technical depth rather than acquiring deeper knowledge of their client's business. These folks should probably avoid vertical specialization in favor of something horizontal in nature.

It is possible, though very unlikely, that you will find client-side concerns about conflicts of interest if you pursue a vertical specialization. I can count on one hand the number of vertically specialized clients I've worked with who have actually had CoI concerns raised by their prospects. If you do have a prospective client voice this concern, there's an 80\% chance it's a late-stage sales objection. Their mindset has moved from excited to cautious, and they're now playing devil's advocate and trying to de-risk an opportunity they were previously excited about. (Ref the Blair Enns article about this). As they do this, they'll turn over every rock and leaf looking for a reason the proposed engagement might be a bad idea. A baseless CoI concern might surface, and you can most likely defuse it if you're prepared for it. (Give ideas on defusing, like suggest it impugns your professionalism that they even asked but your a mensch so you're not jumping down their throat, etc.). There's a 20\% chance that CoI is a real issue. It's a variation of the insider trading situation. It's unethical for someone to trade on information they shouldn't have access to because, as Matt Levine points out, this is a form of theft. So if you're running ads for Company X, you could ``insider trade'' on the information you yourself have gained from working with that company and help their competitor Company Y gain an advantage against Company X. You're stealing from Company X by using the insider information you have to help Company Y.

\section{About Horizontal Specialization}

This is focusing on a type of technology or evergreen business problem, with little or no concern for what vertical the problem shows up in. A horizontal specialization leads to a horizontal market position.

\subsection{Platform specialization}

I don't love the terminology here since platform can mean different things to different people, but I don't have anything better to refer to the following group of things because from a specialization perspective, they work fundamentally the same way:

\begin{itemize}
\item A programming language (ex: Python, Ruby, C\#)
\item A framework (ex: React, VueJS, Laravel)
\item An actual platform you can build stuff on (AWS, Linux, Windows, Salesforce)
\end{itemize}

With a platform specialization, you are staying close to the skill end of the skill-expertise spectrum and going deep there. This is, at least for software developers, the most natural and comfortable way to specialize. The very complexity and breadth of skills that fall under the umbrella of modern software development almost mandates at least \emph{some} skill-based specialization. However, except in rare cases, this form of specialization \emph{alone} is not enough to achieve the full benefit of specialization. Why?

If skill alone is what you're selling, then low barrier to entry encourages race-to-the bottom price competition. So you've got price competition from below.

You'll have skill competition too, often from \emph{above} as folks who have a better position than you (ex: members of the core team who also do consulting, etc.) enter the supply side of the market.

Businesses are set up to commoditize and (often) internalize skillsets and de-couple those from specific individuals to make the whole thing more machine-like, systemetizeable, and robust in the face of personnel changes.

Combine the above three factors and you get a situation where it's difficult to differentiate based on skills alone. It's hard to have a good, purely skills-based answer the question ``There are 500 other options with basically the same skillset; many of them cheaper, some with more impressive experience. What's special about you?'' You have to bring something more to the table, and that's what specialization ultimately does for you.

There is one situation where platform specialization works well, and it's when the platform is at the left side of the Rogers curve/Hype Cycle. If you maintain a platform specialization over time, as that platform moves rightward the environment becomes increasingly less able to support an expertise-based business and it becomes increasingly more likely that you'll have to build an output-based business powered by scale, operational efficiencies, and (perhaps) cost of living arbitrage. Reminder: I'm talking about a \emph{pure} platform specialization here, not a multidimensional specialization that includes a platform specialization as the horizontal component. The multidimensional specialization lets you recruit additional advantages to compensate for skill commoditization as you move to the right of the Rogers curve.

Except for the above multidimensional specialization, I always encourage folks to avoid platform specialization because it largely deprives you of the opportunity to make the skill -\textgreater{} expertise leap. If you're pursuing \emph{superficial} specialization because you aren't interested in making this leap, you would be right to discount my warning a bit, since you're not interested in the thing you're depriving yourself of. My warnings about competition from above and below still apply, though, and so your best bet is probably to adopt a loose vertical constraint around your platform specialization (ex: Salesforce consulting for the professional services industry) to create at least a little differentiation and address at least some marketing inefficiencies.

\subsection{Pure horizontal specialization}

This is a type of specialization that focuses mostly on \emph{problems} that are visible to business decision makers. A horizontal specialization leads to a horizontal market position.

You could certainly specialize in solving a problem extremely few businesses have but care about deeply, or alternately you could specialize in a problem many businesses have but few care about deeply. But wouldn't it make your business more lucrative and business development easier to specialize in a problem that \emph{enough} business have \textbf{and} \emph{enough} of them care deeply about solving?

I've been doing this advising-people-on-specialization thing long enough to have seen a few examples of problems few businesses have but care about deeply, as well as examples of the other situations named in the previous paragraph. Let me illustrate a few for you:

\begin{itemize}
\item \textbf{Optimizing WordPress page load time}. There are businesses that care about this, but they tend to solve the problem by hiring a FTE or throwing money at hosting, not by hiring a specialized developer to do custom work. In a sales conversation they'll recognize the importance of the problem, but they'll rarely follow through with the action of hiring you because there always seem to be more important issues for them to address, unless page load time is a truly critical part of their business strategy in which case they'll treat it as a core priority and hire a solution for the problem that looks like an FTE.
\item \textbf{Migrating away from Heroku once it becomes cost-ineffective}. There are businesses that care about this, and it \emph{is} a situation of elevated risk combined with clear financial upside, which would lead you to think it's a great problem to specialize in. However, in reality it's more complicated. Most businesses will put this off as long as humanly possible because it involves \emph{so much} risk and change, and it's not directly related to their usual core strategic priority at the time Heroku's expense becomes a problem, which is product market fit or shipping features faster or something like that. Bootstrapped businesses will do the migration away from Heroku themselves because this is all the can afford, while funded startups are spending other people's money and not penalized short term for doing so as long as they're seeing progress on more important issues like PMF or user growth, so in this way it's a sort of anti-goldilocks problem.
\item \textbf{Escaping ``spreadsheet hell''}. There are a huge number of businesses that essentially run their back end on a rat king of spreadsheets. This tends to be hugely offensive to a developer with relevant skills in search of a problem to apply them to. They see that ``spreadsheet hell'' and instantly see the improvements that a database and custom software could bring to bear, and it's difficult for them to imagine how any potential client could suffer another minute of that spreadsheet hell. And maybe your prospective client even admits it's a sort of hell! But even if you're living in hell, there's a risk and switching cost in moving someplace nicer. What's that saying about the devil you know being better than the one you don't? There are plenty of prospective clients who \emph{will} pay you to help them move out of spreadsheet hell into a Customsoftwaresville, but there are also many who will put off making this move as long as humanly possible because they're superficially residents of a kind of hell but fundamentally residents of the right hand part of the Rogers curve, and by now you should know what that means for their eagerness to change.
\end{itemize}

So what kind of business problems \emph{can} you focus on and find strong demand from the market?

Well, when I survey the list of horizontally-specialized businesses listed in \textless{}http://specializationexamples.com\textgreater{}, I see some patterns:

\begin{itemize}
\item If they focus on a specific platform or technology (ex: VR, computer vision, machine learning), they tend to be ones residing on the left half of the Hype Cycle or one that's a genuinely difficult, frequently unsolved problem that requires significant experience and judgement to address (ex: Rails security).


\begin{itemize}
\item If they focus more on a business problem than a specific technology (ex: employee motivation, inventory management, digital transformation, compliance, security, monitoring, DevOps, helping big companies compete with more nimble startups, cloud service cost management), there are several shared characteristics:


\begin{itemize}
\item 1) There's an upside that somebody with budget and authority to spend it cares about, generally defined in terms of revenue increase, cost reduction, risk management, or competitive advantage
\item 2) There are enough businesses that lack an internal capability to handle this problem to form a viable market
\item 3) Technology is a part of the solution, but the ultimate solution is measured more in terms of business impact and less in terms of deliverables. In other word, the primary deliverable \emph{is} business impact. There's a saying about intelligent people having deeply-considered opinions that they'll change easily in the face of new information. \emph{Strong opinions, loosely held} the saying goes. This is the relationship that many pure horizontal specialists have with technology. Their allegiance is to business impact; the tech is a tool to achieve that end and it's a tool that is easily changed out or bypassed altogether in favor of non-technology solutions (change the process, hire a temp, etc.) if those would provide more impact or better ROI.
\end{itemize}
\end{itemize}
\end{itemize}

This leads to what might the most useful simple definition of a pure horizontal specialization: Using technology to solve a business problem or create some other form of impact a range of businesses in different verticals all care about.

Compared to a platform specialization, a pure horizontal specialization is relatively more concerned with business impact and relatively less concerned with technology skill, since in many cases the skill can be rented for a mere fraction of the ROI it produces, especially on the right half of the Rogers curve when commoditization has started to create a relationship between the supply of competent skill and the price being charged for that skill that's more favorable to the demand side than the supply side.

Specializing Horizontally is useful because:

\textbf{It's more naturally compatible with how developers think}. (If you're talking about a big group of people--and given that there may be up to 30 million professional software developers, I am!--you're going to generalize and miss the mark sometimes because of that generalizing. I apologize if I'm missing the mark about \emph{you}, but I've worked with enough of your peers to know I'm mostly right about most of these general group characteristics.) Software is ``eating the world'', and custom software is powerful for a number of reasons, but one of those reasons is its flexibility. A single category of tools with millions of potential applications. I can't think of another tool that exists in the world with this much utility and flexibility. You could think of emotional ``tools'' like empathy or resilience or curiosity that have more utility and flexibility, but that's cheating--those aren't ``out there'' in the world. Anyway, software is incredibly flexible. In fact, a lot of its utility springs from its flexibility. It's by nature a general-purpose tool. This leads developers to think in the same general-purpose way. ``Software can be a part of solving almost any problem! That means \emph{I can too}!'' Well, yes and no. Software is almost always a subsystem within a larger human, financial, or mechanical system. And insight into the larger system ranges from beneficial to mandatory in order to \emph{make software produce value}. That insight into the larger system is often mandatory in order for the software to achieve its goals. So you can learn \emph{once} how to build software and you have command over this amazing, flexible, general-purpose tool. But the \emph{context} in which that tool operates is not a single, uniform worldwide context. It varies based on \emph{what the software is being built to do}. In fact, the context varies a lot! But this is the part that's easy to miss. It's easy to focus on how flexible the tool is, and miss how varied the context is and how diverse and potentially troublesome those variations in context can be. Anyway! If I describe all the ways of specializing and ask the typical developer to choose based on their preferences, they'll tend to choose something horizontal in nature because it fits how they're used to thinking.

\textbf{It's easier for the typical dev to imagine how horizontal specialization will remain an interesting, deep challenge for them over time}. While it's not actually \emph{true} that horizontal is inherently more interesting than vertical, it \emph{seems} to be true because of the aforementioned ways that devs are used to thinking about things. This lowers the emotional barriers to entry to a horizontal market position for devs. There are so many \emph{experiential} aspects to what it's like to be an in-demand specialist that it's hard to convince ``non-believers'' that vertical specialization can be just as interesting as horizontal, but I'm telling you it can. To be fair, it \emph{can be} if it's compatible with you, and where you want to take your career. In later chapters I'll talk more about how to know yourself in a way that leads you to choosing the right form of specialization. Just know for now that vertical and horizontal both can be equally interesting. Neither one is systematically more or less interesting.

I'm not sure I can prove this, but I believe horizontal specialization (or a very narrow horizontal focus coupled with a somewhat more broad vertical focus) is a more suitable specialization model for the so-called ``lone wolves'', meaning solo consultants who have no desire to build a team or agency business. This is because the pure vertical focus generally offers a broad discipline (ex: marketing, management consulting, custom software development) to a specific vertical. The breadth of the discipline itself benefits from a team with a blend of capabilities. For example, sure, there are developers with lots of competence in full stack development, design, DevOps, and the other sub-disciplines needed to ship a software product. I've worked with folks who can handle everything from PCB design to mechanical and motor system design to embedded realtime system code. But\ldots{} a team of sub-specialists might offer a better value proposition if the promise is something like ``custom software for manufacturing'' (pure vertical focus with broad discipline), while a lone wolf could credibly live up to the promise of ``inventory optimization for manufacturing'' without the support of a team of sub-specialists.

Specializing Horizontally is risky because:

\textbf{It can require more work and lead time/runway and marketing sophistication to find clients}. One of the very fundamental differences between vertical and horizontal specialization is the relative absence of external signals of need from the latter. To illustrate this, let's compare how two different specialists might connect and build trust with clients:

Juliette specializes in marketing for franchises. This is a vertical focus, specifically an audience focus. Her marketing can be as simple as reaching out to franchises (they're easy to find because their business model is pretty clearly externally visible) and saying a version of this: ``Hi! I have no idea if you have an expert in franchise marketing in your corner already, but if you don't--or if your current agency doesn't understand franchises as well as you'd like--we should talk. I've been specialized in marketing for franchises for several years now and I'm sure I could bring some fresh ideas and much-needed depth of experience to move the needle for you. Again, if you've already got this covered, I'm really happy to hear that and please excuse the interruption, but if you don't and want to talk I'd be happy to connect.'' Now Juliette will almost certainly invest in other forms of marketing like content marketing, speaking, and so on. But\ldots{} if all she had was a list of email addresses for CMOs at 500 franchises and the email copy above, I'd bet real money on her landing a few exploratory sales conversations from that email list, that email message, and a bit of persistent followup.

Mike specializes in infrastructure and application monitoring. This is a horizontal specialization, specifically a pure horizontal specialization. He's not a platform specialist because he's not focused on a single tool, he's focused on the business problem of using monitoring to move some desirable needle like application availability (I suppose it's more accurate to say he \emph{prevents} that needle from moving away from where it needs to be at the very high uptime end of the scale:) ). How does Mike connect and build trust with prospective clients? Let's imagine he uses Juliette's approach and cold emails CIOs with an approach similar to Juliette's. The reality is he's going to get nowhere. He told me as much \{link to interview\}. What \emph{does} work for him? Writing an O'Reilly Media book, launching and editing a popular weekly email list on the subject of monitoring, and conference speaking have been far more effective lead generation approaches for Mike. This is what I mean when I say a horizontal specialization can require of you more patience and skill in your marketing.

\textbf{After you find them, you may be in for a longer sales cycle or an unpredictable/heterogenous sales process}. Horizontal specialists often (but not always!) specialize in business problems that are somewhat idiosyncratic \{possible not the best adjective\}. That's what makes them opportunities! The client hasn't developed an internal capability to handle the problem, few other service providers have the nose for opportunity and risk tolerance to specialize in this same way, and the problem is not an ongoing thing your client wants to outsource. If you're an agency selling marketing services, you can hook into a somewhat normal, standard process your client has for selecting and purchasing your services. But if you're solving a problem like Mike in the above example, there's almost certainly no standardized process for working with a monitoring expert. So the sales process is going to be much more improvised, at least on the client side. This can be good or bad. It's good if your buyer is like, ``Just make this happen. I can write you a check myself.'' You've just bypassed a normally costly process (at least in terms of your time) and closed a deal with a few conversations. But if you have to crawl an opaque web of decision makers or are dealing with a buyer who can't just cut a check without higher up approval, then it's not great that there's no standard purchasing process because you have to feel your way through a landscape that might not look much at all like the next client where you have to feel your way through a different landscape, adding time and effort (cost, really) to the sale. But! This potentially more difficult, complex sales process can be worth it if the problem you solve is important and urgently felt by your clients. So I'm not at all arguing for or against horizontal specialization here, just trying to alert you to something that may become a fact of life for you if you specialize in this way.

\textbf{It can be easy to confuse the difficulty of the solution with the \emph{economic value} of the solution}. This one is insidious, and leads to the heartbreaking specialization no-go situations described above (optimizing WordPress page load time, as one example). It's so easy for you, as a technologist, to see how difficult some problems are to solve and conflate the difficulty of solving the problem with the economic value of solving that problem. Or\ldots{} it's also easy to evaluate the economic value of solving a problem from your perspective as a technologist and not from the perspective of a business. Yes, businesses do seemingly dumb stuff all the time. Some of it \emph{is} actually dumb. After all, not all businesses succeed, and sometimes their lack of success is due to truly dumb decisions. But sometimes business do ``dumb'' things for smart reasons, and sometimes those ``dumb'' things lead to positive outcomes for the business. Maybe they operate in a regulated environment and do stuff that's technically dumb (ex: paper records instead of electronic) because of the demands of regulation, and the cost of regulatory risk or non-compliance is actually far higher than the cost of the ``dumb'' thing they're doing. Remember what I said earlier about expertise vs. skill. Skill sees this situation and thinks it's crazy and dumb; wasteful, inefficient, and shortsighted. Expertise sees the same situation and understands why the ``dumb'' decision was the \emph{right} decision. The antidote here is simple but not easy. It's to learn to see things from both your perspective \emph{and} your client's perspective, and to have enough insight into their perspective that you can almost read their mind. If you can make use of their worldview almost as fluently as your own, you'll be able to understand what problems could make for an economically viable pure horizontal specialization. There is a special exception here, which usually takes the form of developing specialized horizontal expertise that few businesses would pay consulting rates to access, but lots of your peers will pay individually small but collectively significant amounts of money to access. A few examples here: Adrianne Rosebrock with computer vision, TODO more. This works because instead of trying to sell expertise that few businesses understand, you ``aggregate'' the understood need for that expertise coming from your peers and sell your expertise to them in the form of training, coaching, mentoring, etc. Doing this often gets you corporate clients anyway, but it's a second-order effect of being successful selling training to individuals who then become sort of like an unpaid sales team for you (in fact they're paying \emph{you} twice! Once for the training and then a second time in the form of credibility, access, or both to corporate buyers).

\section{About Blue Ocean Specialization}

Finally, we have what I refer to as Blue Ocean Specialization. I have to confess, I don't love this term, but I can't figure out a better one. Over successive iterations of TPM I've standardized my terminology and that's good, but I just don't know a standard term for this kind of specialization so I have to invent one.

Blue Ocean Specialization is where you specialize your service delivery. This is often synonymous with productization (where you standardize scope and pricing for your services), or, more specifically, innovative service productization where you standardize your scope in a unique way that's attractive to a narrow spectrum of clients.

This approach to specialization is best explained through examples:
- \href{https://worstofalldesign.com/}{https://worstofalldesign.com/}
- \href{https://knapsackcreative.com/}{https://knapsackcreative.com/}
- \href{https://audienceops.com/}{https://audienceops.com/}

Blue Ocean Specialization is useful because:

\textbf{It can be an easy, low-risk integration to your existing business}. It need not be seen as a risky or disruptive move because at first it's ``just another service offering''. Not that you should be haphazard about adding service offerings (because every additional item on the menu increases cognitive load), but it is something you can treat as a service offering experiment rather than a significant change in strategy. You can test it out on your existing lead flow and iterate based on what you learn by doing that. Based on the feedback and success of the service, you can start to invest more or less heavily in this offering and--if it makes sense--allow the service offering to ``take over'' your business.

\textbf{It can be done with little market research or insight outside of listening closely to your clients}. All the insight you need to develop a good blue ocean service can come from your clients. You look for patterns in what they need and build something that fits that pattern.

\textbf{It may not require any kind of extra marketing effort}. It can be as easy as adding a summary of the service to a website or PDF to start exposing it to your existing stream of leads. The first round of marketing could be as simple as emailing all your past clients or, if you're regularly marketing to an email list, starting to mention it there in your regular emails. This is all really low-effort marketing.

\textbf{It can be a safe way to test a new market}. If the blue ocean service is vertically-specialized--even if the ``host'' business is not--it can be a way to test a new market through outbound marketing. You can go as far as quarantining the vertically-specialized service on its own minimal website that operates as a sibling brand that's not at all or only loosely related to your main brand.

\textbf{It can be a way to become comfortable with the whole idea of narrowing or specializing or focusing}. Think of it like a ``specialization internship'', where you experiment in a low-stakes way with some of the realities of specializing. Will you get bored doing this standardized service over and over again? Why not find out with the scope constrained to a single service rather than across your entire business? Will solving a somewhat consistent problem over and over again really allow you to stack experience and gain real expertise? Why not find out within the small ``sandbox'' of a specialized service first? \{maybe footnote my My Content Sherpa experiment?\}

Blue Ocean Specialization is risky because:

\textbf{It's easy for all of us--even experienced marketers--to imagine patterns that aren't really there or miss patterns that really are}. That's because we're human and have all sorts of sophisticated mental filters that work great for basic survival but prevent us from accurately measuring real-world opportunity or risk.

\textbf{It's easy to build a ``Frankenstein'' blue ocean service that doesn't make sense}. It's surprisingly easy to get excited about the idea of building a productized service, and then build a productized service that doesn't sell well. I've done this myself. Twice! Think about all the Kickstarter projects that have failed to reach their minimum funding goal (62.88\% of all Kickstarter projects at the time of this writing). I wouldn't be surprised if the same rough succeed/fail split exists with productized services for the same underlying reason: many products are designed and marketed with a solution-first approach rather than a market-first approach, and the resulting product just doesn't resonate with what the market truly desires because the desires of the market were ignored or under-appreciated while designing the product. Like I said, I've done this twice myself. I wouldn't call My Content Sherpa (the first) and Drip Sherpa (the second) failures, because they did sell and support me for a while, but they also weren't long-term successes the way other services like Audience Ops have been.

\textbf{It might be too small of an experiment, or an initial product-market-fit failure might be discouraging}. Building on the above point, if you're looking at a productized service as a market validation test, then a failure of the test might be a) irrelevant or b) discouraging. It might be irrelevant because the productized service was poorly designed (solution-first rather than market-first approach), poorly marketed, or poorly priced, not because the potential specialization approach you are testing is invalid. And it might be discouraging because you put a lot of up-front work into the solution and the general over-weighting humans do of potential losses distorts your perception of what a failed test means (the overweighting comes from either an actual behavioral aversion to loss, or an amplified emotional reaction to losses vs. gains \textless{}https://www.apa.org/science/about/psa/2015/01/gains-losses.aspx\textgreater{}). So rather than seeing the failure of the test as a thing that happened for a variety of possible reasons and a small step towards clarity, it's easy to see a failed test as a discouraging setback that should not be repeated in any form.

\textbf{It's possible to lock yourself into low profitability you can't solve without unacceptable comprise}. Also related to the solution-first point above, it's possible to construct a productized service that sells well but isn't profitable to deliver. I did this with both of my services, because to scale correctly they required that I build and manage a team. I joke that most introverts would rather eat a bowl of broken glass than make a cold call for business development, but I'd almost rather eat a bowl of broken glass than manage a team of people.

Surprise! There's actually a sixth kind of specialization, which is really an amalgam of two other kinds

\subsection{Vertical + Horizontal = Multidimensional}

When you combine a narrow vertical \emph{and} horizontal focus, you have what I call a multidimensional specialization.

For the conceptual purists: Vertical and Blue Ocean specializations always include some sort of horizontal aspect. That's because a pure vertical specialization would look--and actually be--ridiculous: ``Anything You Need for Manufacturing''. Even generalist freelancers don't make claims \emph{that} ridiculously broad. A pure vertical specialization would imply that the same firm offers everything from CEO coaching to supply chain consulting to legal services to IT support to catering to janitorial services. Even General Electric, which has the promise of generalism right there in its name, makes choices about what \emph{not} to do, and has a bigger list of what they don't do than what they do. :) So there's always an implied horizontal in any vertical specialization. For that horizontal component of a vertical specialization, there's a range from broad to narrow. Examples:
- Broad horizontal component within a Vertical specialization:
	- Marketing for Manufacturing
	- Management Consulting for Franchises
	- Staffing Services for Construction

In cases like these, the horizontal aspect is generally a \emph{discipline} (ex: marketing, custom software development, accounting, legal services) that's been focused on a specific vertical. In other words, without the vertical constraint, you'd just be another generalist because generalist tend to focus on a discipline without a vertical constraint. But! When you add in the vertical constraint then you've started to specialize in a way that creates marketing efficiencies and the potential to cultivate exceptionally valuable expertise.

\begin{itemize}
\item Narrow horizontal component within a Vertical specialization:


\begin{itemize}
\item Inventory Optimization for Manufacturing
\item Outbound Sales for SaaS Products
\item TODO: more examples
\end{itemize}
\end{itemize}

This is the more audacious choice.

The difference is this: the narrow horizontal + Vertical specializations generally choose a single problem to focus on within their vertical specialization, while the more broad horizontal + vertical specializations focus on a broad discipline (marketing, accounting, legal, software development, etc.) applied to a specific vertical. Another difference is this: if you're currently a generalist, interested in specialization but intimidated by how to go about it, a clear vertical focus combined with a broad horizontal focus where the horizontal matches what you're doing now (marketing, full stack development, etc.) is almost certainly your best entry point to the world of specialization because it gets you many of the advantages of specialization with a nice risk/reward tradeoff and sets you up for more narrow specialization in the future as you're ready and more confident thanks to the success of your first specialization move. In other words, clear vertical + broad horizontal is the ideal first specialization move for a beginner, while the clear vertical + narrow horizontal is more suitable for folks who have more experience in their self-employment career or are willing to take more risk or do more research/validation.

So\ldots{} some sort of multidimensionality to your specialization is normal, not rare. That said, it is more bold and risky to go very narrow on both the vertical and horizontal dimensions. The potential upside is that this multidimensional narrowness makes it much easier to find your ideal clients (ex: Matt in Istanbul and the 6 local clients), connect deeply or frequently with them, become an insider to their world, and cultivate exact-match expertise for their most pressing and valuable problems.

\subsection{Blue Ocean + another kind}

You might combine Blue Ocean specialization (a unique service delivery model) with a vertical specialization, a horizontal specialization, or both. This is rare but not unheard of. Some hypothetical examples:

\begin{itemize}
\item \textbf{Fixed cost cloud data compliance monitoring for HIPAA-regulated data}. Pretty narrow, specific Blue Ocean + Vertical + Horizontal components are present in this specialization. This description reads a lot like the description of a SaaS, but it could also be a service powered by humans and a bit of custom software at first and then morph into a SaaS as you understand the problem space and customer needs better after delivering it as a human-powered service for a year or three.
\item \textbf{Content marketing as a service for healthcare businesses}. This specialization includes Blue Ocean + clear vertical components.
\item \textbf{App mockups and a pitch deck in 1 week for a fixed price}. This specialization includes Blue Ocean + narrow horizontal components.
\end{itemize}

Some real examples:
- TODO

A bunch of good, clear examples and case studies
- TODO

\section{Chapter Summary}