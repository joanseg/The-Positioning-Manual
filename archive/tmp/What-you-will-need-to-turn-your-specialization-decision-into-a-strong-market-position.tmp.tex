\chapter{What You Will Need to Turn Your Specialization Decision Into a Strong Market Position}

Once you've decided how you want to specialize, it takes time, discipline, courage, and follow-through to turn that decision into a strong market position. Here's specifically what you'll need to do that.

\section{A declaration of focus}

You need your new focus to be very clear to prospective clients. So how you declare this focus to prospects depends very much on how you connect and build trust with clients.

Ironically, the easiest \emph{and} most difficult starting point for declaring your new focus is your website.

It's difficult because most of my clients are not experienced marketing writers, and additionally they're used to writing websites that focus on themselves rather than on their clients needs and problems. This means when they sit down with a blank text editor or CMS screen, they kind of freeze up, and what they end up writing kind of sucks. There's generally the perception that because your website is so public and anyone literally anywhere in the world can see it, it's a high stakes change. It feels like your website is a megaphone that's powerful enough to reach the entire world's population. And in a way it is. HOWEVER\ldots{}

At the same time, the easiest place to start declaring your new focus is your website, because \emph{in reality} it's probably a low stakes change. Most of my clients have a website, but when I ask them how many leads they can attribute to their website, they answer something like ``1 a month'' or more usually ``1 a \emph{year}''. So your powerful ``megaphone'' is in reality not reaching all that many people at all. A sandwich sign at a busy street intersection would probably be more effective at generating leads. :) So even though it's perceived as a high stakes change, updating your website to declare a new focus is more likely a low stakes change for you, so it might make sense to start declaring your new focus there.

There are other venues where you can declare your new focus:

\begin{itemize}
\item Email marketing content
\item Social media presence(s), like your LinkedIn profile, etc.
\item Sales conversations with prospective clients, where you verbally declare your focus in the privacy of that conversation
\item Off-site content marketing (done off your website, like talks, podcasting guesting, etc.)
\item Print marketing assets
\item Your personal and professional network
\end{itemize}

\subsection{Some tips for declaring your focus:}

Keep it simple. Like, ``cave man speak'' simple. Prospects will appreciate the lower cognitive load, and feel like you don't have anything you're trying to hide behind excessively complex language. Don't try to perfectly qualify clients on your site by having a long checklist of stuff you're looking for. That overly complicates things for prospects, runs the risk of alienating good prospects by making you seem strangely picky, and adds friction. Now if you're at a place in your career where you're drowning in good leads, of course ignore this advice and add the right kind of friction to make sure you only spend valuable phone time with good prospects. But those earlier in their career can gain an outsized benefit from one simple thing: \emph{more conversations}. A simple message on your site can help avoid the friction the reduces the number of conversations you get the opportunity to participate in, so keep things simple!

Be clear and avoid waffling. If in doubt, use one of the following recipes for your site headline:

\begin{itemize}
\item \{thing you do\} for \{market you do it for\}


\begin{itemize}
\item Ex: Custom software for manufacturing
\item Ex: I help companies fix their horrifying AWS bill
\item Ex: DevOps as a service
\end{itemize}
\item TODO: add some more positioning statement recipes
\end{itemize}

Frame your message in terms of your clients needs, desires, and how your focus benefits them

Wherever possible, let your previous clients do the talking for you through stories about \emph{their} successes (with your help, of course), quotes, testimonials, case studies, and using their household-name logo on your site. If you have a choice between saying ``we have deep skill in \{thing\}'' and a previous client saying the same thing through a case study, testimonial, or quote, then use the version where your client says it.

Wherever possible, talk about how your process or unique IP facilitates positive results, but only if your process or IP is sufficiently unique. If an objective outsider would lump your process in with an industry-standard process like agile, XP, or the old standby ``design, develop, deploy'' 3 or 4-stage ``process'', then don't emphasize your process because it's not sufficiently differentiating. Taking something that's the bare minimum and talking about it like it's something special is embarrassing, because it makes you look like a n00b. This would be like a construction contractor bragging about having general liability insurance and saying that makes their business special.

At first, don't worry too much about scrubbing your website for old blog articles, portfolio items, or case studies that are inconsistent with your new focus. Eventually you will want to do this, but it's not a top priority at first. At first, nothing is more important than your site's ability to articulate a clear, simple message about what your focus is, and then providing evidence that you are a credible source of skill or expertise within this new area of focus. Focus on those two things first, then later you can spend time going deeper into your site to scrub old content and replace it with new content that's more consistent with your new focus.

Finally, don't overload pages on your site. Each page should have at most one ``job'', with one exception: if you've decided to specialize and don't have much of a web presence at all right now, I recommend you set up a 1-page web site to start with, and build that out over time. Here's an example of what I mean: https://1pageleadgen.site. So on a 1-page website, the home page will have the job of declaring your focus, providing evidence that you're trustworthy, and inviting prospective clients to take some kind of next step.

\section{A platform for connecting}

You'll also need some sort of platform for connecting with people. Marketing types generally refer to this as ``lead generation''.

A lead is someone who has taken some kind of action that indicates a general level of interest in your business. This might include:

\begin{itemize}
\item Opting in to your email list
\item Trading an email address for some sort of content marketing asset like a white paper
\item Attending a talk you've given and then trading contact info with you afterwards
\end{itemize}

A prospect is someone who has expressed a more specific interest in working with you. ``Prospect'' is short for ``prospective client''. This might include:

\begin{itemize}
\item Emailing you to inquire about your services
\item Filling out a service-specific contact form on your website
\item Contacting you after a referral from someone who knows about your work
\end{itemize}

So you'll need a way to generate leads. There are lots of ways to do this. I've vetted and profiled the best ones here: http://trustvelocity.com

\section{A habit-based method for earning trust}

You'll also need at least one way of building trust with leads. Sometimes you can earn sufficient trust \emph{while} generating the lead, and sometimes it takes time \emph{after} someone has become a lead to earn their trust.

I believe that wherever possible, the activities you carry out to generate leads and earn trust with those leads should be \emph{habitual} for you. The two most damaging marketing failures I see with the kind of people I work with is a failure to focus sufficiently, and a failure to build a lead generation/trust-earning habit, and so as a result they go very long periods of time without doing either, and it causes a predictable feast-famine cycle. Now, these activities won't start out being habitual. You have to form the habit! And it may take more than 90 days of what feels like challenging work to form certain habits. But it's worth it.

\subsection{Earning trust \emph{while} generating the lead}

Some lead generation activities have the effect of both generating a lead \emph{and} at the same time earning trust with that lead.

These activities have one thing in common: they tend to involve your voice, and the more socially risky the situation, the faster trust can be earned:

\begin{itemize}
\item Conference talk
\item Guest webinar expert on a known, trusted person's webinar
\item Solo webinar
\item Host in-person educational event
\item A live or only lightly-edited media interview
\item Podcast guest appearance
\end{itemize}

\subsection{Earning trust \emph{after} generating the lead}

We usually think of this as ``lead nurture''.

There's a really interesting spectrum of options here, and some of them offer poorly aligned incentives, so I want to dig into this topic a bit.

\subsubsection{The spectrum}

On one end of the spectrum, we have a multi-step process involving some level of automation or personalization, often referred to as a marketing funnel.

On the opposite end of the spectrum, we have more simple, less automated approaches like sending a regular installment of content to your entire email list, or things that function like a radio broadcast in that anyone who ``tunes in'' can partake of the podcast or YouTube video or blog article you've just published.

The poles of the spectrum are not mutually exclusive. You can combine elements from both extremes on this spectrum, or create a sort of hybrid lead nurture system that would live somewhere in the middle of this spectrum. That said, there are tradeoffs inherent in any choice you make about lead nurture, and it's wise to understand those tradeoffs before you make decisions about how to approach lead nurture in your business.

Before we get to that, one more critical concept: ``racking the shotgun''. As far as I can tell, this concept comes from marketer Perry Marshall. He tells a story about how an experienced card shark took his protege into a card room, pulled a sawed-off shotgun out of his coat, and cycled a shell into the chamber. He didn't fire the round, he just ``racked the shotgun''. That sound caused some of the gamblers in the room to immediately look up to see where the familiar but threatening sound was coming from. The card shark told his protege: ``Notice who looked up. That's who you need to worry about; you can hustle everyone else but watch out for \emph{those} guys.'' So that's where this idea of “racking the shotgun'' comes from. It's a signal that not everyone will respond to. In fact, it's calculated such that only the right kind of people will respond to it. People with a certain level of awareness, a certain level of urgency, a certain history, a certain vocabulary, a certain top-of-mind concern, a certain way of seeing the world, members of a certain social group, or some more narrow combination of several of these factors. Here are some hopefully common examples of racking the shotgun in real life:

TODO: examples. Consider including TEI as one of them.

So how do you decide where to land on this spectrum of lead generation approaches? Here are some things to consider as you make this decision:

If you did a pseudo-random sampling of 100 examples of each end of the spectrum, you'd get the sense that highly automated, highly personalized lead nurture tends to be used by a certain type of business that looks a lot like the typical ``internet marketer'' business. And you'd get the sense that simpler, less automated approaches tend to be used by more traditional, more ``corporate'' businesses. And this pattern might lead you to have a negative or positive view of these approaches simply based on \emph{who} tends to use the approach. Do you find internet marketer types personally sleazy? You'll probably also find their methods sleazy as well. Find corporations stodgy and boring? Then you may dismiss the methods they use as ineffective or out of date. If you have no idea what I'm talking about with this whole ``sleazy internet marketer'' thing, do a Google image search for \emph{ internet marketer lambo}, look for images of a guy who fits the ``bro'' stereotype standing next to or sitting in a Lamborghini, and follow a few of those images back to the site they came from and you'll see what I'm talking about. The heuristic of ``do what people I like do, avoid doing what people I dislike do'' is one way to make a decision, but consider that it may not be the \emph{best} way to make \emph{this} decision about how you do lead nurture. Now to be clear, I think plenty of internet marketers \emph{are} actual human sleaze balls. But I avoid dismissing their methods, because most of their methods are in fact neutral. Whether the method is sleazy or not is a function of who is using it, their intentions in using it, and numerous implementation details, not the method itself.

Automation is not the same as intelligence, yet as the one setting up the automation, it can feel like it is. And that feeling can undermine the effectiveness of your automation. Bill Gates said this in the most timeless fashion when he wrote:

\begin{quote}
The first rule of any technology used in a business is that automation applied to an efficient operation will magnify the efficiency. The second is that automation applied to an inefficient operation will magnify the inefficiency.
\end{quote}

This frames the benefit or harm of automation in terms of efficiency, so let's go with that framing and adapt the Gates quote thusly: ``Automation applied to a business that is efficient at earning trust from leads will magnify the efficiency. Automation applied to a business that has not yet figured out how to efficiently earn trust from leads will magnify that inefficiency.'' I often talk about premature automation, which is the idea that you're jumping into the world of marketing automation before you have a well-developed capacity to earn trust from leads without automation. In the premature automation situation, you are very likely to magnify the effect of your own mistaken, un-tested assumptions about your leads rather than scale and magnify your ability to earn trust from your leads faster, or earn trust from more leads.

Incentives: building lead nurture automation is not always easy, but it is always easier than building operational efficiency. That's part of why people who are actually good at building actual operational efficiency at scale are usually paid astronomical amounts of money. They're building the real foundational asset, and those who apply automation to it are successful mainly to the degree that they are building on top of a good foundation. As a solo business owner you may be doing both, but remember that the operational efficiency--or in this case your ability to earn trust efficiently--is the foundation, and automation is the optimization.

I worry a bit about the incentives that modern marketing automation set up for technical folks for several reasons. First, modern marketing automation tools promise an ``easy mode'', and rarely talk about the need to build atop a solid foundation. If you're running a SaaS (and lots of modern marketing automation tools use the SaaS business model), it's easy to focus on CLTV \textgreater{} CAC and let a free trial and a no-hassle cancellation policy substitute for a moral compass for whether someone really needs your software or not. This leads you to build a marketing story that positions your marketing automation product as a magic bullet for lots of things that it \emph{can} do if you're building on a solid foundation but definitely \emph{can not} do if you lack that foundation. It's easy for the SaaS person to say ``well, it's free to try and easy to cancel, so we'll just let customers figure out if it really works for them''. This leads to product marketing with unrealistic claims and insufficient customer screening. But worse, over time it leads to a \emph{belief in the market} that the right tool will substitute for the need to do the dirty, un-sexy work of building a foundation. Really think about this: How many of your beliefs about marketing (remember, this is simply connecting and building trust with prospective clients) come from direct experience, and how many are formed by the marketing and sales copy of \emph{companies with a marketing product to sell and questionable standards for selling it}?

In short, part of the misaligned incentives come from how SaaS companies market their products. The second source of misaligned incentives comes from the fact that modern marketing automation tools can be a fun distraction from the real work of getting good at earning trust and cultivating exceptionally valuable expertise. I speak from experience here. I lost quite a bit of time to the fun but unproductive pursuit of magical automation solutions to a problem that could never be solved with automation: my deficient ability to attract leads and earn their trust efficiently. Technical people (and though I'm not a developer, I was a sysadmin and technical trainer for 10 years, so I definitely have a finely honed appreciation for the pursuit of shiny objects) are usually predisposed to enjoy solving certain kinds of problems merely for the satisfaction of solving the problem rather than the business benefit of solving that problem. In other words, we get an emotional ROI on solving the problem, and that emotional ROI has nothing to do with whether the solution even \emph{has} a business ROI.

For lots of technical folks, the emotional ROI of figuring out marketing automation exceeds the emotional ROI of figuring out how to efficiently earn trust from prospects, and so we dive right into maximizing our emotional ROI and skip past the essential foundational work of building an efficient trust-earning operation. This is a costly mistake, and it comes from misaligned incentives. The answer is to become aware of and honest about what's actually happening (hopefully I've helped with that here), get real about whether you're willing to do the dirty, necessary foundational work, and then apply effort and discipline to build the foundation before you move on to the more emotionally rewarding optimization work.

The highly automated approach also makes it easy to hew towards a solution-first approach rather than a market-first approach. Part of the reason why is that marketing automation solutions are often positioned as very effective ways to persuade or spur leads to take action. If you're coming from a solution-first approach, you'll be tempted by everything that seems able to help you sell your solution, and the less the actual market actually wants it, the more desperate you may become. I've sen lots of my coaching clients struggle and then finally figure out what their market wants. When they do, it's like pouring gasoline on a fire. It's nearly explosive, and while they might add in marketing automation to scale things, marketing automation is never the thing that turns the small fire into a huge roaring fire. It's the fit between what the market urgently needs and what they're offering.

My favorite example of this is Tom, who \emph{sells marketing automation services!}, but he had the patience and discipline to refine his offer to the point of fantastic product-market-fit on Upwork of all places before he moved into applying marketing automation to his own business. Additionally, he spent months doing a grueling daily publication challenge to further deepen his expertise and get feedback from his audience before building on that foundation of insight and expertise and trust-building efficiency with marketing automation. It takes significant discipline to do this, and Tom has definitely reaped the rewards of this discipline and patience. But if you haven't found that fit between what the market urgently needs and what you offer, and you're getting impatient, it's very easy to start seeing premature automation as a solution because that's exactly how lots of marketing automation tools market themselves!

Finally, marketing automation can play into tendencies you might have towards social awkwardness or avoidance of realtime communication. I share in some of these tendencies, so don't worry, I'm not criticizing or making fun of you. But, as I often like to say: unfortunately, you're in a relationship business. It's ``unfortunate'' because many of you would prefer to be in an isolation business, or a maverick genius business, or a leave me alone so I can code business. But unfortunately, relationships play an outsized role in your success, and for many of us, that means we need to work on improving an area that we're naturally weak in. Marketing automation seems to provide a way out of doing this hard work. We think: ``I'll pour my best efforts at relating and connecting into an automated email sequence and \emph{that} can do the hard, uncomfortable work for me! And better yet, it's far more scalable than my own time is!'' And maybe it can. Sometimes. But if you want better results in your business, I urge you to improve two things: your ability to connect with leads, and your ability to earn their trust. Half of your ability to do these two tasks better comes from relationship skills, and the other half comes from cultivating economically valuable expertise.

Wow. Sounds like I'm pretty anti-marketing automation, huh? Nope! I just know who I'm talking to here, and at least 80\% of you need to do more foundational work in the following areas \emph{before} you do much with marketing automation:

\begin{itemize}
\item Choosing a narrow focus so you know exactly \emph{who} you're trying to connect with and earn trust from
\item Gain insight into their world, so you know exactly \emph{what} value you are creating for them
\item Cultivate a \emph{point of view} so that your marketing is meaningfully different from others with a similar skillset or body of experience and so that leads find something attractive or repellant in you
\item Build a \emph{body of work} that demonstrates real, deep expertise rather than commoditized skill
\end{itemize}

Let's look now at the other side of this spectrum (more like broadcasting, less like automation \& personalization) and consider the lead nurture approaches we might find there, which would include the following:

\begin{itemize}
\item Regularly publishing to a blog, email list, or both--with content that's not personalized at all.
\item Publishing a podcast
\item Speaking at an event
\item Publishing a video on YouTube
\item Publishing an article in a niche publication your audience reads
\end{itemize}

These approaches all look like some form of broadcast. If you buy into the promise of segmentation, personalization, and automation, then the idea of broadcasting the same message to a group seems inefficient and outdated. I'd like to make a case for why it might not be.

If you take what I'll simply call the ``broadcast approach'', you'll tend in one of two directions. You'll either play it safe and try to speak to the broadest possible audience (which means you have nobody specific in mind as the beneficiary of your content) or you'll be bold and think more in terms of building a body of work over a significant period of time for \emph{somebody in particular}.

Why only \emph{one} of these two directions? I'm not 100\% sure, but like oil and water, they just don't seem to mix. You're either trying to maximize broad appeal, which means you mistakenly believe you're playing to avoid losing, or you're playing to win, which means you actually understand how things like content marketing actually work. If you're trying to maximize broad appeal, then going very narrow and specific with your lead nurture feels threatening. ``What if a great lead shows up and we completely miss connecting with them in our lead nurture content?'' If you actually understand how content marketing actually works, you see an Internet overflowing with me-too content and so you zig when others are zagging, and you go narrow because you know that will be irresistable to your \emph{ideal} prospects, of whom there are more than enough to support your business.

If you broadcast and go narrow and specific with the focus of your lead nurture content, you create several other beneficial outcomes:

You'll have to make a decision about how best to serve your leads. You'll correctly sense that it's inappropriate to treat them all like they're at the same stage of the buying process, and so you'll be largely relieved of the urge to push them to buy or take some buying action. You'll correctly sense that this kind of pressure would be a real turnoff to those who are earlier in the buying process. Those leads would rightly see you as pushy. And so you'll lay off sales pressure. What will you do instead? As my friend and colleague Liston Witherill so wonderfully puts it--you'll serve, not sell. And as long as you have a reasonably relevant offer, the sales will happen anyway, and they'll happen in direct proportion to the impact that your attitude of service creates. And they'll happen not because you pushed, but because leads approached \emph{you}, which incidentally flips the sales power dynamic very much in your favor.

So if you're not sending super personalized lead nurture content to narrow, specific segments of your audience where you meet them exactly where you think they are in the sales process with content that's fine-tuned for that stage of the buying cycle (how many of you would even know \emph{how} to create that kind of content even if you had the hyper-targeting capability?), then what's left for you to do with your lead nurture content? My answer to that has long been: go deep, be generous, and play the long game. Another way to put this would be: build a brand for yourself as a true expert.

If you look at most experts, the value they have to offer, and the price for accessing that value, it looks a lot like this: \{graph with ``free'' on one end and ``very expensive'' on the other and two tall, narrow bell curves, one centered over the free end, and one centered over the very expensive end, with very little in between\}. The considerable price experts can charge for the application of their expertise affords them enough time to produce a significant body of work for free or very low cost (\$19 books, for example), and they'd often prefer to maximize reach for their ideas, which is why they invest so heavily in free content. In other words, they can \emph{afford to be} generous, and so they are. And they know that there's no limit to how deep that can go with their free lead nurture content because there's always enough prospects out there who will pay a premium price for the specific application of an expert's expertise to their specific situation or needs, so they don’t feel like they have to hold back in the free content, or use it as some kind of cheap tease for their services. Viewed through this lens, an expert who publishes fresh new insightful content to an email list of leads on a weekly or more frequent basis appears a lot more successful than one who sets up some automated lead nurture sequences and then retreats away to focus on their evidently less profitable client work.

\subsubsection{“The Terminators'' vs. everybody else}

Now that I've laid out my case for why you might \emph{avoid} marketing automation, let's talk about why and how you might make use of it :)

If you were in a situation where you could only do broadcast-style publication at relatively high frequency or automated, personalized lead nurture, I'd recommend the broadcast style publication every single time because it has more beneficial side-effects than automated lead nurture. Granted, this may only be true for those who are trying to cultivate deep, valuable expertise, but you're the ones I care most about helping, so my advice is calibrated for you. So again, it's possible you're in a situation where you have to choose between broadcast or automation.

But it's more likely you can combine elements of both approaches. And in that situation, I recommend you do combine them! But in a specific way. Before I lay that out for you, let's talk about ``The Terminators''.

There are leads who--somewhat like The Terminator in James Cameron's 1984 film by the same name--are on a very specific mission to find a solution to their problem. It's like they're walking around with an internal heads up display, scanning and seeking for something very specific. I do believe you can create lead nurture content for these ``Terminators''.

Remember ``racking the shotgun'' from earlier in this chapter? What kind of sound would get the experienced gamblers to look up? What kinds of signals would match would these Terminators are looking for in their heads up display? Instead of looking for Sarah Connor, they're looking for a solution to a specific problem.

TODO: research on the concerns of various stages of the buying cycle and ways of racking the shotgun that could speak to these various concerns.

So group your leads into two camps: The Terminators, who are seeking a solution, and everybody else, who is a potential future client but they currently lack the singleminded focus of your Terminators. You don't have to do this grouping with any kind of actual formal segmentation or marketing automation trickery, you just periodically rack the shotgun in a way that speaks to your Terminators. The others will barely notice as you continue to build your expertise brand with them so that when they become a Terminator (because their problem has moved from not-urgent to urgent, for example), they think of you and your expertise.

So what do you do for The Terminators? I suggest an email autoresponder sequence, sometimes referred to as an ``email course''. One that speaks to the stage in the buying cycle you want to connect with. Here's one example of how I do this:

\begin{itemize}
\item I have an email course I call ``Coder to Consultant''. It speaks to someone who is starting to feel an urgent need around changing their business identify from coder to consultant.
\item I have a page where anybody can opt in to this email course: \href{http://coder2consultant.com}{http://coder2consultant.com}
\item I'll periodically mention this course in the emails I send my entire list of email leads. Those who are feeling this urgent need to shift their business identity hear me ``racking the shotgun'' and they can either visit the ht\href{tp://coder2consultant.com}{tp://coder2consultant.com} page to opt in or click an in-email automation link to be added to that email sequence.
\end{itemize}

So The Terminators can enter an email course in one of three ways:

\begin{itemize}
\item They can stumble across it while visiting your site. This won't happen if it's not available and prominently featured on your site, so make it\ldots{} available and prominently featured on your site. :) The top of your home page is one good place for this kind of thing.
\item They can stumble across you guesting on a podcast, giving a talk for someone else's audience in a webinar or at a conference, running an ad for your course, or elsewhere online. As part of the content you are creating for someone else's audience, you can have what's known as a call to action that directs interested parties to your email course page.
\item They can notice you talking about your email course in your regular lead nurture content. This means they're already on your list, but perhaps your email course hasn't caught their attention before now because they weren't interested or ready to take action until now, or perhaps this is just the first time they've noticed you talking about it.
\end{itemize}

Definitely decide how you want to earn trust after someone becomes a lead (lead nurture) based on what's best for you and your business. I hope you see that there are some real differences between the personalized, automated approach when compared to something that looks more like broadcast publication. I think the latter gets unfairly overlooked because it's more work, but it has some benefits that I hope are now apparent to you. Again, you can combine approaches and I've seen that work very well, and I do that myself in my own business. Overall, just make sure you're choosing your approach based on its actual merits when applied to your business over the long term, not based simply on convenience or the short-term emotional ROI of one approach over the other.

\section{A relevant offer, or ladder of relevant offerings}

When it comes to your service offerings, there's another spectrum at play. On one end is a ``hire us for whatever'' kind of offer: completely custom scope, completely open-ended, and completely custom price (or more likely, a guesstimated estimate and a final price derived from hourly billing with a pretty low accuracy with respect to the original quote). And on the other end of this spectrum is some sort of packaged service offering where the scope and price are both pre-determined, and the delivery process is fairly standardized, systematized, and perhaps optimized for profitability. This is often referred to as a productized service because like many products, its configuration and price are standardized across all customers.

One thing deep expertise can do is help you move from the custom to the productized end of this spectrum. There's nothing inherently superior about a productized service, but there is something about it that signals that you are an expert, and that signal is this: you know what to expect before you get into the engagement, and for this reason you can define the contours of the engagement without depending on the client to do that for you. There's nothing exploratory about it. Your ability to set a fixed price signals confidence in your ability to execute the the work with great consistency and few surprises. Brain surgeon’s don’t ask for your input on where to make the incision and what to do once they get in there for this same reason.

If you prematurely productize a service, the lack of this same expertise can cause low profitability because you set the price too low or it takes too much work to deliver the promised result or surprises slow things down, and it can cause other customer experience problems. Plenty of experts deliver services on a completely custom basis, and this can work out very well. So again, neither custom nor productized are inherently superior. But as you cultivate expertise, I hope you'll set the following as a minimum bar for how you define or package your services:

Most generalists talk about their services based on the skill or discipline that underlies that service. This is another signaling mechanism that reinforces their status as vendors of a service defined by outputs rather than one defined by its impact. They're saying, in essence, ``Here's a bunch of semi-related skills. We don't know the best use for them, so we'll leave that decision up to you, dear client.'' This would be like the brain surgeon having a website where they list their equipment and skills. This is fun to think about, so let’s try an abbreviated example:

\begin{itemize}
\item Equipment:


\begin{itemize}
\item Scalpel
\item Bone saw
\item TODO: more crazy shit like this.
\end{itemize}
\item Skills


\begin{itemize}
\item Can make cuts within 1mm of the intended cut point
\item Can stand on feet for many hours while performing surgery
\item Can remain focused and not check email every 5 minutes
\item TODO: more crazy shit like this
\end{itemize}
\end{itemize}

A completely normal Services page on a generalist firms website would be a grid or list of ``services'' that are actually a list of skills or disciplines. If it's a dev shop, you'll see a list of languages, frameworks, and functional areas of skill. Ex: Ruby, Python, C\#, React, Node.js, DevOps, Testing, AWS infrastructure. Or if it's a digital marketing agency, you'll see a list of the components of a campaign or digital marketing solution. Ex: Facebook ads, copywriting, WordPress design, SEO, content marketing, CRO. This is exactly like an auto repair shop that lists all of its equipment on its website instead of a list of what that equipment can do. Auto repair shop acting like dev shop: Hydraulic lift, impact tools, 300-piece socket set, oil changing equipment, engine computer diagnostic system, tire repair components, etc, etc (you get the idea by now). Actual auto repair shop: We can perform a smog check, repair or replace damaged tires, change your oil, tune up your engine, perform yearly maintenance, etc, etc (you get the idea by now). It's inputs compared to outcomes. Tools/skills compared to results those tools/skills can achieve.

To further think this through, imagine what would happen if an auto repair shop had the exact same set of tools, but in Scenario A they know they are focused on repairing automobiles, and the Scenario B their memory has been tampered with and they forget what kind of machinery they are focused on repairing. They look at their garage full of tools and say, ``Wow, we could repair anything from a chainsaw to cars to an industrial generator to a dragline excavator! Really, anything with an internal combustion engine is fair game for our services. So let's make sure people know that when they visit our website.''  In Scenario A, their website looks like what we'd expect an auto repair shops website to look like. But in Scenario B, might not their website look a lot like the one where they list their tools because they have no idea what their customers will want them to use those tools for? In other words, might not their website look a lot like the typical professional service firms's website where services are really just a list of skills or disciplines?

If your list of services is really just a fancy list of skills or disciplines, then you are sending a signal that you don't know and haven't decided \emph{who or what those skills are for}, and so your website starts to look a lot like another kind of businesses website: an equipment rental company. We fully expect an equipment rental company to have a website with a list of equipment and the day rate for each piece of equipment, because they literally have no idea what you will use that equipment for. The same trench digger might be used to install an irrigation system in your yard \emph{and} later, by another customer, to install a fiber optic cable between two office buildings to enable a large and highly profitable expansion of some sort. The economic value of the latter use is much larger than the former, but the equipment company charges both customers the same day rate for the trench digger. They're not in the business of caring \emph{what you use their tools for}, and if your list of services on your website is only a list of skills or disciplines, then your website is sending the exact same message: ``Here are our tools. Use them for whatever you like.''

While a fully productized service would define what the service is for, the scope and delivery mechanism, and the price, you don't have to go that far to signal expertise. The minimum is this: \emph{define what or who your service is for}.

If you can do at least that, then you'll have made a significant step forward. Define who or what your service is for, and do so in a way that is \emph{specific} and \emph{objective}. Avoid stuff like this, which is a non-specific or totally subjective way of defining who or what your services are for:

\begin{itemize}
\item ``Innovative small businesses'' (How would you measure innovativeness in order to exclude poor fit prospects?)
\item ``Dynamic businesses'' (What does ``dynamic'' even mean? Who gets to define how ``dynamic'' the prospective client is?)
\item ``Fast-growing businesses'' (How will you measure their growth? Where will you set the cutoff? Why are your services only good for fast-growing businesses?)
\end{itemize}

The basic level is to define who or what your services are for, and then, if you can, move on to explain \emph{why} your services are optimized for this kind of business or business problem. This is where it gets harder, because you actually have to construct a coherent argument for why your business focus, experience, and track record align well with the needs of a particular type of business or business problem. If you don't know your ideal clients well, this is both risky and difficult. It gets easier to make this argument after you've been focused in a particular way for several years. So consider this \emph{why} component of your services page to be the advanced level. Again, the basic level of refinement for your service offering is to declare who or what they are for. That's enough to start to differentiate you from the ocean of generalist competition out there with their list of ``tools for rent'' on their services page.

Finally, you may want to consider a ``ladder'' of service offerings. When you construct a ladder of services, you are creating options for your clients. I've heard this referred to as a ``choice of yesses'' rather than a ``yes or no'' decision. There are several good ways, and several bad ways, to organize a ladder of services. The following are generally good ways to do so:

A spectrum of services along a scale of increasing involvement from diagnostic only to diagnostic + complete solution to diagnostic + complete solution + desirable add-ons. The desirable add-ons often take the form of ongoing support, or ``insurance'' of some kind. Notice there is no ``without diagnostic'' option available, for the same reason you can't just waltz into a brain surgeon's office and hop onto the OR table for a brain tumor removal without going through a very thorough diagnostic first.

A spectrum of services along a scale of increasing impact from a quick fix to a deeper or more permanent fix to a deeper fix + desirable add-ons. The desirable add-ons might take the form of skill transfer, ongoing support, or ``insurance''.

A spectrum of services across a variety of delivery models, from done-for-you to done-with-you to teach-you-how-to-do-it-yourself.

And I'd suggest avoiding the following ways of organizing your ladder because they signal a lack of expertise:

A mix of services that include a ``no diagnostic'' option. This signals that you either have no unique diagnostic or strategy value to contribute, or you consider your diagnostic or strategy value interchangeable with that of others. This is like waltzing into that brain surgeon's office, saying ``I'd like you to remove this tumor that my doctor friend Pat diagnosed'', and then the surgeon says ``Great, let's do it'' with absolutely no validation of their own.

A ladder of services with poorly-differentiated ``rungs''. Will prospects have difficulty understanding why they should choose one service over another? That might just be because they're an inexperienced buyer of services like yours (that's not always a bad thing, and with some of the more specialized horizontal service offerings you may fairly regularly encounter this), or it might be because your ladder of services is simply confusing or poorly organized.

You may choose to not list your entire ladder of services on your website. Some parts of the ladder may only be exposed to your clients, and not to the general public. Borrowing terminology from the world of direct response marketing, I often call these ``back-end'' services. They're services that are easier to sell to people who have already paid you and therefore trust you and pose a lower cost of sale than non-clients, or they are services that only make sense to sell to existing or previous clients. For example, imagine that your ladder of services includes design and implementation of a custom CRM for contractors. When you do the design work, you use a proprietary research method to understand your client's CRM needs, and you translate the results of this research into how the software is designed. But, beyond just the software, the data your research collects on your client's sales team is helpful for making other changes to the business process the company uses in their sales department. You might offer a ``back-end'' service where you help your client interpret the findings of your research and make process changes in concert with adopting their new custom CRM. Offering this service on your website might not make sense because it could be hard to help prospects see its value without first doing the research, and you're trying to reduce any extra cognitive load on the website.

Implicit in the idea of a ladder is that clients will climb it. This is where the analogy of a services ladder breaks down, because some ladders are organized such that some clients will enter it only at the highest level and others will, over time, actually climb it as they move from a small, entry level engagement towards the bottom of the ladder to a more robust, deep engagement higher up the ladder. The main point here is that you shouldn't take the ``ladder'' analogy too strictly. It's more about the ``choice of yesses”—or a small menu of focused relevant entrees—than it is about a ``small to big'' arrangement of services.

\section{An ecosystem of support}

Eventually, you'll need an ecosystem of support. Remember that what we're talking about here is what you need to turn your specialization decision into a strong market position. Remember that your market position is your \emph{reputation}. Remember also that I like to say: unfortunately, you're in a relationship business.

One of the ways you can cultivate the reputation you want is to recruit the help of other people:

\begin{itemize}
\item People who can refer you when the right kind of prospective client asks them for a referral.
\item People who can get you in front of their audience and help you connect with the right kind of prospective clients.
\item People who can call attention to your work because it's relevant to their audience.
\item People who have a complementary product or service who can refer you to their customers or clients.
\item People who admire your work and spread the word without you asking them.
\end{itemize}

One of my coaching clients is cultivating a relationship with a Wall Street Journal (WSJ) reporter whose beat (the intersection of tech, business, and culture) overlaps with my client's focus. My client is playing the long game of slow, gradual engagement, starting with social media interaction. Eventually, this WSJ reporter will be part of my client's ecosystem of support. I don't know exactly how or when, but I call this out here because it's a good example of what I mean by an ecosystem of support. You can run your business alone, but you can't reach the highest possible level of success without other people.\{@TK: source\} calls this a “circle of eminence”, which sounds pretentious and dumb but captures the same idea.

Cultivating an ecosystem of support is something you can do intentionally, but it's not something you can develop a granular, precise plan for. It's much more about keeping an ear to the ground for beneficial relationships and opportunity, and then taking action when they present themselves.

Lastly but not leastly, you need something I've mentioned several times already: time and disciplined followthrough. You can specialize with some courage and a quick decision making process, but you can't build a strong market position without time and lots of disciplined followthrough on your part. The time is best measured in years, not months or weeks, and the discipline is about what it takes to complete a PhD degree in a timely fashion.

\section{Chapter Summary}