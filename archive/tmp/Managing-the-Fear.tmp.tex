\chapter{Managing the Fear}

Chapter 3 warned you about The Fear. But, now that you've peered more deeply into the world of specialization and what your options might be there, it's a good time to talk about practical ways to manage The Fear.

The first fear management approach is to \textbf{lower the stakes}.

Lowering the stakes might look like specializing a digital marketing funnel as a low-stakes way to experiment with various possible specializations before committing more fully. Or alternately, build a market-first (rather than solution-first) productized service as a way to test demand for a specialized service that could eventually ``take over'' your business and enable a generalist to specialist transition with less existential fear. Both of these approaches fit under the superficial specialization umbrella discussed in Chapter 7. In the right situation, these can be useful approaches.

My main caveat here is to properly align your expectations. Lowering the stakes might be the exact right decision because a low-stakes experiment is all your risk profile and personality can handle (see Chapter 12 for more on this). But lowering the stakes might also reduce the potential upside for you! Or you might half-ass it because there's not much on the line, and as a result get half-assed results. So lowering the stakes is not a panacea, and it's not the only right approach.

The second fear management approach is to \textbf{gather more evidence}

Specializing can feel like making a decision that has high stakes (significant potential consequences). When you make a decision like that without a lot of information, it's naturally a bit intimidating. You find yourself saying things like ``Am I really just guessing here?'' or ``How can I be sure this will work out well for me?''.

The 100\% honest answer is that you can't be 100\% sure it will work out well for you. I can't be sure I won't die driving to the grocery store later today, and according to the Insurance Information Institute, that terrible possibility is far \emph{less likely} to happen than accidentally poisoning myself with a prescription or illegal drug! So again, we can't be 100\% sure of much in this life.

But what you can do is gather evidence. You can look ``look before you leap''. You can identify patterns that inform your specialization decision.

Chapter 12 goes way deeper into this, but for now just know that the best antidote to uncertainty is the right kind of data interpreted in the right kind of way. I refer to this as evidence, and Chapter 12 suggest what kinds you'll need, and how to gather it.

``I'm \textbf{mad as hell}, and I'm not going to take this anymore!''

I've seen some folks use the emotional energy of anger to fuel a transition from generalist to specialist. I think this is part of what helped me make the leap in my own career.

Thanks to our mammalian fight-or-flight wiring, anger can temporarily override our self-preservation characteristics. Now imagine that you're not feeling indiscriminate anger, but a more focused, ``righteous'' anger at how operating as a generalist has become harmful to your business. This kind of focused, righteous anger could be a very effective antidote to The Fear, and a very effective catalyst for bold decision-making as you decide what specialization opportunity to pursue. This isn't something you'd do on purpose, but if you happen to be feeling angry about how things are going as a generalist, that anger could fuel a (still hopefully carefully considered) move to a specialist market position.

\textbf{Courage and risk tolerance} are the best antidote.

The two fear management approaches described above come with compromises. This one doesn't: Do something courageous! Take a calculated risk! If you can do this with the question of how you focus your business, you can do it with lots of other aspects of your business, and I can't think of a real downside to acquiring this new ability to take calculated risks.

I'm not recommending wild or indiscriminate risk-taking. I'm simply recommending you do your homework on how best to specialize (again, Chapter 12 has more on this), gather enough evidence to validate your decision, and then take assertive action on that decision.

If you're not currently in the position to make this kind of change to your business, of course that's OK. Your focus instead should be taking smaller risks and smaller forms of assertive action, perhaps around increasing rates or profitability.

\section{Suggestions for cultivating courage}

Specialization is not a face tattoo. I mean to be both serious and humorous here:

To contextualize how serious a decision specialization is, let me sketch out a spectrum of levels of commitment  to change in order from low to high:

\begin{itemize}
\item The clothes you put on first thing in the morning. How hard is it to change these?
\item Buying a television using Amazon.com. This can be relatively low cost on the front end and low switching cost if you decide to get a different one, upgrade, or get rid of it within the no-questions-asked return period.
\item Buying a car. More costly--both in terms of up-front cost and switching cost--than buying a TV, but still a decision that can be changed and un-made. But making the change does have a bit of friction because you need to go to the dealer in the first place, and to change either go back to the dealer for a trade-in or sell it yourself, both of which entail real effort.
\item Buying a house. Take the car example and basically 10x the level of work and potential switching cost.
\item Getting legally married. You can get un-married, but not without some level of paperwork that--even in Oregon where I got divorced--is more extensive than what you signed to get married in the first place. And that's not counting the emotional and financial cost of a separation and all that might happen during and after the separation/divorce.
\item Getting a tattoo on your face. Yes, these can be removed, but that's painful, and the affected skin is never the same as it was before the tattoo. It really is, in some sense, truly permanent. It's the only thing on this list that \emph{is} permanent.
\end{itemize}

So my first suggestion for cultivating courage is to realize where specialization actually fits in this spectrum from easy-to-change to impossible-to-change. Depending on several things (how complex your marketing infrastructure is, how big or complex your business is), I'd say specialization is somewhere between buying a TV and buying a house in terms of how difficult it is to change. If you're evaluating in terms of permanence, you should put a \emph{lot} more thought into getting a face tattoo than any of the other decisions on this list. :)

Now, I need to remind you of the difference between specialization and your market position. Specialization is the front-end decision; your market position is the reputation you build up by specializing and following through on it for years. So a market position is much harder to change than a specialization decision. It's easier for a services firm to change their market position than it would be for Bill Clinton to change his reputation as a womanizer, but it's still a fair bit of work for the services firm to change how the market perceives them. But the specialization decision? That's dramatically easier to change.

The takeaway here is that \emph{you have freedom to experiment with your specialization}, especially earlier in the process. I hope knowing this helps you make a more courageous decision. It's easier to be courageous when you realize you're making a decision that can be changed or un-made.

\section{See how specialization works for others}

I do a lot of labor-of-love work in this area to help show how specialization actually works:

\begin{itemize}
\item http://specializationexamples.com
\item http://consultingpipelinepodcast.com
\item http://theselfmadeexpert.com
\end{itemize}

You might also find someone in your network who actually has specialized and talk to them! There's at least a 90\% chance they'll be flattered you asked!

Ask them the following:

\begin{itemize}
\item Did it work?
\item What made it work or not work?
\item What challenges did specialization present?
\item What benefits did it create?
\item What advice do you have for others wanting to specialize? What might you do differently next time?
\end{itemize}

\section{Run the numbers}

Again, nothing is 100\% guaranteed, but here's what I've seen regarding the financial upside of specialization.

I can think of multiple examples of generalists specializing and working less while generating the same amount of revenue, or working the same amount and generating more revenue. Connie--a self-employed mother of two--cut her workload in half and maintained the same revenue (100\% increase in profitability). Val--another self-employed mom--maintained her workload and tripled her revenue (300\% increase in profitability).

Professional services firms that squeeze every drop of benefit out of specialization and all the downstream advantages thereof (ability to value price, ability to sell advice rather than or in addition to implementation) can break into the vaunted territory of \$400,000 in revenue per full time equivalent (FTE). I'm not saying this is common, and I'm certainly not promising this for you, but it is possible.

So run the numbers for yourself. What would a 100\% to 300\% increase in profitability look like for \emph{your} business? Get specific about what this would mean for you. Maybe it's one of these:

\begin{itemize}
\item Finally able to buy a house
\item Extend your runway from 3 months to 12 months, lower your stress level
\item Feel less pressure to bill 40 hours/week
\item Ability to invest in your highest-ROI investment, which is yourself
\end{itemize}

What you're doing here is \emph{building a vision} for what your future could look like. That's why it's important to be specific. When you daydream about something desirable, you're specific. This is just ``daydreaming on purpose'', and it can help you move through fear and indecision.

\section{Estimate the cost of \emph{not} specializing}

Avoiding change seems like a way to spare yourself the cost of changing, but avoiding change also has a cost of its own.

This is because the world keeps on changing, and if you're running a business with a skills-based model, you're more exposed to the downside of change than an expertise-based business model is.

You don't get to decide how long skill in Objective C can command super-premium rates. The predictable market forces, other service providers in the market, and Apple determine that, and they do so without much regard at all for how their decisions and actions effect your hourly rate. So Swift or Reactive Native partially replace Objective C, and while from a technical perspective Objective C still be used to build great apps, things have changed in a way that de-values Objective C skill. The inevitable hype around Swift or Reactive Native does some of the work of devaluing Objective C skill. Commoditization--wherein attractive Objective C rates attract competent competition from lower cost of living areas--does the rest.

Without barriers to entry, licensing or formal certification requirements, or a professional standards body, super-premium rates for skill along don't last very long in the world of technology. You can pretty safely bet on a 5 to 7 year lifespan for super-premium rates.

When you're young, 5 to 7 years feels like a really long time, and making \$200 hour feels like being wealthy. When you're past about age 40, 5 to 7 years feels like much less time, and \$200/hr feels like a bare minimum for access to your maturity, judgement, and ability to avoid stupid decisions (largely because you realize that making \$200/hr is not the same thing at all as being wealthy).

So, even if you don't specialize, you'll still need to change with the changing times in order to remain competitive as a software developer. And I bet if you really think through the cost of upgrading your skillset at least every 3 to 5 years, you'll see that it's a significant time and emotional cost, if not an actual financial cost.

Compare this cost to whatever you perceive the cost of specializing to be. It might level the playing field, or even tilt it in favor of specializing.

To be clear, I am trying here to make a strong case for specializing and support you the best way a book possibly could, but I'm not trying to ``twist your arm'' and force you to do it. It's a little bit like getting married or being in a long-term romantic relationship. If you don't fundamentally want to do it, it won't work. Romantic partners and the market can both smell someone who is pretending to want them, and neither of them ultimately react well to that realization.

\section{Chapter Summary}