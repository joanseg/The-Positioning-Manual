\chapter{Why Position Your Company?}

TODO: revision notes

\begin{itemize}
\item Perhaps reference DCB’s hourly rate research. Mention profitability.
\item Pre-req to value pricing
\item Better clients
\item Better business
\item happier you
\end{itemize}

This manual is over 23,000 words on how to develop a desirable market position for your business. But the question of \emph{why} to do this is easily answered in just 4 words:

\textbf{Better clients, higher fees}

If you're used to thinking in terms of hourly, daily, or weekly rates instead of per-project fees, that's fine--the benefits of a narrow market position still apply to you. Better clients, higher \emph{rates}.

\section{But Wait, There's More!}

Those are just the headline benefits of a narrow market position. Even more benefits appear when you really look closely. To thoroughly understand the benefits, let's take a step back and assess the situation you're probably in if your business does not focus on a narrow market position.

\subsection{1) \textbf{You started working for yourself with little preparation}.}

You didn't study the world of business for months or years before making the leap to self-employment. Instead, the move to self-employment was more\ldots{} improvised. An unexpected layoff, a shitty boss you were fed up with working for, or an unexpected offer of contract work might have been what caused you to make the leap to self employment.

This was \emph{exactly} how I became self-employed. I had a great boss but got laid off in 2008 (along with everybody else at my employer), coasted on unemployment for a few months, and then said, ``Screw it, I can be self-employed!''. I then proceeded to spend 5 years beating my head against the brick wall of being an undifferentiated generalist while constantly facing the prospect of financial ruin and a shameful retreat to the ``safety'' of a FTE job.

\subsection{2) \textbf{Business development is either an afterthought, a struggle, or both}.}

If you had to sum up your business development approach in one word, it would be \textbf{luck}, or maybe \textbf{hustle}. Maybe you hope for a lucky break while refilling your drink at those depressing networking events where 75\% of the crowd is unemployed freelancers, and the other 25\% is big agencies scouting for cheap talent to put into the next project grinder.

Maybe you depend a little too much on referrals from past clients. And maybe your outreach to prospects is based on a generalist value proposition and a lot of hustle, as in cold email blasts saying something like\ldots{}

\begin{quote}
We can handle all of your development/design/marketing needs. We're a \emph{full service} agency!
\end{quote}

Finally, you are \emph{almost never} sought out because your services solve a very specific, expensive problem for a client. Instead, you win jobs because you are affordable, ``great to work with”, or seem more ``together'' than the alternatives.

\subsection{3) \textbf{You operate from a position of weakness in rate/fee negotiations}.}

Your rate or fee is dependent on how badly you need or want whatever work is on the negotiating table, and you offer discounts (even when they're not asked for) based on your eagerness to win the work or the level of anxiety you are feeling around your pipeline. You offer payment terms that are extremely generous (more than 7 days net) to every client, after the contract is signed you are shuffled off to the client's finance/accounting department where you are treated like a low-level vendor rather than a valued business partner, and when payment is late you automatically assume the position of a supplicant and make tense, carefully worded requests for payments to low-level bureaucrats within your client's finance/accounting department.

\subsection{4) \textbf{Your business feels like it is treading water}.}

Perhaps you are growing very slowly, or perhaps your growth has already plateaued and is limited to 10\% or 20\% a year ``cost of living'' rate increases that are met with grumbling acquiescence from clients. Your business growth strategy is  ``more of the same'' or\ldots{} hope, and you take little time to focus on finding new ways of creating value for clients.

\subsection{5) \textbf{You almost never say no to work}.}

Every project that does come your way is met with an enthusiastic ``Sure, we can do that!'' and a mad scramble to put together a creative, persuasive proposal as quickly as possible. After you pitch or email in your proposal, you find yourself on pins and needles while you wait to hear the results.

You really have no idea whether you will win the work, and you check in with your prospective client periodically to diplomatically ask how the decision process is going.

If any of this rings true for you, I want to reiterate that I was in this \emph{exact situation} for 5 years. Every painful thing I've listed above is something I've experienced firsthand, many times over. I've been there, it sucks, and there is a way out.

If you're \emph{not} in this situation because you've already focused your business on a great market position or have found some other way to charge premium rates and develop a strong pipeline of interesting, challenging work, \emph{go buy yourself an ice cream right now!} You've earned it, and you can skip reading the rest of this manual.

However, if any of the above rings true for you, then let me tell you how becoming a leader in a narrow market position will improve your condition.

I suppose this is the right place to say you \emph{could} position your business as an undifferentiated generalist. This would technically still be a form of positioning. However, it would solve no business problems at all for you, so in this manual I'm only going to talk about narrowing your focus in a way that positions you as a leader. In fact, any time I refer to positioning in this manual, that's what I mean: becoming a leader in a narrow market position. There really is no other way to position a professional services firm for maximum success.

\section{The Benefits of a Narrow Market Position}

In addition to giving you access to better clients and higher fees, becoming a leader in a narrow market position will make each one of the items I called out above \emph{much, much better}! Let's look at each one in turn:

\subsection{1) \textbf{You started working for yourself with little preparation}.}

Positioning is not a time machine that can go back and fix your past unpreparedness. But, when you position your business correctly, you finally address the root cause of \emph{so, so many} business problems. You \textbf{finally decide} exactly who you serve, what you do for them, and how you are different from others doing similar work. Making this choice is a pre-requisite for solving other business problems like fee-setting, billing, and client relationship problems. Making this choice is the \textbf{foundation} for building a robust professional services business.

We can learn a lot from product businesses. So much of the strength of a good product business comes from the product itself. How well does it address a real need, how well-designed and marketed is it, how efficiently can it be produced, distributed, and maintained?

Without a well-designed product that meets a real need or desire in the market, you don't have a product business--you have a mess. That's patently obvious, yet in the world of professional services, many firms operate with no equivalent to a product. Their services are so ill-defined and broadly scoped that potential buyers have no idea what need or desire those services match up with.

Focus is just as important to a professional services firm as it is to a product business.

\subsection{2) \textbf{Business development is either an afterthought, a struggle, or both}.}

The linchpin of a business development strategy is positioning. If you are, for example, a ``Rails shop'' or a ``front-end developer'' or a ``CTO for hire'', you are in direct competition with every other human being who is using the same label for their business (fn1). When your target is that broad, that's a \emph{lot} of competition, as I'm sure you know from direct experience. Furthermore, you have no ``unfair advantage'' over that competition, and you depend fundamentally on luck or an extraordinarily helpful professional network to bring you in contact with clients who are a great fit for you.

Yet, if you focus on a narrow market position, the game changes \emph{completely}. If, instead of a ``Rails shop'' you are a company that ``builds modern reporting systems on top of legacy systems data to reduce damaged freight insurance claims for large privately held freight companies by up to 40\%'' (fn2), your approach to business development becomes dramatically more efficient \emph{and} effective. Here’s why:

\begin{itemize}
\item You can very quickly have a list of prospective clients. Not just a list of \emph{types} of companies that could be your clients but a list of \emph{specific} companies with names of \emph{specific} people you can market to, meet with, and get signed checks from. You intersect a list of large freight companies with a list of users of legacy systems and \emph{boom}, you have a prospect list that--if you have chosen a niche of the right size—will keep your company fed, growing, and winning almost every project you want for a long time. Getting this information is not complicated, and once you have it, you have a huge part of your game plan for business development locked in.
\item Your value proposition to that list of prospects stands out in a crowd like a red Ferrari in a parking lot full of gray station wagons. When those prospects become aware that it's possible to save millions or tens of millions of dollars a year in insurance claims \emph{and} that there's a company with a track record of reliably creating that \emph{exact outcome}, your business development becomes like ``selling water to a thirsty person in a desert'' rather than ``selling ice to an Eskimo''.
\end{itemize}

\subsection{3) \textbf{You operate from a position of weakness in rate/fee negotiations}.}

To return to the previous example in \#2 above, if you have chosen a market position that sets you up to solve a very specific, expensive problem for clients like the software solution that saves millions in damaged freight insurance claims, then the power dynamic shifts in your favor. Remember, ``expensive problems'' may be ones where solving the problem produces an impressive return on investment (ROI), or a significant cost savings.(fn3)

A solver of expensive problems becomes a partner rather than a vendor, and partners are entitled to an equitable share of the results of the partnership. Partners get to speak up for themselves when they don't like the terms of the agreement, while vendors have to ``go along to get along'' or risk being replaced. If the options for replacing you are very limited, you naturally have much more power in the relationship.

\subsection{4) \textbf{Your business feels like it is treading water}.}

If you are a generalist doing some kind of technical work like programming or web development, you are likely able to charge \$75 to \$150 USD an hour. If you are billing in that range, how much professional development (within business hours) can you allocate the time and funds for? How much time can you spend developing new services, products, or entire lines of business?

Your ability to grow your business in innovative new ways is limited if you are spending 75\% or more of your time doing billable work. Operating as a focused specialist gives you access to dramatically higher rates (often \$200/hr and above) or the ability to apply value-based fees to projects because you are solving an expensive problem. This means that you can spend a greater portion of your time moving your business out of water-treading mode.

\subsection{5) \textbf{You almost never say no to work}.}

This is one of the benefits of being a focused specialist that you just have to experience to fully appreciate. I really can't tell you how much emotional energy I burned up because, as a generalist, I was in a financial position where I really needed to say ``yes'' to every project that came my way.

Later on in the projects that became nightmares, I would ask myself ``honestly Philip, did you see this coming before the contract was signed? Was there some sign that was right there in front of your face?'' The answer was \emph{always} yes. The nightmare client had \emph{always} given some sign of what was to come: balking at my rates, communicating with me only at their convenience, or some other red flag that later made the project very painful. Or in other cases, I had taken on a project that was at or beyond the limits of my expertise, making the project painful for a different set of reasons.

A focused specialist can say no to a good portion of the clients that come their way if necessary. Quality of work life goes up dramatically as a result of the ability to throughly vet prospects based on what really will produce a good fit for both client and consultant.

\section{Conclusion}

In conclusion, remember that finding a narrow market position is the most efficient way to deal with the common issues that lead to a “feast/famine” cycle in your business.

\section{Footnotes}

Of course, this may be a smaller pool of competitors too in cases where a client wants to work only with a local company, etc. But more and more, especially in technology services, clients are sourcing from a global pool of talent.

This is a hyper-specific example, mostly to make a point. You do not have to be this specific in your messaging, but you must be quite a bit more specific than you probably have been thus far.

Incidentally, properly positioning your business produces both a strong ROI and a cost savings to you as well. The ROI comes from your future ability to attract better clients and charge higher fees, and the savings from having fewer clients that are a poor match for your business.