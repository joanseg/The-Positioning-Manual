\chapter{It's Gonna Hurt}

\section{Why is Positioning so Difficult?}

I wouldn't be writing this manual if properly positioning a privately owned technical services firm was easy. And you wouldn't be reading either if it was easy.

So why is it not easy?

\subsection{It Brings Up The Fear}

Know that \emph{you will face some gnarly fears} as you move through the work of narrowing your business focus so that you can provide more value to your clients. Moving from operating as a generalist, ``A to Z'', or ``full service'' firm to operating as a focused specialist will almost certainly cause you to fear that:

\begin{itemize}
\item You have chosen the wrong thing to focus on
\item You are not worthy of commanding premium rates for your work
\item You are cutting off access to desirable, profitable work
\item You will quickly become bored with your choice
\end{itemize}

Collectively, I call these fears \emph{The Fear}.

\subsection{It Challenges Your Sense of Identity}

Properly positioning your professional services business is also difficult because it is \emph{so personal}. A lot of us start our self-employment under a personal brand. We don't really think of ourselves the way that a manufacturer of backpacks or wallets or speed controllers for electric motors would think of their business. They would almost certainly have a non-personal business identity from day 1.

When we in the professional services business try to develop a narrow market position, it's so wrapped up in our \emph{identity} of who we are, even if we have a non-personal business name. We are creative people, and we love the thrill of exploration and new learning curves. Even more, we often take real pride in how our personality attributes are woven into our businesses.

It comes with the territory of being creative that we \emph{don't} like to think of ourselves as narrow specialists. In fact, we jump to unfairly characterizing narrow specialists as ``one-dimensional'', ``wonks'', and even ``boring'', while we neglect the incredible potential for exploring the intricate \emph{depth} of a specialty.

\subsection{It's Mysterious}

Using positioning to your advantage is also \emph{mysterious}. For beginner business owners, it's not 100\% apparent that dramatically narrowing your focus is one of the keys to opening up access to higher fees and better clients\footnote{Proper pricing and effective marketing are the other two keys.}. We think that the path to those forms of financial and career success lies in expanding our skill set to the bleeding edge, networking, simply demanding higher rates or fees, a fancy new web site, or some other less difficult way.

\section{Positioning is Vital}

If you want to gain access to higher fees and better clients, there are many things you can try. You can do things like the aforementioned expanded skill set, networking, demanding higher rates or fees, or a fancy new web site. There are four other approaches that largely bypass the need to narrow down your focus:

\begin{enumerate}
\item \textbf{Luck}: I lucked into \$150k worth of work when I started working for myself. Luck giveth… and later luck took that work away with no prior notice, so beware relying too much on luck.
\item \textbf{Connections}: Knowing the right people can be a temporary substitute for becoming mature and disciplined in your business development.
\item \textbf{Leverage an offshore team}: Taking advantage of the rate differential between what developed country clients are willing to pay and rates charged by talent from low cost of living areas can be a solid way to run a business without narrowing down your focus. You may miss out on the ability to be selective about your clients, but you’ll have a reasonable business anyway.
\item \textbf{Ride a platform that is crossing the chasm to mainstream status}: From about 2008 to the time of this writing in mid-2016 (and perhaps slightly beyond), the only business development a good iOS developer needed to do was to keep their LinkedIn profile up to date. A platform exploding in popularity can be a temporary substitute for a narrow market position.
\end{enumerate}

If any of those methods work for you:

\begin{enumerate}
\item \textbf{Congratulations}--seriously--congrats on the quick win!
\item \emph{Be careful}, because you have likely come across a non-repeatable one-off win that will not result in sustained growth for your business.
\end{enumerate}

There is only one repeatable path to premium rates and better clients that is under your control, and that path involves narrowing your focus in order to increase the value you deliver to clients. This path sets you up for sustainable growth, and solves a myriad of problems in your business.

\section{You Will Face The Fear}

If you specialize your marketing, you will for sure face The Fear. Because of The Fear, I find that many people tend to retreat into generalizations and excuses to not position their business as a specialist.

I can identify. I resisted specializing for the first 5 years I worked for myself. At first, it was because being a generalist seemed more interesting and exciting, and I didn't understand the benefits of specialization. Then, it was because of The Fear. Eventually, I transitioned away from being a generalist writer for anybody with a pulse and a checkbook to my current specialized focus. But I lived on the brink of financial ruin for those 5 stressful years before specializing. Anything that made an already-precarious situation seem more risky was not a welcome change. So I get it\ldots{} narrowing your focus is a challenging process!

Here were my fears about positioning myself as a specialist. I'd bet money they are the same ones you face when you consider specializing:

\begin{quotation}
What if I pick the wrong niche?

What if I am unknowingly committing career suicide?

What if I can't deliver on my claims of expertise?

What if I pick a \textbf{really boring} niche and before I know it I'm crying myself to sleep every night on \$900 silk sheets while my soul slowly dies?
\end{quotation}

The good news is threefold:

\begin{enumerate}
\item Almost every person who has positioned themselves as a specialist has good things to say about it.
\item The Fear is pretty common to everyone who makes this transition. By knowing what to expect, the entire process will be a business development power play on your part rather than an exercise in battling your own demons.
\item There are ways to reduce the risk of what feels like a \emph{very risky} transition. You do not have to bet the farm on a new, untested positioning.
\end{enumerate}

\section{The Fear}

As you develop a new market position, be prepared to experience these four common fears.

\subsection{Loss Aversion}

You will at some point feel like you are excluding desirable clients by narrowing your marketing and by saying no to clients that you could potentially help but fall outside your defined focus.

The truth is that you \emph{are} doing this, but you are \emph{also} opening up access to a more \emph{profitable, desirable, deeper niche market} that has greater earning potential for you. You cannot gain access to that niche market without becoming focused and demonstrating your expertise through your marketing.

\subsection{Imposter Syndrome}

Unless you are an unusually self-confident person, you will question your expertise and your worthiness to charge \$200/hr plus\footnote{That's just an example of a rate that's well beyond generalist territory. Of course, your mileage may vary.}. This goes away after a while, but I believe this particular fear does cause some people to shrink back from the path of specialization, because deep down they know that they must believe in their own value, and they really don't at first.

The good news is that with focus comes confidence. If you are a generalist, you are \emph{always} dealing with learning curve and unexpected issues. When you specialize, you quickly learn how to deal with these common issues and your confidence increases \emph{very quickly} as a result.

This confidence will give you the strength you need to tackle bigger challenges within your area of focus. Tackling those bigger challenges will help you believe in your own value. This self-reinforcing feedback loop will quickly put an end to your imposter syndrome.

\subsection{Boredom}

It's common to fear that if you specialize, you will become bored with your work very quickly because you are narrowing the scope of what you do. In other words, you are solving a narrower range of problems for clients. You may fear that ``doing the same thing over and over again'' will get boring.

The fear of boredom is a false fear, rooted in an unrealistic picture of what it is really like to specialize. To outsiders, the world of specialization is perceived like that gray, drab world depicted in Apple's famous \href{http://bit.ly/1xEXr8B}{``1984'' TV commercial}. Those who inhabit it are perceived as robotic geeks who lack the interpersonal skills needed to communicate with others, or who are so consumed with their own enthusiasm for a narrow topic that they simple can't relate to others.

To get a more realistic picture of specialization ask yourself: ``who gives the most interesting \href{http://ted.com}{TED talks}? Generalists or specialists?'' In fact, who gives \emph{any kind of TED talk}?

\textbf{Specialists get invited to speak at TED}.

Specialists get to solve interesting problems \emph{all the time}. Solving interesting problems is not doing the same thing over and over again, it's drawing close and going deep into a subject. Specialization may from time to time involve tedious or unpleasant work to be sure, but who can better afford to hire help with those tasks? A \$90/hr generalist, or a \$350/hr specialist?

\subsection{Fear of Shrinking Brain Syndrome}

Finally, I have to mention that for almost any person working in the technical end of professional services (software development, IT, etc.), much of our identity is wrapped up in \emph{how much we know}. I'm no different--it's a point of pride for me to understand a wide range of subjects at some depth. I want to be perceived as being intelligent and current on a range of issues.

I think we technical folks fear specialization in part because we won't want to narrow down our knowledge domain, and we don't want to give up our identity as people who can speak intelligently to and actually solve a wide range of problems. Ultimately, we don't want to have to say, ``I can't help you with that.''

Again, this particular fear is a symptom of misunderstanding the pleasures of going deep with a specific subject. Furthermore, your business positioning may have little to do with the subjects you explore in your down time. And again I'll ask, who can afford more down time to explore matters of personal interest--a \$90/hr generalist or a \$350/hr specialist? Yes, you may have to adjust your expectations around getting paid to learn on the job--which is one of the ways the \$90/hr generalist gets to tinker with so many interesting toys ``on the clock''--but I trust you can see the benefits of making that shift.

\subsection{Picking the Wrong Niche}

The final fear is probably the most insidious, because it is psychologically complex. It involves both \href{https://en.wikipedia.org/wiki/Loss_aversion}{loss aversion} and \href{http://en.wikipedia.org/wiki/Analysis_paralysis}{analysis paralysis}, and the two seem to play off each other.

When you move from ``A to Z software development'' to a specialized subset of that work for a well-defined market vertical, it \emph{feels} like you are facing a \emph{tremendous loss} of potential work. You wonder if your chosen niche is large enough to support your continued growth. And then when you try to pick one of the \emph{many} things you could do to create value for clients, you get overwhelmed at all the choices!

If you are used to a ``take all comers'' approach to business development, making a choice and putting a stake in the ground feels incredibly risky. You are used to perceiving yourself as ``following the money'', and you can fool yourself into thinking that it's ``smart'' to be ``flexible'' and ``accommodating'' of the variety of clients that come your way. This mindset will tell you that choosing a single type of client to work with is like rolling the dice or playing Russian Roulette: very, very risky.

While I can't completely de-risk this choice for you, I will provide as many tools as I can in this manual to help you make a prudent choice about which market position to pursue.

\section{Let's Do This Together}

The rest of this book is all about helping you move through these fears and make the best positioning choice you can. There is a reason why Blair Enns calls positioning ``\href{http://www.winwithoutpitching.com/read-it-online/the-purpose-of-positioning/}{the Difficult Business Decision}''! It takes courage and commitment to push through the fear.

\section{Conclusion}

In conclusion, remember that the biggest obstacle to finding a desirable market position is not a crowded market place or strong competition. It’s the fears that naturally arise during the process.