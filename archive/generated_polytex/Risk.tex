\chapter{Risk}

You probably mis-perceive the risk levels of various activities. Let's look at one manifestation of risk: accidental death. Specifically, the lifetime odds of dying from various causes.

Which one seems like a more likely cause of death for most people? Unintentional drug poisoning or car crash? How about dying in a fire (literally) vs. dying in a plane crash? I'd bet you guessed wrong on both of those. I would have too if I didn't have these actual numbers from the Insurance Information Institute:

\begin{itemize}
\item \textbf{Accidental drug poisoning}: 1 in 70 lifetime odds of dying this way
\item \textbf{Motor vehicle accident}: 1 in 102 lifetime odds of dying this way
\item \textbf{Exposure to smoke, fire and flames}: 1 in 1,506 lifetime odds of dying this way
\item \textbf{Air and space transport accidents}: 1 in 10,101 lifetime odds of dying this way
\end{itemize}

--Source: \href{https://www.iii.org/fact-statistic/facts-statistics-mortality-risk}{https://www.iii.org/fact-statistic/facts-statistics-mortality-risk}

So, you're more likely to die of accidental drug poisoning than in a car crash, and more likely to die in a fire than in a plane crash! I would have guessed the oppose on each one of those pairs. To me, this points to one simple takeaway: we tend to misunderstand risk! This is because we over-weight factors that aren't correlated with risk (car crashes \emph{look} gruesome while drugs look like clean- small pills or pure liquids), under-weight factors that are, and suffer from all sorts of cognitive biases that interfere with our ability to accurately assess risk (the availability heuristic is one biggie). Let's try to clarify the topic of risk with some important supporting ideas and a self-administered risk assessment that's a simplified version of what I use with my clients.

I am not a psychologist. I don't even play one on TV. That said, it's useful to think about two ideas that attempt to generalize a basket of personality characteristics and motivations into ``types''. As with all generalizations, 100\% fidelity to individual cases is impossible, so take this with a grain of salt.

\section{Missionary - Mercenary spectrum}

One useful idea is that we all fall somewhere on a spectrum with ``missionaries'' on one end and ``mercenaries'' on the other.

Missionaries are motivated by a sense of purpose that's not fundamentally money-making in nature. Missionaries certainly may find a way to get paid--and perhaps quite well--for pursing their purpose, but that's not what drives them. It might be a desire to change the culture (or some small aspect thereof), it might be a desire to improve things in some specific or far-reaching way, or it might be a desire to build something new they just feel driven to build.

This raises the question: if you're a missionary, do you invent your mission or does it find you? What I've seen is that they're both common, and neither is inherently superior or more effective.

Mercenaries are in it for the money. This blunt description might seem like a dismissal but it's not. It's just that mercenaries are simple to describe. :)

Now obviously, even if human behavior was simple enough to describe with this missionary-mercenary generalization (it's not), we'd still find elements of both types in every person. Even missionaries are going to act like mercenaries at times when they need the money more than they need the satisfaction of living out their mission. And even mercenaries will seek the meaning and satisfaction of some kind of mission, even if that mission is chosen and optimized to maximize its potential monetary upside.

That said, when I ask coaching clients to place themselves somewhere on this missionary-mercenary spectrum, almost everybody has an easy time doing so. So there's \emph{something} true and resonant about how this idea generalizes our motivations into a simple duality.

How do these two types respond to risk? I haven't found any significant difference in how they respond to risk, \emph{however} I believe there is a difference in how they respond to \emph{complex, long-term projects where there are many opportunities to make mistakes, recover from mistakes, or not recover from mistakes}. I believe the missionaries do better with these kinds of projects. In other words, I believe missionaries do better with risky, lengthy projects because they have something other than financial payoff motivating them. Missionaries tend to have an ability to envision a medium or long-term future where their mission has created significant value, and the vision of that future value keeps them motivated through the lengthy journey it may take to get there. It would be tempting to say that mercenaries should avoid these kinds of projects, but that would be exactly the kind of over-generalization I want to avoid.

\section{Entrepreneur - Craftspeople spectrum}

There's another way to generalize human motivation that's a bit more specific to creative professionals (which includes developers). It's another over-simplified spectrum.

On one end are the entrepreneurs. There are lots of definitions of entrepreneur, but for services businesses I prefer to think of an entrepreneur as one who decouples their income from the time they spend working on client deliverables. You can do this in a variety of ways, not all of which require building a team where you leverage other people's time. There are plenty of entrepreneurial businesses with one person doing all the value creation and delivery, but in a way where much of that value creation is de-coupled from their time.

On the other end of this spectrum are the craftspeople. They love their craft above almost everything else. They might be heard saying ``if only I could spend all my time on my craft\ldots{}''. Or alternately, when asked what's special about their business, their response centers almost completely on their abilities as a craftsperson.

It's tempting to elevate one of these types above the other, but that would be completely unfair and arbitrary. It takes both types for modern society to function and thrive. Both types can make excellent business owners. Both types can become successful specialists who build wonderful market positions. And neither type is more likely to have a better or more satisfying career.

Now that said, entrepreneur types will usually be more flexible about how they structure their business, while craftspeople will likely be \emph{less} flexible. Entrepreneurs will be less constrained by the realities of the market, while craftspeople will sometimes be more constrained by those realities. This has real implications for how you choose to specialize.

If you're on the Craftsperson end of this spectrum, you face two constraints, and certain combinations of those constraints can present a very difficult obstacle to successfully specializing.

Let's start exploring this using this diagram, which I often use as a one-picture explanation of how specialization works.



The entrepreneur has a very large Stuff They Can Do circle, because they tend to be aware of how easily acquired skill is (either through renting the skill or learning it at a competent level just-in-time), so they'll tend to start with the Stuff Clients Need circle, figure out what clients value \emph{and} is rare on the supply side of the market and gravitate there. Thus, they'll define Stuff They Can Do based on Stuff Clients Will Pay Top Dollar For. They'll start on the left side of that diagram and move to the right in order to build a business model.

The craftsperson approaches things from right to left on this diagram. They tend to assume that Stuff They Can Do is fixed and difficult to change, and the Stuff They Can Do is constrained by their current skill as a developer, designer, artist, writer, etc. They can easily imagine increasing that skill (going deeper into it or expanding it), but they can less easily imagine building up an entirely different skill, especially one that has been dictated by marketplace demand.

This is so partly because for many craftspeople, their skill is part of their \emph{identity} as a person. It's part of how they see themselves as someone who is a contributing member of society.

So this is the first constraint craftspeople face. They're relatively attached to their current definition of Stuff They Can Do, and changing it can feel like a crisis of identity, or a major reinvention of self.

The second constraint is the market (this is the Stuff Clients Need circle on my diagram). You have even less control over what the market wants. In fact, it's healthiest to simply believe that you have \emph{zero} control over what the market wants. While it's not 100\% true, acting as if you have no control at all over what the market wants makes you a better small business owner. So yeah, let's go with that. You have \emph{zero} control over what the market wants, and that's just the way things are. All you can do is understand what the market currently wants or try to predict what it will want soon or predict what it wants, doesn’t know it wants, but would definitely want If you invented it for them.

There definitely are situations where for a given person, there is \emph{no overlap at all} between the Stuff Clients Need and the Stuff You Can do circles. In this situation, if you're completely inflexible about changing the size, shape, or position of the Stuff You Can Do circle, what you have is a standoff--an irresolvable conflict--and I can't imagine building a successful business under those conditions.

For the craftspeople reading this, I want to soften the above with a few really interesting examples of craftspeople who would seem to be in that standoff situation but have made it work.

Ross MacDonald specializes in making paper movie props. The copy of \emph{A Farewell to Arms} that Bradley Cooper's character in \emph{A Silver Linings Playbook} throws through the window? Ross MacDonald made that prop. The Pawnee Charter that appears at various places in \emph{Parks \& Rec}? MacDonald made that too, in addition to thousands of other paper movie props. This guy's an amazing craftsperson, and he's made specialization work for him.

Sean McCabe started his business with a surprisingly successful course on hand lettering. I don't know the guy at all, but he sure looks from the outside like an entrepreneurial craftsperson, because he's cultivated significant skill in multiple crafts (hand lettering, podcasting, writing, internet marketing) and turned each of those into courses that he sells and apparently makes good money selling.

The takeaway is this: there are weird and wonderful examples of craftspeople who have made a business out of their weird and specific skill. And maybe that could be you! But do be warned that your dedication to your craft can be both an asset in some specialization scenarios or it can be a liability that prevents you from specializing at all.

Finally, it’s worth noting that this analogy breaks down totally if you think of entrepreneurs whose craft is the business they are building. This dissolves any difference between these two types. The analogy is still valuable, this edge case notwithstanding.

Bringing this all back to risk: if you are pretty far towards the pure craftsperson end of the spectrum, you will perceive the entrepreneurial aspects of self-employment as relatively risky because they threaten your identity. For example, a ``pure entrepreneur'' would likely see all of the following as opportunities they could take advantage of. In fact, each of the changes I'll list below are a key part of a real, successful business model, an example of which I've indicated in parentheses:

\begin{itemize}
\item The commoditization of WordPress skill (WPCurve)
\item The commoditization of design skill (99designs)
\item The commoditization of writing skill (AudienceOps)
\item The absolute glut of people wanting to build a digital startup and the glut of overseas software development talent that's qualified to help (Rootstrap has productized the early stages of digital product design—from concept to wireframes+pitch deck—and leverages an offshore team to profitable offer development services)
\item The rise of tools like Squarespace that commoditize web design (Knapsack Creative and Worstofall Design use these tools to deliver “website/brand in a day” services)
\end{itemize}

For each of these changes in the marketplace there are craftspeople who actually suffer real financial damage, and there are entrepreneurs who leverage the exact same change into a great business. The different outcome is more a result of mindset than anything else. As I like to say: everything is an opportunity.

I want this book to provide a useful, detailed, nuanced \emph{framework} that helps you specialize and cultivate a great market position for your business, so I tend to avoid broad, absolute declarations. That said, if you are a pure craftsperson and feel inflexibly attached to that identity, your specialization options will be limited, and you may be better off pursuing other ways of improving your business. Partially productizing your craft is one such option.

\section{Risk tolerance vs. risk profile}

A brief sidebar on expertise. I often say this about expertise: \emph{ Specialization is a voluntary narrowing that is in service of a future wider impact. You subtract until your focus is narrow enough to penetrate into the hypodermis of a mystery, then you recruit the needed resources from any corner of the world to unravel that mystery and synthesize an effective solution.}

This is exactly what's happened with my very narrow (very ``meta'') focus on specialization. At the superficial layer of expertise (what I call the ``epidermis of expertise''), my thinking was ``Specialize, stupid! Just do it! It's great for everybody!''. I remained narrowly focused and reached the next layer of the problem: the ``dermis of expertise''. The problem become more complex, and required a more nuanced solution. Actually, the problem was \emph{always} more complex, it's just that I didn't realize this until I got past the epidermis of expertise.

The dermis of expertise is where you maintain the narrow focus, but you start casting your eyes towards the horizon, seeking additional helpful resources, be they skill, adjacent or non-adjacent expertise, or anything else that helps you solve the problem. And as you continue on to the deepest later of the problem (the ``hypodermis of expertise''), you start to recruit these other resources and incorporate them into your solution.

This is how experts transform an initial narrow focus into a complete solution to a difficult, complex problem. It's very much like a hypodermic needle. It's narrow and sharp to allow it to easily penetrate deep, but once deep in the skin it releases its contents which diffuse broadly into the surrounding tissue.

Again, this models how I became interested in risk, risk tolerance, and risk profile. I initially focused on how self-employed software developers can use specialization to their benefit, but ended up needing to learn about risk tolerance (to name just one non-adjacent area of expertise), and the best place I found for doing that--aside from some academic research on prospect theory--was the world of professional wealth management. Does this make me a non-specialist? Not at all! This is how specialization works. You go narrow so that you have more than a snowball's chance in hell of getting past the epidermis layer of the problem, and once you are past that layer you eagerly go broad in terms of \emph{how you solve the problem}.

This is why I see the ``specialist vs. polymath'' debate as a waste of time. It takes a rare special case (the truly successful polymath) and compares it against another rare special case (the irrelevant specialist). Both cases are edge cases that don't help much in terms of suggesting best practices that apply to more common cases. In other words, it's rare for a specialist to find an interesting, complex problem that lies at the dermis or hypodermis of expertise and resist going broad in search of a complete solution. Bonified polymaths have told me that they see themselves more as serial specialists. Anyway!

It's easy to think of risk solely in terms of risk tolerance. But the investment advice/wealth management world has a more nuanced take on risk, and this is beneficial to us here.

The key insight from the investment advice/wealth management world is to distinguish between someone's emotional and intellectual response to risk, and their ability to withstand a financial loss.

This actually gives us three ways to look at your relationship to risk:

\begin{itemize}
\item \textbf{Risk tolerance}: your emotional and intellectual response to risk.
\item \textbf{Risk capacity}: your ability to withstand a financial loss.
\item \textbf{Risk profile}: A synthesis of the above two, which defines your relationship to risk. This is the single most useful of these three ways of looking at risk.
\end{itemize}

So to work out a client's risk profile, I have developed a survey that explores the following areas:

1) \textbf{Your self-reported comfort with risk}. This is a useful ``bullshit detector'' when compared against more accurate indicators like behavior, so I'm always interested in seeing what kind of delta might exist between self-reported risk tolerance and measured risk tolerance. Note that my measurement is imperfect, but it's informative. TODO: data on this.

2) \textbf{Your past behavior with financial risk-taking}. What we say and what we do are often quite different, and behavior is a more accurate signal of risk tolerance. There are other forms of risk-taking, like risking social status by doing something your social circle doesn't accept or understand. But those are not easy to quantify, so I've stuck with what is easier to quantify and represents real risk-taking for most folks: spending money now under conditions of uncertainty about how that money might deliver a return on investment in the future and whether it might deliver a return at all.

3) \textbf{The length of your ``runway''}, defined as how long your business could survive without landing any new clients starting today, measured in months. This is not the only way to measure risk capacity, but it's a pretty good one. There are other forms of financial resilience beyond a savings account, but again those are less easy to quantify than ``months of burn in a savings account, line of credit, or both“.

4) \textbf{Your response to prospect theory questions}. Wikipedia does a pretty good job of defining this for us:

\begin{quote}
Prospect theory is a theory in cognitive psychology that describes the way people choose between probabilistic alternatives that involve risk, where the probabilities of outcomes are uncertain. The theory states that people make decisions based on the potential value of losses and gains rather than the final outcome, and that people evaluate these losses and gains using some heuristics. The model is descriptive: it tries to model real-life choices, rather than optimal decisions, as normative models do. The theory was created in 1979 and developed in 1992 by Daniel Kahneman and Amos Tversky as a psychologically more accurate description of decision making, compared to the expected utility theory. In the original formulation, the term prospect referred to a lottery. The paper ``Prospect Theory: An Analysis of Decision under Risk'' (1979) has been called a ``seminal paper in behavioral economics''.
\end{quote}

--source: \href{https://en.wikipedia.org/wiki/Prospect_theory}{https://en.wikipedia.org/wiki/Prospect\_theory}

Prospect theory is really about how we as individuals and as a group of human beings evaluate potential losses and potential gains. Is a gain considered just as desirable as avoiding a loss? If not, how do we emotionally weight gains compared to losses? Do we work harder to avoid losses than to seek gains?

Prospect theory is not a simplistic answer to these questions, but a general theory that seeks to explain how humans answer them. To make matters more complicated, the groundbreaking work on prospect theory by Daniel Kahneman--who won a Nobel Prize in Economics for his work—has not always replicated by subsequent studies (\href{https://www.apa.org/science/about/psa/2015/01/gains-losses.aspx}{https://www.apa.org/science/about/psa/2015/01/gains-losses.aspx}). All this to say this area of inquiry into how humans handle risk is imperfect and still developing, but prospect theory is still a useful tool that's more sophisticated than simply asking ``how do you respond to risk?''

5) \textbf{Your DiSC work personality profile}. David C. Baker has done some really great research into the relationship between entrepreneurial risk-taking and the 4 work personality characteristics the DiSC profile seeks to quantify. That's why I've started using this tool as part of my risk assessment work with clients. It casts a useful, nuanced light on the same issue the questions above are getting at, but from the perspective of personality rather than past behavior or risk/reward preferences.

Taken together, these five factors give a good picture of your risk profile which again is a combination of your emotional comfort with risk and your ability to withstand a financial loss.

At the end of this chapter I'll share a simplified tool you can self-administer to assess your own risk profile.

\section{Head Start vs. Find Your People/Opportunity vs. Entrepreneurial Thesis}

To complete your understanding of how risk fits into specializing, we need to dig into one more big idea, which we might think of as three different decision making heuristics. These are patterns I've noticed among the specialists I've interviewed for my podcast, and they form three different ways of answering the question: ``how do I decide where to specialize?''. These three decision making heuristics each present a different risk/reward tradeoff, so we'll move through them in order from lowest to highest risk.

Note: Chapter 12 will bring these three heuristics together into a single process you can self-administer, so as you're reading through these heuristics there's no need to try to \emph{implement} them until you get to Chapter 12.

Another note: These heuristics will portray 3 different decisions making styles, and it may seem that you should fit neatly and cleanly into only one of those 3 buckets. The reality is almost always more nuanced, and you may find yourself combining two of these approaches. That's totally normal, but I'll talk about them as if they are 3 things with distinct borders between them, even though they're really not.

\subsection{\textbf{Find Your Head Start}}

When you find your head start, instead of looking equally at all the factors that might make for a good specialization, you focus on one: where is your most significant head start?

\{TODO: remind folks what constitutes a head start, possible pull from ch10\}

You don't ignore those other factors. Here's what you actually do instead:

\begin{enumerate}
\item List every way of specializing that might be a fit for your business.
\item Rank each of those ways of specializing based on how much of a head start you have there, and how beneficial that head start would be.
\item Use other factors to eliminate options from your list. Other factors might include:


\begin{itemize}
\item The risk of specializing in this way
\item The low profitability of working with clients like those
\item A lack of interest on your part in doing work like this
\item Low demand from clients for this specialized offering
\item How deep your insight into the problem or the needs of the vertical goes
\end{itemize}
\end{enumerate}

At this point you have a shortlist of specialization options. They're all free from dealbreaker flaws, and they're ranked based on how beneficial your head start in each one is.

Now, if your shortlist is longer than 1 item, you still have a decision to make, but this simple decision making heuristic has hopefully supplied you with a shortlist of manageable size.

\subsection{\textbf{Find Your People/Opportunity}}

The next decision making heuristic is to ignore your head start and instead focus on a different single factor: what kind of clients do you prefer to serve, or what problems do you find most fascinating and compelling?

Again, you're not ignoring other factors, you're just using a two-pass rank-and-eliminate heuristic, where the first pass is to compile a list of interests. This time, it goes like this:

\begin{enumerate}
\item Make a list of kinds of clients you are confident you would enjoy working with. Big fat important note: you are \emph{not} allowed to list stuff like ``pays invoices on time'', ``innovative'', or ``lets me lead''. These are not ways of identifying clients that lead you to a useful specialization decision. Of course we all want to work with great clients and avoid bad ones, but remember specialization is a way to make your marketing work better and set you up to cultivate very valuable expertise, and having a focus on ``clients that pay invoices on time'' does neither of those.
\item Your list does not need to be limited to types of clients you have worked with in the past. That's why I call this heuristic Find Your People/Opportunity, because it's based on the idea that you'll volunteer to take on more risk so that you can seek out types of clients you may not have worked with before but for some reason really would like to. It's risky to do this, but if you can handle the risk it can be a way to improve your career.
\item Rank your list by how much you would like to work with each kind of client you have listed.
\item Then, as in the Find Your Head Start heuristic, eliminate options from your list based on other important factors. As you do this, remember that this list is, by its nature, a list of potentially risky ways for you to specialize. You really do need to have a combination of 1) a risk profile that allows you to handle a fair bit of risk, 2) patience, 3) discipline, and 4) a burning desire to do the work necessary to build a market position \emph{without} having a head start that gets you closer to that market position. Reasons to eliminate options might include:


\begin{itemize}
\item An insurmountable lack of access to buyers in this area of focus
\item No conceivable overlap between what this market values and what you are willing to offer
\end{itemize}
\end{enumerate}

This approach to deciding how to specialize is more risky. This is a good place to remind you that for all the power, reach, and flexibility that modern digital marketing gives us for generating leads, and for all the power that Google and LinkedIn give buyers for finding solutions they may not have been aware of before they searched, you're still in a \emph{relationship business}. Your ability to forge trusting relationships with buyers is the beating heart of your ability to run a thriving consulting business. This doesn't mean you need to be an extrovert or have world-class people skills. Lots of nerdy introverts, like myself, do it. So the risk this heuristic poses to you is that you could choose to specialize in a way where it's difficult or impractical for you to forge trusting relationships with enough buyers to make it work.

If you're using the Find Your Head Start heuristic, you're probably going to specialize in a vertical or problem you have past experience with. You'll know what to expect. You may already have a useful amount of access to buyers or credibility. With the Find Your People/Opportunity heuristic, you may be moving into unfamiliar territory with surprises in store for you. The solution, of course, is to do some market research to make the new territory less unfamiliar.

This is also a good place to remind you that I see the kind of people who read a book like this as way above average in terms of creativity, resourcefulness, and willingness to learn and change. My clients are all like this, and it's a never-ending joy to work with people with this kind of drive and growth mindset. So if you have a fire in your belly to Find Your People/Opportunity and the risk profile to handle it, I believe you can build the access and trust you need, perhaps in less time than it takes to get a masters degree.

\subsection{\textbf{Entrepreneurial Thesis}}

The final decision making heuristic I've seen out in the wild is what I call the Entrepreneurial Thesis. This is actually a variation of the Find Your People/Opportunity approach, but it's better suited to someone who is more mercenary or entrepreneurial in their overall orientation to risk \& opportunity. This mercenary or entrepreneurial orientation is often correlated with a reduced need for craftsman-like enjoyment of the work. Or rather, the craftsmanship shows up in the design of the business rather than the creation of deliverables.

Instead of looking for what kind of clients you prefer to serve, or what problems you find most fascinating and compelling, you are looking for the opportunity that seems like the best bridge from your current position to a significant entrepreneurial opportunity.

I've interviewed several business owners who deployed this decision making heuristic when they decided how to specialize. They speak about interest and enjoyment in distinctly different ways than people who gravitate towards the other two decision making heuristics. Here are some verbatim quotes that illustrate this point:

\begin{itemize}
\item TODO: look up the Elliot interview


\begin{itemize}
\item TODO: look up the Matt Rogish interview
\end{itemize}
\end{itemize}

Notice how Elliot and Matt both talk about the opportunity they chose to focus on in terms of its \emph{business potential} first. In fact, Elliot goes so far to compare the opportunity he pursued to janitorial work. Dirty, smelly, and mundane. That's how most of us would think of janitorial work, and it's exactly those repellant characteristics that attracted Elliot to this opportunity! He saw the inherent un-sexyness of the work as a ``moat'' that would deter most competition, and one that would also tend to keep his clients from adding this expertise in-house and therefore ensure long-term demand for an outsourced solution that Elliot could provide.

So the Entrepreneurial Thesis heuristic is more like the Find Your Head Start heuristic and much less like the Find Your People/Opportunity heuristic in that the \emph{potential business upside of the opportunity} is the most important factor in how you decide to specialize, and factors like enjoyment of the people or enjoyment of the work are much less important.

In fact, you may have a head start that you can leverage into an entrepreneurial endeavor. Matt Rogish is a good example of this. His strongest head start was in DevOps work, and the cost and pain of migrating away from Heroku created the opportunity for him to build a team that could pursue that opportunity. Does he dislike DevOps work? Not at all. But every aspect of how he pursued the opportunity he saw to deliver DevOps as a service was done in an entrepreneurial, decouple-time-from-revenue way.

There's one more subtle distinction I want to talk about here, because it's especially relevant when pursuing an Entrepreneurial Thesis.

It has to do with the idea of putting your market first.

As n00b freelancers, lots of us do the opposite of a market-first approach to our businesses. We start with some skills that we probably developed while working a job, and we launch into self-employment with this skillset as our focus. We define ourselves or our value proposition based on this skillset.

Eventually, some of us arrive at the market-first orientation after the frustration with the ceiling that a skills-first orientation puts on our career. We start to wonder if we could generate more revenue or create more impact with a different approach.

The answer is yes. And the different approach is a market-first approach.

A market-first approach starts with a focus on a market, either defined vertically or horizontally. We focus on that market, and then do whatever we can to literally fall in love with it. There's really no better term for developing the empathy and the intimacy with the market. This love affair leads you to put the needs of the market first, and as you better and better understand those needs, you better understand where there might be value in addressing certain of those needs. This deeper understanding of what the market needs and values leads you to re-focus or even re-invent your services in a way that targets the more valuable needs the market has.

Taking this approach all the way may lead you to an Entrepreneurial Thesis. I have a coaching client doing just that with a need that she and a few other see around the problem of software contributing to physician burnout. What makes this an Entrepreneurial Thesis is that there's no competition that validates that others can make this kind of focus work. She's a pioneer. Things can work out \emph{very well} for smart pioneers. Among their ranks are companies like AirBnB and Uber and other ``unicorns''. But it is risky! Taking a market-first approach is the best way to de-risk an Entrepreneurial Thesis. The other approach of dreaming up an idea, skipping a thorough a half-assed validation phase, and moving straight to execution is a surefire recipe for failure. Much better to let your market generate ideas for you.

Again, I've listed these three decision making heuristics in order from lowest to highest risk. In Chapter 12 I'll bring these concepts of risk profile and decision making style together into a simple process you can follow to get closer to deciding how your business could specialize and get started building a great market position. So that means we need to do one more thing in this chapter, which is get a handle on your risk profile.

\section{Risk Self-Assessment}

TODO: build a version of my current risk assessment that doesn't include the DiSC profile and automate it so that it calculates risk score and sends that back to folks who take it, and link book readers to that version.

\section{Chapter Summary}